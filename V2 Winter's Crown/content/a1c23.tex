\chapter{Ludmila Zahradnik}

Great, Ludmila thought to herself.

 

What she had initially believed to be some sort of audience regarding the business Lord Cocytus had for her suddenly turned into a test of strength to determine her suitability as a leader of Demihumans. She didn’t even have any particular desire to lead said Demihumans, yet the imposition enacted by Lord Cocytus only offered two options: take them as her own subjects, or have an enclave under a separate line of authority arbitrarily pop up in the middle of her territory.

 

She glanced to the side: Lady Shalltear showed no adverse reaction to such a thing happening in what was her own demesne by extension, so Ludmila was left to her own devices. Scanning her surroundings – over the expanse of fish farms and the forest with the Lizardman town beyond – she saw that a small crowd was assembling. Perhaps these Lizardmen knew what was going to happen all along.

 

“Were you aware of this, my lady?” She asked Lady Shalltear.

 

“I had absolutely no idea,” she replied. “I told you that you’re closer to him in disposition than I. The proof of strength thing sounds just like what you said about Demihumans, yes?”

 

“I would not say it is just as I said,” Ludmila sighed. “I definitely did not expect to come here and end up with a fight on my hands.”

 

“Well, keep in mind that they’re subjects of the Sorcerous Kingdom as well.” Lady Shalltear replied lightly, “Resurrections are costly, so don’t have too much fun.”

 

A distance across from her, the Lizardman appeared to be readying himself. He had a few unknown adornments which had the appearance of accessories, but whether they were magical or simply decorations she could not tell. A tall, hide shield was fastened onto his left arm, and he sported a crude vest fashioned from layers of the same material. He pulled a gleaming scimitar from the sheath at his side, working it in smooth motions that caught the sunlight with its polished surface as he warmed up. Ludmila glowered dubiously at the weapon: how many strips would her dress be sliced into by the end of the match?

 

Looking to the side hesitantly at her lady’s maid, she found that rather than share Ludmila’s sense of trepidation over the safety of her new dress, Aemilia’s fists were balled up tightly; an excited expression painted all over her face. It was apparent that her side was entirely bloodthirsty – she wasn’t sure if Lady Shalltear or Aemilia was more so.

 

With a quiet sigh, Ludmila stepped forward. More Lizardmen continued to trickle in, forming in a loose line of curious spectators that grew to encircle them some distance away. Lord Cocytus studied her, head shifting slightly.

 

“Your weapon?” He asked.

 

“My apologies, Lord Cocytus,” Ludmila lowered her head. “I was unaware that this audience would include a duel.”

 

“Oh!” Lady Shalltear piped up, “I have this one I’ve picked ou–”

 

“Shalltear,” Lord Cocytus said. “Weapons you can wield would be too powerful.”

 

“Mmh…I can’t say that’s wrong, but, while she’s my vassal, she’s not going to be running around biting people. Probably.”

 

Lord Cocytus swept his halberd out over the gathered Lizardmen.

 

“Choose,” he told Ludmila.

 

She scanned the small, but growing, crowd again. Half held weapons or wore them, but they all appeared to be fashioned out of the materials available to work with locally. None had anything resembling the masterfully crafted blade wielded by her opponent. Her first thought was to mirror his armament somewhat, but all the shields that she saw were the same: made to be held by Lizardmen who looked well over two metres from the tip of their noses to the ends of their tails. The short clubs of bone and wood, stone axes and flint daggers did not look very promising against an opponent so much larger than she was, either.

 

“I guess that spear will have to do,” she said after comparing a few polearms held in the crowd around them.

 

Looking askance at her, then glancing to Lord Cocytus, the Lizardman came forward after the Lord of the Great Lake nodded. It looked down at it hesitantly before handing it over. There was not much to say about the weapon: a two-and-a-half-metre long shaft of Ash, which was mostly straight. There was a stone blade slotted into the end and bound in place with lengths of rough rope. She hoped it was sufficient to defend against the scimitar with. Turning away briefly, she tested its balance and weight while she thought on what she could do.

 

Gauntlets, nope. Bracers, nope. Boots and greaves, nope. Helmet and gorget, nope.

 

Lacking anything but her civilian attire, she felt perfectly ready for a stroll through pastoral farmland. Ludmila sighed again, frowning to herself as she mulled over how things had come to this. After a moment’s thought, she gave her head a shake.

 

According to Lord Cocytus, that the Lizardmen find new places to live and learn was not his idea, but the will of the Sorcerer King. A supreme sovereign demanded excellence from his subjects and, as one of the Sorcerous Kingdom’s nobles, it was her duty to deliver. It was not an imposition – it was obligation: to render His Majesty’s will into reality. The first step would be to prove that she was suitable for the role.

 

Turning back, she saw Lord Cocytus look to the both of them briefly before stepping away to watch from the side. The Lizardman's expression seemed to pick up when it was apparent she had no further preparations, and his thick tail slapped the wet ground twice. Ludmila realized what the strange sound back at the statue finally was, and she took a mental inventory of her opposite number.

 

Ludmila guessed that he was well over half again her weight, not quite standing as tall as she. Though Lizardman limbs appeared shorter relative to Human ones, he still had a powerfully built upper body and his legs and tail were similarly so. Russet scales covered his back and sides, and she guessed that they provided ample protection on top of his other equipment. The tall shield was positioned a bit strangely: along the length of his body rather than upright, probably due to its leaning posture and long torso. The odd placement – for a Human, anyways – was something that suggested a possible advantage in her favour.

 

The shore where they were fighting was wide and without obstacles. Her first thought was to lead the fight into the trees nearby, where she would be able to exploit her advantages as a Ranger and the Lizardman’s probable weaknesses given where their homes were built, but she wasn’t sure if she was allowed to do that in this duel.

 

“Are we restricted to this location, Lord Cocytus?” Ludmila asked.

 

The Lizardman started at her question: he probably knew why she had asked. Lord Cocytus did not miss his alarm.

 

“Hmm…within the spectators,” Lord Cocytus answered. “On the shore. No terrain advantages.”

 

She was unfamiliar with Lizardmen, but judging by her observations of their posture and gait as they moved about the settlement, she would still have the advantage in reach and agility where they were. While he registered to her senses as being a bit weaker than herself overall, he also clearly had advantages over her both in his natural traits and equipment.

 

Considering his mass and size, as well as the large shield and superior weapon, closing the distance between them was the most obvious and safest option on the Lizardman’s part. She would have to fight defensively, keeping him at bay until she could figure out his probable range of movement and find a way to strike decisively past his shield. Or she could probably tire him out…but that probably wasn’t the point of the contest.

 

Facing forward, several moments passed before Ludmila raised her spear into a field guard, leading with her left hand. She set her face into an expressionless mask, but she thought it would be nice to have at least some fingers left after the fight.

 

Seeing both combatants take their stances, Lord Cocytus let out a blast of frigid air.

 

“Begin!”

 

Ludmila tensed, ready to move at a moment’s notice, but the Lizardman did not immediately explode into action as she had expected. It was puzzling, but this suited her just fine as they took their time gauging one another. He advanced several paces and she drifted back to maintain a four metre gap between them. He looked over with some consternation, peering at her skirts, until the spearhead darted forward and glanced off of his snout. She recentred herself as he recoiled from the blow.

 

Between her stance and the length of the spear, she could easily strike him, but he could not even reach half the distance. In addition, it seemed that her opponent was having a harder time reading her than she was him, and her skirts were partially concealing her footwork. Maybe he was mistaking her dress for armour as well, considering the vaguely militant appearance that resulted from Aemilia’s modifications.

 

A dim clamour rose around them after the sudden strike. With the sudden awareness that the spear could move far faster than he could react, the Lizardman assumed a more conservative stance. Unlike a Human, however, he did not raise his shield – instead lowering his head, which already curved naturally into a slouch, to be in line with the shield that was positioned lengthwise with his long body. This made it extraordinarily hard for her to score any solid blows. The stone spear wasn’t able penetrate his coat of thick upper scales even with that fairly strong thrust, and his underbelly was at an angle with the ground and entirely covered by the shield.

 

His movements appeared to become more committed in the new stance as well: Ludmila surmised that it was effective against the weapons that the tribes of the Great Lake used, giving him the sense of confidence which was relayed in his movements. She did not have a good read for his more subtle body language, but, based on how impossible it seemed to injure the Lizardman now, he was a veteran of some sort.

 

No…that wasn’t right; he entirely lacked the scarring commonly found on Demihumans who fought with enemy tribes. It was probably the result of training, and he had fallen into a familiar stance well suited to sparring with his peers. A young warrior, somewhat similar to herself – this was probably the basis of Lord Cocytus’ selection.

 

She maneuvered to gain different angles on the Lizardman while continuing to test his defences, trying to determine what he could do from his position. The spearhead darted in and hammered into the bottom of the shield; the next strike came in a half second later, glancing off of the Lizardman’s shoulder through the opening she had created. His torso jerked roughly but there was still no apparent damage.

 

He didn’t seem to like the last hit to his shoulder. As the spear withdrew, his late attempt to parry abruptly changed direction and the scimitar effortlessly sliced along the front end of the spear. Their gazes both followed the long curl of shaved wood as it fell to the ground, and a strangled sound came from the side – probably from the Lizardman that had lent her its weapon.

 

Not only was the scimitar of masterwork quality; it was almost certainly enchanted as well…and even the gathered spectators knew beforehand. A part of her cried foul: why was she stuck using a stone spear against a magic weapon? Her internal complaints turned dark as the Lizardman let out a strange sort of hissing chuckle, and the spectators clamoured all the louder. She honed her fighting edge by another degree, mentally shoving the distraction aside.

 

The Lizardman continued to hunker down behind his shield, scimitar occasionally flicking out in continued attempts to catch the darting spear. He appeared to have come to terms with taking hits on his upper side, keeping his shield tucked in to defend his face and belly. The purpose of the defensive tactic was clear: she needed to end the fight quickly – or at least change the flow of the fight – or she would eventually be left with a pile of wood shavings.

 

Based on their series of exchanges, Ludmila believed that she had a fair grasp on her opponent’s capabilities. While he held his current stance, she couldn’t really hurt him. She needed some way to break it. His maneuverability in the low posture was poor, but it would still enable him to bull his way forward powerfully with his compact legs. Closing distance was the sensible course of action when a Human shield user was faced with a two-handed spear. The problem was whether it also made sense in Lizardman combat, and whether she could goad him into doing so to dislodge him from his defensive stance.

 

The spear lanced forward again, striking the edge of the shield in front of his face. Ludmila dipped her weapon under the expected retaliation, bringing the spear back up and jabbing him just beneath an eye as the late parry shifted his posture. Once again, the stone point scored no apparent injury. She watched his tail waver slightly after the blow, but his stance remained unchanged – Demihumans with tails tended to have superior balance in their natural postures, and the Lizardman seemed no different in this regard. Combined with his patient, methodical, approach to the fight, it was quite difficult to deal with.

 

Why was he so defensive with all of his advantages? Unnecessarily letting an opponent measure one out wasn’t very wise – he should be resolving the fight before any of his own weaknesses could be identified and exploited. Maybe Lizardmen just had that sort of disposition: favouring drawn-out defensive tactics over aggressive ones that would resolve combat quickly. Most Demihumans did leverage their natural advantages in combat, after all. She had not scored any wounding blows, so she didn’t even know if they could regenerate to help make up her mind on the matter.

 

Ludmila circled around to the right, and his feet shuffled over the ground as he pivoted to face her. It looked awkward, compared to performing the action in a fully upright position. The spear poked at his feet a few times to see what sort of reaction she would get, then jabbed him in the tail when she noticed it trailing behind his turn. His scimitar was on the far side of his body throughout, and a muffled hissing sound issued from behind the shield. She had seen enough.

 

She stopped circling and planted herself solidly, and he charged her with shield before him; gleaming blade raised. Rather than directly giving way to his rush, she punched at his shield with the stone point while leading him around in a sharp, counterclockwise path. It had the effect of bleeding away his initial momentum, and he was forced to maneuver awkwardly again with his weapon hand on the wrong side of his body. When their steps almost slowed to a stop, Ludmila switched her grip to lead with her right and committed to her assault.

 

Feet flowing beneath her, she now circled clockwise with the spear positioned to attack his exposed lower right side. It gave the impression of her somehow significantly outpacing his attempts to bring his shield into play again, and the rapid sequence of jabs over the Lizardman’s body eventually led to it desperately flailing its scimitar in a vain attempt to keep up. Gashes appeared over his underside. Crimson blood traced over scales. The enchanted blade took another chunk out of the spear. She needed to collapse his form and finish him off.

 

Switching her grip again, she reversed her maneuver. Circling back into the Lizardman’s continued turn while he was still focused on trying to deal with her flurry of attacks, Ludmila entered the blind spot provided by his shield. Her spear thrust down between his legs. She drove her foot into the shield, pressing forward with her full weight. The Lizardman fell awkwardly in a heap. She couldn’t let him recover.

 

Ludmila came forward with her spear, but he scrambled away as well. His tail lashed before her skirts; she stomped it to the ground. The Lizardman jerked to a stop and howled in pain. She drove her weapon down at the base of his skull.

 

“「Evasion」!” The Lizardman cried.

 

His head and shoulders blurred, somehow managing to move out of the way. The spear blade buried itself two-thirds of the way into the sandy soil. His desperate defence confirmed another suspicion: that the spear could get under his scales from her position. With her boot still planted on his tail, Ludmila leaned back as she withdrew the spear to put him down for good.

 

The Lizardman followed the stone point with a wide eye as the spear was extracted from the ground, and he twisted desperately to get his shield between them. She casually batted the poorly-structured attempt at defence back down to the side and drove the blade forward again.

 

“Ludmila.”

 

The spear stopped. The sound of the waters lapping on the shore returned.

 

The Lizardman’s head was turned to the side, jaws open. Inside his mouth was the blade of the spear: it had found his long, pink tongue – pinning it to the ground. The scimitar was on the ground a few metres away. Pained choking issued weakly from beneath her as bright crimson blood flowed onto the ground. The spectators stood paralyzed: tails pointed stiffly at the air; mouths hanging open.

 

“I believe the outcome is clear.”

 

Again, Lady Shalltear’s voice drifted from behind her. Ludmila let out a breath.

 

She yanked the spear from the Lizardman’s mouth and removed her heel from his twisted tail. His screams of agony echoed over the shore. He rolled over on the ground, curling up as he cradled his head in his hands. Lord Cocytus came forward again with another, smaller Lizardman, adorned in some sort of tribal mystic’s garb over its olive scales. He nodded, and it leaned down to tend to the wounded combatant.

 

A few minutes later, the Lizardman returned to his feet. Lord Cocytus spoke.

 

“A blade. Rings. Necklace. Bracelets. Anklets. Magic items: prizes, for accomplishment. Gifts for a promising new warrior.”

 

A blast of cold air followed. The nearby Lizardmen shivered.

 

“Yet, when your mettle was put to the test, in place of honor; in place of respect: hubris. Her name, you knew: yet your own you did not offer. Looked down; laughed at her weapon. Complacency. Conceit.”

 

Lord Cocytus turned his gaze to Ludmila, tilting his head down to look over her.

 

“That ring,” he asked. “What is it?”

 

“A Ring of Mental Fortitude, Lord Cocytus.”

 

“The rest?”

 

“Casual dress, my lord – nothing magical.”

 

Lord Cocytus shook his head slowly, rubbing it briefly with one of his clawed hands. He spoke once again, loudly enough to be heard by the Lizardmen all around them.

 

“You Lizardmen: valour, against impossible odds – in nine months…have you forgotten? Is this what my ways have wrought? Honour. Valour. Respect. Lost. Last month, insolence before Lord Ainz; now, this shameful display.”

 

They lowered their heads at his reprimand, and there was no sign of the excitement or tension that came before. Another blast frosted the air, caking the blood-soaked ground in a thin film of pink ice.

 

“I thought to rule with tolerance; benevolence. An open hand – to encourage and reward – for fealty, service and accomplishment. Yet, now, by one who has nothing, I am reminded of what must come before. Without duty, all service is shallow. Without conviction, will is weak. Perhaps, in my zeal to produce results, I have committed an injustice to the memory of the past. Too open-handed. Too many nice things. The root of decadence…has wormed its way in.”

 

He paused to look around himself, looking over the Great Lake and the settlements along its shores. Then he looked to the southeast, to some place not apparent to her.

 

“To rule, without employing terror: Lord Ainz’s command to me. But Shalltear has received no such command; this vassal, undoubtedly hers. You Lizardmen have become brazen, even to Lord Ainz. Respect and fear, forgotten: you must learn again.”

 

Lord Cocytus turned his head back down to her, but his voice still sounded over the spectators around them.

 

“You have proven your mettle, Baroness Zahradnik,” he told her. “Their lives are yours.”

 

Ludmila planted the base of the mangled spear into the ground and lowered her head in a curtsey.

 

“I am unfamiliar with their kind,” she said solemnly, “but I will do what I can to carry out His Majesty’s Will for them, Lord Cocytus.”

 

With a slight nod and a short grunt, Lord Cocytus turned and strode away with his insectoid escorts. The spectators parted for him as he made his way back to the village. The crowd dispersed – save for the six Lizardmen who had followed them around since their arrival. Lady Shalltear and Aemilia came to where she and they stood, looking down at the still-frozen traces of blood.

 

“While I appreciate the fact that you answer directly to me,” Lady Shalltear said, “when a member of the King’s Cabinet says something, it would be prudent to heed them.”

 

Ludmila turned to face Lady Shalltear with an uncertain expression.

 

“My Lady?”

 

“You stood down at my word, yes?” Lady Shalltear asked.

 

“Yes, my lady,” Ludmila answered.

 

“Before you ceased that Lizardman’s terrified wriggling so amusingly at the end,” Lady Shalltear explained with a grin, “he was screaming and wailing quite pitifully. Lord Cocytus called the end of the match when he saw that the Lizardman was clearly broken.”

 

Ludmila furrowed her brow at her liege’s words and turned her head to look at the half dozen Lizardmen. They flinched under her gaze, and the one in question stumbled backwards slightly.

 

“Is that so, my lady?” Ludmila replied, “I hadn’t noticed.”

