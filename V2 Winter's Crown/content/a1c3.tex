\chapter{Ludmila Zahradnik}

Ludmila was on her fourth cup of tea by the end of the second hour, and she was beginning to think that drinking so much hadn’t been such a good idea. Shifting slightly in her seat, she kept a straight face as she continued to address the concerns of her subjects from behind her desk. It was a duty she had assisted in since she was fourteen but, after recent events, she felt amazed at just how mundane it now felt by comparison.

 

“We want to paint the Skeletons, m’lady,” the housewife standing before the desk said.

 

“You want to paint them?” Ludmila blinked, “I suppose it might make for a colourful sight…can they even be painted, Nonna?”

 

“Such…measures would only be temporary,” the Elder Lich replied from behind her shoulder. “They would disappear when repairing damage or simply weather away over time.”

 

The vision of colourful Skeletons working all over her fief faded. Ludmila turned her attention back to the stocky, middle-aged woman.

 

“Was there any particular reason why you wanted to do this?” She asked.

 

“Ah, yes, m’lady,” the housewife replied. “With all the Skeletons around, we’re already starting to mix them up. If more’s to come, it’ll be a mess.”

 

It was something she had wondered about, and the answer seemed to have finally arrived. With some testing between her maids, it was determined that the Undead labourers somehow understood who was supposed to be controlling them and the overall hierarchy of authority. They would follow parents over their children, shop owners over their employees, Ludmila’s Household over her tenants, and so on.

 

The people, however, had no such sense unless what they were doing was distinctly different or separate from the others. While one couldn’t command the Undead labourers of others if they were at the same level of authority, trying to figure out which ones belonged to who in an area full of identical Skeletons was probably quite the hindrance.

 

Roughly seventy percent of her fief’s ‘population’ was Undead at the moment. Every home had also been assigned a single Skeleton labourer: as a means for every family to become familiar with the presence of the Undead and how to use them for various tasks. As with the first batch of immigrants, the adoption process was much the same and every household eventually saw their trusty skeletal assistant as an inseparable part of their daily lives.

 

The lease rates for various types of Undead labour were still in flux as the central administration continued to collect data on their overall productivity, but Clara had put forward the notion with the Royal Court that household Skeletal labourers should be provided freely as a means to familiarize the citizens and future generations on their use. Deliberations were still ongoing, and a decision was not expected until a more comprehensive picture on their overall value was developed.

 

“Since you will have to repaint them eventually anyways,” Ludmila told the housewife, “keep the markings simple and proper. If I see something strange painted on one of them…”

 

“Of course m’lady,” the woman nodded in agreement, “we’ll try to keep things in line; I’ll let the others know.”

 

With a swirl of heavy skirts, the petitioner turned and left the Hall. A wiry young man in a rough labourer’s outfit appeared to replace her, taking off his brown worker’s cap and placing it over his chest.

 

“A-afternoon, Baroness,” he bowed, “I, er, uh…”

 

“Is something the matter?”

 

Ludmila looked down at the list of petitioners provided by Nonna. The man was a journeyman carpenter; currently tasked with fashioning furniture for the new village. He and his family seemed to have adjusted well, and there were no reported issues surrounding them.

 

“That is, er – no, nothing at all,” he said.

 

She patiently waited through his tongue-tied response. They looked across the desk at one another for several seconds before he seemed to realize that Ludmila was waiting for him to continue speaking.

 

“Oh! Uh…yes, about the instructions your Lich handed down to us about the feast day preparations: it said to prepare tables that could be reused for the market there, but there’s no mention about style or appearance.”

 

“The weavers will be making some coverings for the occasion,” Ludmila answered, “so you don’t need to worry overly much about appearance. Though their first use will be for the feast day celebration, their main use will be as tables for market stands. Keep them simple, functional and durable: if the merchants that come by want something more, well, then I suppose they’ll have to order it from the local Carpenters.”

 

“I understand, Baroness,” he bowed slightly. “Back to work then – I’ll leave the measurements with the weaver here in the harbour.”

 

The Carpenter scurried out of the hall. Upon crossing the Squire Zombie, he gave a startled shout and jumped backwards, bouncing off of the Death Knight on the other side of the door. Didn’t he notice it on the way in?

 

Ludmila waited for the next appointment. The door to the hall opened but, rather than another petitioner, Wiluvien entered the manor. Ludmila glanced down to the papers on her desk: there should have been at least two more.

 

“Did something happen to the others?” She asked her chambermaid.

 

“It seems that they’ve already migrated to the farming village, my lady,” Wiluvien replied. “One yesterday, the other this morning.”

 

Ludmila looked over her shoulder to Nonna.

 

“I will add it to the items to be cross-referenced from this point forward,” the Elder Lich said.

 

Ludmila looked out of the window beside her: it was still mid-afternoon, and it appeared that she had seen to the questions of the harbour’s current residents. The sight of the two makeshift tents down on the flats near the timber yard reminded her that she still needed to visit the unasked-for migrants. It occurred to her just then that she should have asked her petitioners about it when she had them. She would have to see about their mysterious appearance after looking over several other things in the village.

 

“How is your family doing, Wiluvien?” Ludmila turned back to address her maid.

 

“Lluluvien and I are most pleased to be here, my lady,” she lowered her head. “Your lands are so beautiful; we far prefer it to E-Rantel. There is less work here for us at the moment, but it seems only a matter of time before things grow to the point that we have more things to do. As for our mother…she is not much changed from when we recovered her from Fassett County. Her health visibly improves, but she is still listless: we do not fear for her life, but recovery may take time.”

 

“I cannot even begin to imagine everything she’s been through, so she may have all the time that she needs,” Ludmila said. “You may return to your home – Aemilia will send for assistance if it is required.”

 

She looked back down to the list of petitioners as Wiluvien curtsied and withdrew from the Hall. While the unexpected end to her audiences for the day was the result of an oversight rather than an actual error, it still made her wonder where the limits of the Undead administrator lay. It had certainly not been outlined in the manuals. When she had initially encountered them, Ludmila’s thoughts were that they were simply unfamiliar with the specifics in management of Human lands. After several weeks, however, she decided that they were simply not experienced with administrative matters at all until recently.

 

“Nonna,” she said, “is the Elder Lich ordered for the new Lichtower sufficient to manage the civic administration of the new village in its entirety?”

 

“Barring oversights such as the one which was just identified and situations without precedent,” Nonna replied, “administrative anomalies should be next to nonexistent. Management of the civilian administration, logistics and the defence of the constituent territory of each Lichtower are all well within the capacity of a single Elder Lich as far as my experience here has suggested.”

 

“We have barely scratched the surface of what needs to be done here, so there is still much to learn. Can we expect every administrator dispatched to us to perform accordingly?”

 

“Though there should be practically no difference in the capabilities of my peers, who are all summoned by His Majesty, there is also no precedent to suggest that it may be the case…”

 

Ludmila shifted her seat around to face her aide. Though she could still not get a clear read on her Undead features, Nonna appeared to be deep in thought.

 

“If you know of any problems that may arise…” Ludmila prompted.

 

“Personally, I do not believe there will be any issues,” Nonna said, “but the data collected so far from the other territories is…unreliable, so this assessment is at best founded on conjecture.”

 

“Unreliable? Have there been problems in the other territories?”

 

“Adoption of Undead servitors has been embraced to various degrees,” Nonna replied, “due to efforts to balance Human and Undead labour pools, as well as factors involving religious practices and cultural perceptions. Each territory has enacted its own approach based on their specific circumstances, so creating standards for even basic skeletal labour has been a challenge.

 

“The Elder Lich administrators have also been presented with their own difficulties. If I were to describe it simply, they have been…injected into existing systems of Human administration, and each Noble and their vassals have personal styles of rule that are non-uniform quality. Only Corelyn County has appeared to achieve perfectly seamless integration between new and existing systems, which were already at a high standard. The methods there are still distinctly different from your own approach, which has optimized its labour, administration and all aspects of its planning around the availability of Undead servitors.”

 

“So I guess it really was a boon to be able to start from nothing,” Ludmila said somewhat ruefully.

 

“It appears to be the case, yes. The only locations that have been able to roughly mirror your early progress have been the recently resettled areas of the crown lands that were razed last spring.”

 

Ludmila made a face. The various elements of development within her demesne were supposed to give her an edge in attracting migrants. Though it was presumably for the greater good of the realm, the notion that her ideas would be used elsewhere before she could reap their benefits made at least a small part of her want to cry foul.

 

“Does that mean that villages like the one under construction here are going to be popping up all over the territories directly managed by the crown?”

 

“While the Lichtowers and their fortified villages have received a positive review,” Nonna said, “it was determined that the resources are not readily available to convert the hundreds of settlements in the crown territories into such strongholds. Materials found here are impractical to source and not worth the expense of importing. Broader defensive arrangements have been devised instead, using elements from His Majesty’s armies patrolling those lands.”

 

“Then what did they use from the information collected here?” Ludmila asked.

 

“They primarily revolve around the findings made from Undead labour for farming and forestry, as well as the use of Undead Beasts for light transportation on the road networks. Most territories do not have major obstacles such as the marshes in yours, so Bone Vultures are notably absent as well. The villages are managed by Human chiefs, who in turn report to the Undead administrator assigned to oversee their constituency.”

 

“I see…” Ludmila placed her chin on the back of her hand, “do you think that I went overboard with the design of these villages, then?”

 

“The goals of your development and the purpose of the lands directly administered by the crown fundamentally differ,” Nonna said. “While your goal is the rapid development of an urban centre, the crown is primarily concerned in the production of basic commodities. They will be allowing the excess population to flow to the city through traditional mechanisms, as improvements to E-Rantel are still ongoing.”

 

“The House of Lords has been issued no such directive,” Ludmila frowned, “should I be making considerations for this commodity production? I’ve been entirely focused on laying the foundations for advanced industries rather than expanding basic ones.”

 

“There has been no indication of such a policy being drafted. As long as the lease for His Majesty’s Undead servitors is maintained, the central administration should have no issues with their current usage. As Lady Shalltear is your liege, the matter of your taxes is also something you will need to take up with her – they are not the purview of the crown.”

 

The last part Ludmila understood as such, but it was nice to have confirmed. She rose from her seat, pushing the chair back under its simple, wooden desk and stretched away the hours of her afternoon audience. Aemilia began to move seeing this, preparing for an outing.

 

“Then everything is in order for now, I suppose,” Ludmila said, “unless there is something else you can think of. When the new administrator arrives, I will be relying on you to familiarize it with how we do things here. Each Lichtower will report to you, as well as the office that will eventually be set up in the port town here. This is markedly different from the old systems that I am used to: a member of the gentry or a village chief does not administer this many people, nor anywhere near this degree of productivity, so we will both be learning how to make these new administrative structures work.”

 

While Ludmila was quite excited by all the new prospects presented by the advancement of her fief, Nonna displayed the same dispassionate expression as always.

 

“I’m curious,” Ludmila said. “Are all of the Elder Liches in His Majesty’s administration kept as well-apprised of the state of the realm as you are?”

 

“Those working directly for the central administration may be, yes,” Nonna replied, “but probably not as well as myself. Due to the contributions that I have made to the administration in my time here, as well as the progress made in this fief, I have become a sort of...consultant for my peers. My feedback is valued throughout the realm.”

 

“Do you mean to say that all of this information about what is going on elsewhere is due to the other Elder Liches contacting you to complain?”

 

“None would consider complaining in their service to His Majesty,” Nonna replied. “Even the smallest contributions have value, and no effort is considered in vain. The correspondences are for practical purposes…though if one were to use such measures, it could be said that they are envious of my position here.”

 

“Two point five percent more envious?” Ludmila smirked.

 

“That should be the figure listed on the most recent assessment, yes.” Nonna said, “But, at this juncture, I would say what is presented in numbers belies what occurs in practice.”