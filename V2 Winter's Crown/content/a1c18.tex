\chapter{Ludmila Zahradnik}

“We are four hundred,” the Volkhv said, “and fifty.”

 

“So…four hundred fifty?”

 

The sun had nearly crossed below the western ridge of the valley, granting a reprieve from the sweltering heat of the day. Gusts of cool evening wind played over the fields, and Ludmila’s skirts occasionally whipped forward as she and the Volkhv continued on their way towards the farming village.

 

“Four hundred and fifty would be more precise,” the Volkhv explained. “The four hundred are like me – Druids who live in the alpine meadows and amongst the craggy peaks. The fifty are Rangers who stalk below.”

 

“Four hundred Druids?” Ludmila’s words rolled out in near disbelief, “What about the rest?”

 

Magic casters were relatively rare, and the current shortage of them in the Sorcerous Kingdom made that abundantly clear. While the territory was still a part of Re-Estize, there were probably only six hundred or so professional magic casters in the Duchy of E-Rantel. Once divided into their various vocations and considering the potency of spells that most could use, it was easy to see how recent events had led to their current situation.

 

According to the information gathered by the House of Lords, it was estimated that eighty percent of the magic casters working in the duchy had fled before or during the annexation process. They held positions which were always in demand throughout Human lands, so it was a simple matter for them to uproot themselves and find new homes for themselves.

 

The majority who remained were either divine casters who opted to remain and serve their congregations, a handful of crafters, and the few adventurers who had signed on with the repurposed Adventurer Guild. The Magician Guild, even if one counted the magical crafters who were members under the organization’s umbrella, was a ghost of its former self and gutted of anything useful: they had quickly emptied their inventories to be transported to Re-Estize before the transfer of authority. Needless to say, the various trade guilds under the Merchant Guild similarly saw the same decline in membership.

 

Specifically, the dire drop in the mages who provided utility and crafting services to the citizenry was nothing short of a silent crisis: until something was done to restore these losses, the Sorcerous Kingdom would be mostly dependent on expensive imports from the Empire and beyond. The nation was in the oddly embarrassing position of having very little in terms of magical industries, despite possessing superlative magical might.

 

“The rest are the fifty…I said this, yes?”

 

“I meant those that are not Druids or Rangers,” Ludmila told the Volkhv. “Amongst my people, becoming a Druid is relatively rare.”

 

“Ah,” he replied, “so it is like that. Our people are all born Druids or Rangers – it is not a path to be pursued like some other races we have observed. I made an erroneous assumption: you expressed yourself as a Ranger, this youngling also exhibits the qualities of a Ranger, those males we passed also have Ranger-like traits, so it was my assumption that your kind are all born Rangers, and you a Lord of your race.”

 

When he put it in those terms, Ludmila supposed that Humans were actually in the minority when it came to how each individual developed. When it came to Demihumans, it was more often than not that each race lent to what would be considered a specific vocation by Human measure. Gnolls were natural Rangers, Ogres were Barbarians, Trolls were Fighters. Even if a member of that race filled the role of a Druid, Shaman or some other magic caster, their natural strengths would still come into play.

 

Though she knew of several, a race of natural magic casters was unheard of in the immediate vicinity of Warden’s Vale. She could not recall any who were either one thing or another.

 

“My race is referred to as Human…but I thought you knew who I was when I expressed myself?”

 

“I did, and I still do. Knowing the quality of one’s soul does not define an individual in those terms, it simply expresses what one is.”

 

Despite herself, she knew what he was saying. Their culture appeared to be one which revolved around nature, straightforward and true to spirit and form. The way in which the Volkhv perceived the world and conveyed his words made much more sense when viewed through that lens. Rather than perceiving her through individualistic terms, she was being identified by her place within a much wider framework.

 

Ludmila suspected that, even if they were Druids, they would not be overly useful to the duchy at large – they were probably akin to those Druids who lived apart from civilization as hermits, or in isolated circles that watched over their little part of nature. Perhaps time and understanding would help her figure out how they could further contribute to the overall state of her demesne.

 

“Alright, so these four hundred…what do you eat? Do you build? Craft?”

 

“We graze on the vegetation that grows at those heights that we live in: lichen, moss, alpine plants. In the frozen seasons, we move down into the warmer elevations. We do not build. The Rangers will occasionally fashion objects, but we Druids do not. Between our bodies and our magic, we have everything we need.”

 

The Krkonoše seemed to loosely have the grazing habits of the sheep or goats that they resembled. They were maybe half the size of a mountain sheep, but the Volkhv had demonstrated enough magical might to easily destroy a Death Knight. His placid nature when confronting it just seemed to be how he always was – which was fortunate for their neighbors. Being overrun by hundreds of these cute and fluffy Druids would be irresistible in multiple ways.

 

“Do your people’s Rangers do anything else differently?” Ludmila asked, “By your description, they demonstrate several different habits.”

 

“Hmm…yes,” the Volkhv replied. “They are usually always below, stalking the wooded slopes under our homes and keeping others away. They do not eat the same things that we Druids do.”

 

“I guess that makes sense, considering they do not live at the same altitude. They feed on leaves and grass, then?”

 

The Volkhv eyed her strangely, tilting its head so that two of its eyes could take her in at once.

 

“They feed on us,” it replied.

 

About one hundred metres from the entrance to the farming village, Ludmila abruptly stopped in her tracks and Jelena bumped into her from behind.

 

“I’m not sure if I heard that right…”

 

“They – the Rangers, feed on us – the Druids,” he said slowly. “This should be the meaning of what I said, yes?”

 

“Yes, but…huh? The Krkonoše is a species that practices cannibalism? Er, no wait. I thought you fed on vegetation.”

 

“The Druids do, yes. The Rangers are carnivorous. We are one people, but we are not the same species. Did you think us an insect race which can produce multiple types of individuals? Do Human Rangers eat Human Druids?”

 

“I cannot say that I have ever done that – not knowingly, anyways.”

 

They resumed walking, ascending the ramp leading to the warehouse half of the village. Jelena waved to them as she went her own way, up the ramp to the common area, and Ludmila waved back.

 

Two permanent warehouses had been raised so far, but most of the labour allocated to the storage area had been focused into getting the grain silos ready for the harvest. A dozen had been constructed, raised off of the ground so that their contents could be poured out of a spout below. A dozen more were still at various stages of completion.

 

Walking together to the eastern edge of the village, she tried to puzzle out what exactly the Krkonoše were as they looked over the valley stretching out below.

 

“Generally speaking,” she said after a few minutes, “one intelligent race that eats another does not get along with them past a certain point, if at all. Yet here you are claiming that you are one people.”

 

“I suppose you are correct,” he admitted, “but it can happen if the circumstances allow for it. I do not doubt there are others with such an arrangement in the wide world. As for the Krkonoše, we have been one people for countless ages. The fall of Dragonkind is but a recent memory to us.”

 

“The fall of Dragonkind?”

 

“A somewhat exciting, but brief, event,” the Volkhv’s voice seemed to shrug. “In the span of aeons, it was half a wink in time. Changes come; the world turns – those who cannot survive are lost to the mists of eternity. It is not the first time that the keepers have been overturned and others rise to take their place. The Soul of the World ever flows; ever grows in uncountable fragments which come and go.”

 

“That’s, uh…right, you’ve lost me completely now.”

 

“Perhaps you will perceive this at some point,” the Volkhv said. “Before your own returns to the fold.”

 

As interesting as talk about Dragons might be, Ludmila still had to figure out what was going on with the Krkonoše. Clearing her throat, she steered the discussion back on track.

 

“What are these circumstances that allow for you to exist together as predator and prey?”

 

“We are fecund,” the Volkhv answered, “and life in the high places comes with its own dangers, as you might imagine. Only the strong can survive.”

 

“That was a lot more practical of a reason than I thought it would be.”

 

“Practical...yes. You understand, Warden. Order must be maintained, Balance must be kept.”

 

“If that is the case, how much of the range did you occupy? How much land is required to sustain four hundred of you. And the fifty.”

 

“Our ancestral home is a smaller range of peaks: north of the Zern enclaves and west of the mountains under which the dominion of the Dwarves lies. As only the strongest of our offspring survive, our numbers do not increase very quickly. The Zhrets offer the remains of those whose lives have been claimed by the mountain to the Rangers below. Those who reach the end of their days similarly offer themselves, after the wisdom of their lives has been passed on to the next generation.”

 

“So you are not being hunted for food by your Rangers?”

 

“Nature takes its course,” the Volkhv said simply. “Those who do not grow quickly enough cannot survive. The elements take them. They cannot defend against large, flying predators or fail to compete for mates. They cannot migrate when they need to. Many dangers await on the mountain; the Zhrets find the fallen and send them to the other half of our people.”

 

“How often does your species bear offspring?”

 

“Our breeding seasons come in summer and winter; the females give birth to three or four the next season. The offspring are left to fend for themselves before the next litter is born.”

 

They bred nearly as fast as Goblins, and matured far faster. She started to wonder if this ancient relationship had its roots in necessity. Even a slight improvement in their living conditions would have them covering the mountains in short order.

 

“What happens if your numbers are lower than usual?” Ludmila asked after a rough calculation, “Do these Rangers eat anything else?”

 

“Of course,” he replied, “when out ranging beyond the immediate area of our peaks, they will hunt and consume other things. This creates a wide area around our homes where any who dare approach risk becoming their prey.”

 

“Are there other benefits to your arrangement?”

 

“They do not have to range too far to sustain themselves and, by staying in the area, they keep all others from the lowlands away. They also warn us when changes occur – such as that which has happened recently.”

 

“So…that’s the extent of it? There must be some other reason you consider yourselves one people. What you’ve described so far might be better described as something like keeping pets that eat you on a regular basis.”

 

“There are social aspects as well. We mingle frequently: some of my greatest friends are of the Rangers.”

 

“Even though they’ll eat you one day,” Ludmila said flatly.

 

“To be consumed by an old friend is a desirable thing.”

 

She had no answer to that. The relationship between the two races of the Krkonoše was just far too alien to her own understanding.

 

“What is the lifespan of your people?” Ludmila asked, “Fecund races are usually short lived.”

 

“It is uncommon for us to live past eighty seasons.”

 

“As in four seasons every winter?”

 

“Yes, twenty winters.”

 

“...how old are you anyways?”

 

“Hmm…this will be my thirty-fourth season.”

 

“Are you telling me that an eight-year-old nearly obliterated one of my Death Knights with no apparent difficulty?”

 

“There is nothing strange about this.”

 

“Nothing strange – wait, does that mean your entire race becomes as powerful as you? What tier of magic are your people capable of?”

 

“If one of our young cannot cast the third circle of magic by the end of their first year, they are not expected to survive. We become capable of fourth circle magic by our third year, and fifth circle magic by our sixth year. After that, it varies a little…perhaps one in one hundred will become capable of more.”

 

Ludmila briefly envisioned a wave of antlers, fluff and death if anyone managed to anger the Krkonoše somehow. Another thought crossed her mind.

 

“How about your Rangers?”

 

The Volkhv was silent for a moment, nose twitching in the wind as it seemed to think on its answer.

 

“This is more difficult to measure. The majority become roughly as strong as our third years, while perhaps one in several hundred match with our fifth years. That is the most straightforward comparison I can come up with, fashioned roughly in terms you might understand.”

 

“Has it ever occurred to you that you could conquer entire swathes of the region with your population?”

 

Ludmila said it half-jokingly but, even with what little she knew of magic now, she did not think that Re-Estize or Baharuth could stand against so many powerful casters. Even a single Krkonoše probably had the strength to shatter the entire army of Re-Estize.

 

“Conquest seems a pointless exercise,” the Volkhv replied. “Our ways make us stronger with the passing of thousands of generations, and our magic provides us with all that we need. We are not a race of builders such as yours; having more land ultimately means nothing to our ways: we only need what we need. The passing of the ages makes us stronger, and we need not enter into conflict like the races below. The day will come when more of our offspring survive and our numbers swell, but there are innumerable peaks in the world that we can migrate to. This has happened before, far in the past.”

 

“You mean there are other populations of Krkonoše in these mountains?”

 

“No, only our own,” the Volkhv told her. “Once we were a multitude throughout these ranges. The peaks were higher, but the weather was warmer. A Dragon moved in at some point and, with the addition of her offspring, most of us were hunted off of the mountains over the ages before the remainder finally adapted to survive them. Several millennia after that came the Fall of Dragonkind: leaving the Krkonoše uncontested in our peaks.”

 

Up to this point, Ludmila thought that the Krkonoše’s way of cultivating strength through generations was similar to what was promoted by the tenets of the Six Great Gods. Their conditions for survival, however, were merciless. Being innate magic casters that bred rapidly enabled them to grow a population of astonishing power. Fortunately for their neighbors, their passive disposition was at odds with their sheer strength; Humans would almost certainly exploit such an advantage.

 

The nature of the Krkonoše request felt akin to having wildlife apply for immigration: they would most likely just become a part of nature in her demesne. They did not appear to compete directly with her subjects in any way, so she did not particularly mind them. Since they took care of themselves and did not require any support from her, any benefits she gained in the future from their presence would be essentially free so she could take time to figure that part out. Unfortunately, the arrangement of their society ran afoul of one of the laws of the Sorcerous Kingdom.

 

Subjects of the Sorcerous Kingdom were not allowed to eat fellow subjects. As the Krkonoše would submit themselves to her management, they would become her subjects. Continuing in their practices meant that they would be breaking that law thousands of times a year. A noble could not enact local laws that opposed the crown laws, so she would need to petition for a special ordinance to make an exception for the relationship between the two species.

 

The central administration generally processed things in a speedy manner, so, if she could make a solid case for them, Ludmila thought she could have it taken care of the next time she was in the city. Lady Shalltear would be coming by in the next day or two, so she could consult with her on the matter as well.

 

Another issue concerned the Rangers mentioned in their discussion. Ludmila was taught that carnivorous and territorial species stood a high chance of being problematically aggressive when it came to interactions with their neighbors. Rather than a matter of reason, it was an aspect of nature that was not so easily changed. Demihumans and monsters with such a nature had a risk of exploding violently even when things appeared peaceful mere seconds before. Hopefully, their ancient bonds with the Druids of the Krkonoše meant that there was an avenue for their natural tendencies to follow that did not cause problems when they encountered other subjects.

 

“I do not think that you have described the appearance of your Rangers,” Ludmila said. “They seem to be large and powerful, judging by their diet and your estimations.”

 

“Yes, you are correct,” the Volkhv’s ear twitched. “They average between three to four metres in length as adults, with the appearance of mountain leopards. Some appeared around this place recently, but reported back that one of their number was lost. For this reason, I came to investigate.”

 