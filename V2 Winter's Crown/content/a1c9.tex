\chapter{Ludmila Zahradnik}

“What’s this marker here?” Ludmila asked.

 

She pointed to an icon on the map, which was more prominent than even E-Rantel. It sat roughly two hours northeast of the capital by Dragon, situated near the Great Forest of Tob. She knew that the maps she possessed up until now were not very good by the Sorcerous Kingdom’s standards, but something notable there would have surely appeared on them.

 

“It is the realm that contains my demesne,” Lady Shalltear replied.

 

“It contains your demesne?” Ludmila was thoroughly confused.

 

“Yes,” Lady Shalltear nodded. “On the surface it covers an area not much larger than a small village, but it is much larger than it looks. Since we’re there, let’s start with a route from E-Rantel.”

 

Ludmila added the unlabeled location onto a large table she had drawn up. She had the loose thought that each route should have its costs calculated based on the equivalent land route running between the same two stops, or at least the rates charged for similar conditions by the myriad of merchant lines. Unfortunately, even that was widely varied, as land transportation came in many forms and were owned by villages, merchants and nobles alike.

 

What were ubiquitously known as ‘wagons’ actually consisted of two major components: a chassis which consisted of the working parts of the vehicle, which could be adjusted to accommodate the bed upon which cargo was loaded. An agricultural village might have a box-like grain bed and a wider hay bed, while villages around quarries, mines or logging areas would have configurations of their own.

 

A merchant’s freight wagon – which employed teams of draft animals – could haul roughly three times as much as an Adult Frost Dragon equipped with four Infinite Haversacks. There were even heavier freight wagons that hauled massive amounts of ore and stone from their sources, but they were not used in the duchy for fear of ruining the roads. Timber was more often than not dragged by oxen or horses, sized into manageable portions in much the same fashion as Ludmila’s woodcutters sized their timber to be handled by Death Knights.

 

This made it a headache to price local freight, which was more often than not handled by individual villages and administrators. In the end, they decided to leave local freight to local management and focus on hauling cargo over longer distances. At these distances, a specialized, high-speed courier service also started to appear as an attractive option. All in all, it was difficult to determine what the best course would be without actually fielding the Frost Dragons to collect enough information on the actual demand.

 

“Hm…maybe we should just match everyone else’s rates?” Lady Shalltear furrowed her brow as she looked over all the various reference materials scattered across the table, “My original purpose was to connect all of the remote and isolated areas of the Sorcerous Kingdom, several key locations, and the friendly states nearby. I never considered generating revenues from the service.”

 

“Well, it goes by weight over distance and the class of freight, which determines factors such as security and speed. With how every shipment quite literally has a Dragon guarding it, however, the service is far beyond the high security couriers used by national interests, and they carry far more than said couriers. Only ships can really compete with Frost Dragons when it comes to raw freight over time but, as you mentioned some time ago, the capacity for a port to handle cargo will heavily influence the preferred means of delivery. By Gate or by Dragon or by Wagon, full is full, so clients will choose a balance between cost and speed of delivery.”

 

“Since we’re essentially high-speed freight,” Lady Shalltear said, “what about turning that to our advantage?”

 

“If you’re referring to perishable goods,” Ludmila replied, “smaller scale iterations of the preservation magic that you see used in E-Rantel’s government warehouses are already in use by merchant companies, so we won’t be able to compete there.”

 

It occurred to Ludmila she would not have been able to provide that answer if not for her trip to the west which used Liane’s services. The value of wrangling in her cosmopolitan friend to their side was abundantly clear – maybe they could ask in a roundabout manner until then.

 

“Rushing large quantities of cargo might work,” she added, “but only in the cases where teleportation magic is not available or deemed unnecessary. I can only think that this would happen outside of our borders, as the Sorcerous Kingdom appears to have an abundance of casters capable of teleportation. In addition, while this may be lucrative, I can hardly imagine it to be consistent.”

 

Ludmila looked back down to the map again. While trying to match land-based transportation was probably a workable option, it felt like a waste of the advantages conferred by flight. Her Bone Vultures were ideal for transporting light loads over any sort of terrain, and only the most inclement weather could stop them. Frost Dragons did not care about weather at all, carried more, and were far faster. Her gaze crossed over parts of the map where this advantage could be exploited to its fullest, tracing over mountains, forests and wastelands.

 

“There is no direct route between the Sorcerous Kingdom and the Dwarven cities of the Azerlisia Mountains,” she finally settled on what seemed to hold the most promise. “The main land route available to us runs out east to the Empire and along their western highways. Since imperial tolls drive up the cost of transport between us, it seems a clear opportunity for air freight. Also, deliveries between the Empire and Re-Estize would be promising for similar reasons.”

 

“Lord Ainz has already pledged Soul Eaters to transport goods for the Dwarves,” Lady Shalltear told her. “I’m also fairly certain Re-Estize wants nothing to do with us for the moment.”

 

“Soul Eaters…” Ludmila furrowed her brow, “Wouldn’t the Dragons be better, considering what I’ve said?”

 

“They would.” Lady Shalltear nodded, “However, Lord Ainz surely knows this as well. Some larger scheme must be in play that we currently cannot comprehend.”

 

Ludmila drummed her fingers on the edge of the table. It seemed that they would be feeling their way through this stage of their planning.

 

“Is there a way for us to demonstrate their utility, at least? There may still be some demand from clients in the Dwarf Kingdom for the benefits that only our service can provide.”

 

“Yes, Lord Ainz has seen to that as well.” Lady Shalltear nodded, “After the event in E-Rantel is over, we’ll be relocating to help the Dwarves move back into their old cities as a show of national goodwill. This will also serve as a demonstration of some of the things that the Sorcerous Kingdom has to offer. Most of the work should take about a month or so: the Dwarves have plenty of first-hand exposure to the value of the service.”

 

“His Majesty’s considerations are certainly well thought-out,” Ludmila said. “Not only has His Majesty created the opportunity for us to display the capabilities of the Frost Dragon network to the Dwarves, but we’ll also have a month to gather information, identify problems and refine our systems.”

 

“As expected of Lord Ainz!” Lady Shalltear smiled brightly, “We must ensure that his precious efforts on our behalf are not put to waste – he must have great expectations for our work.”

 

“What are His Majesty’s expectations?”

 

“An aerial transportation network,” Lady Shalltear said simply. “It is our duty to render His Will into reality, to the best of our abilities.”

 

Ludmila nodded slowly. It was a broad expectation, but the micromanagement of the various aspects of a Kingdom was not the role of a King. For the Minister of Transportation, the missive was as concise as it needed to be – it was now up to Lady Shalltear and those entrusted with these duties to figure out how to make it so.

 

She looked back down to her work, adding other locations to the table. As she cross referenced the various rates and schedules, tabulating preliminary numbers for their own network, her grasp on things slowly disintegrated. Every route had its own circumstances which factored into their transportation fees, and numbers alone did not really define them. Without the requisite knowledge and expertise, she could not figure out where advantages lay in any but the most obvious situations.

 

As a Frontier Noble, logistical matters came naturally to her, and it seemed that her sense for it even applied to the aerial transportation network. Matters of commerce, however, she always felt woefully simplistic in. It felt especially pronounced now that she was trying to use what little she knew on the subject on this project.

 

Rather than dwell on her shortfalls, however, she decided to focus on the locations where they probably had the most information on.

 

“What is this body of water marked in Tob Forest?” She asked.

 

“It’s an area that was annexed before E-Rantel,” Lady Shalltear answered. “Several Demihuman species make the lake their home, and Cocytus rules over them.”

 

“Human commerce doesn’t flow through the Great Forest…would these Demihumans have some sort of reference for their own workings?”

 

“As far as I know, they did not trade regularly with the outside world before we found them. They do have rare individuals that have travelled to various parts of the world. Recently, a few have come to E-Rantel, but the majority remain isolated.”

 

“What would they be exporting or importing?”

 

“I thought you might have a better idea, considering how similar your circumstances are to theirs. The surface area of the lake is roughly the same as your floodplains here, though the northern part of the lake is much deeper. Marshes occupy the shallows in the south, and the entire lake is bordered by the forest.”

 

Ludmila could not definitively say what they did and did not have, but she guessed that at least some sort of fish, marsh plants and whatever the inhabitants gleaned from the nearby forest wouldn’t be unreasonable.

 

“What sort of industries do they have?” Ludmila asked.

 

“Mostly hunting and foraging from their surroundings,” Lady Shalltear answered. “There are also fish farms which have expanded since they came under us.”

 

“Fish farms?”

 

“Yes…something like enclosed portions of the shallows where they tend to fish like livestock.”

 

She made a mental note of the curious practice – it seemed like something she might be able to enact in her demesne, which could eliminate several future issues at once.

 

“There are no advanced industries? Metalworking? Tanners and Tailors? Apothecaries and Alchemists?”

 

“The Demihumans there consist of varied tribal groups,” Lady Shalltear replied. “They have a scarce number of Druids and Shamans that tend to their population, and the locally-produced equipment is quite shoddy. Coarse garments, tools and equipment made of wood, stone and animal parts…in short, I wouldn’t expect anything along the lines of what you would expect to find in a Human city.”

 

“So most of their exports would be fish and these raw materials – if they expanded on what they currently have – while they import manufactured goods…this seems difficult to balance in both directions: the tonnage would be severely skewed to exports.”

 

“That should be fine for now. The main goal here is still to connect their isolated communities to the rest of the Sorcerous Kingdom, giving them the sense that they belong to something greater. It is Lord Ainz’s Will that all races and peoples be united under His banner – that they will be able to live in peace, prosperity and harmony. Connecting the disparate peoples is a step towards realizing Lord Ainz’s grand vision for His realm.”

 

Rather than conquest, unification. That the Undead, almost universally reviled, were the ones who sought to bring about this future was an irony Ludmila could only shake her head over.

 

According to the map, the other Demihuman populations living in the wilderness of Tob Forest and the untamed portions of the crown lands existed at a near-subsistence level and had no interest beyond living their current lifestyles. On the Human end of things, every town was already connected to E-Rantel via highway.

 

Ludmila scanned over the map of the duchy again, and she noted one more point of interest.

 

“Since when did this ‘Carne Village’ become large enough to qualify as a town?” She asked, “I don’t recall any large settlements along the northern edge of the duchy developing in the past few years.”

 

Ludmila furrowed her brow as she looked down at the oddly large marker along the northern frontier with Tob Forest – it did not match any other settlements in the area, and was even larger than the towns on the highway. Usually, this sort of explosive growth would stem from an economic impetus – such as discovering rare and valuable resources at that location – but word would have definitely spread throughout the duchy if such findings had occurred.

 

“A large majority of Carne is now Demihuman,” Lady Shalltear answered, “most of which consists of an army of over five thousand Goblins. I hear more Demihumans come out of the forest to join them regularly. As for when it happened…around the same time as the Battle of Katze plains this year.”

 

“How did that happen? Was there an invasion?”

 

“If I heard correctly, the village was attacked by a detachment from the Royal Army. A local village girl led a Goblin army which routed Re-Estize’s forces, and they’ve been there ever since.”

 

The more she heard, the less things made sense. Why would Re-Estize attack one of its own villages? How did a local village girl end up with an army of Goblins? Ludmila decided to move on from the subject before they became sidetracked even further.

 

“What about the towns along the highways?” She asked, “What sort of considerations should we be making for them?”

 

“Since Soul Eaters are readily available here,” Lady Shalltear answered, “Frost Dragons should be unnecessary. The highways are serviceable for the time being, and they will see plenty of traffic as things pick up. If they grow substantially in the future, we will revisit the idea to see if there is any merit. Keep in mind that the points marked are along the primary route which Frost Dragons are used on: even Carne has suitable land routes for their regular needs, so they will rarely see shipments. The rest of the nation will be included in the network, but minor routes will at best be serviced by Bone Vultures to deliver small parcels – much like you do here.”

 

“If all urgent communications and transport in the Sorcerous Kingdom is handled through magical means, would it be safe to assume that all major air routes within the Sorcerous Kingdom will be for freight while minor routes will be for civilian courier services?”

 

“It should be, yes,” Lady Shalltear nodded. “Did you come up with some ideas?”

 

“Not really.” Ludmila replied, “But knowing this makes it easier to determine common rates between the two classes of cargo – at least within our borders. All we have to do is have them match their rates in terms of rough value so we don’t put ourselves at odds with the companies that transport the majority of the freight in our nation by land.”

 

“Do we really need to be stepping so lightly around these companies?” Lady Shalltear frowned, “We could replace them all eventually with Soul Eaters and such.”

 

“These small, local transport companies only stick to servicing the duchy, so they should eventually transition to Soul Eaters as the more economical option regardless. Rather than them being competition, they are actually future clients of the administration who will manage the tasks we need done anyways. It saves us the trouble of taking over what they already oversee, and dealing with the minutiae of their specific locales.”

 

Her liege continued to frown even after the explanation.

 

“Was there something wrong with what I said?” Ludmila asked.

 

“No. If the existing services continue to function to a suitable standard, then there should be no issues. If disruptions or disagreements happen between these companies and the crown, however, Albedo will most likely move to absorb or replace them. Given her tendency towards inflexibility, her domineering personality may simply have such an outcome be an inevitability.”

 

Lady Shalltear’s statement struck an odd chord amidst their long, yet generally smooth proceedings. Though she had not met the Prime Minister personally, through her work Ludmila considered Lady Albedo to be a fairly flexible administrator who was willing to slowly acclimate the population of a conquered land to new ways. She did not think Lady Shalltear would lie to her over something like this, but she sensed that there might be a bit of a bias, at least.

 

“My lady, I have a question, if you don’t mind…”

 

“Whether I mind or not depends on the question, wouldn’t it?”

 

Ludmila thought over her words for a moment before deciding that knowing the answer would help her make better decisions in the future.

 

“This extra sense of rivalry that you two have…is it a result of Lady Albedo’s personality? The way you make it sound, she seems like quite a difficult person to personally get along with, while you seem reasonable and tempered in my experience.”

 

Lady Shalltear’s lip twitched. She crossed her arms over the table as she looked sidelong at Ludmila.

 

“I seem ‘reasonable and tempered’ to you because your traditional duties are somewhat relatable to my own, though the scale is vastly different. This is one of the main reasons why I consider you to be compatible to myself and why things are generally smooth between us: our perceptions are judged by the same rough measures, and you understand what it means to serve in this capacity. Our duties revolve around being the first line of defence of the realm: our thoughts and actions are made in deference to this duty, even in tasks outside of this purview.”

 

Lady Shalltear’s gaze turned away from her, across the table and to some distant place beyond the walls of Ludmila’s manor.

 

“There is an irreconcilable chasm that exists between a faithful vassal that stands in her eternal vigil at the edge of the realm, and one who sits comfortably at the right hand of her sovereign. The gates of His Majesty’s realm have been breached many times in the past, and it is I who has met every single challenge. I understand that, even if everything functions as expected, sometimes it is simply not enough. I have also seen too often occasions where hastily concocted plans drawn from raw intellect are mercilessly scattered to the wind.

 

“Albedo is blessed with a vast intellect alongside her ruthless and calculating nature, but her place is in the highest halls of His Majesty’s realm. She sits in the centre of an apparatus which spans the length and breadth of the Sorcerous Kingdom, where she organizes and plots and plans far above the affairs of her subordinates. She is appointed above all of us, and in her self-assuredness she serves in her capacity, but mark my words: when her schemes begin to fly apart one day you will see what sort of catastrophe occurs as a result. Beneath that smiling face of hers is a merciless Demon which looks down disdainfully upon those whom she deems beneath her, and only His Majesty stands above.”

 

Ludmila sipped the tea that Aemilia had freshly prepared while digesting what Lady Shalltear shared with her. She wondered if the courts of other nations, too, were like this. Re-Estize, at the least, manifested similarities with the machinations of the Great Houses while Baharuth with its consolidation of power into the Imperial Administration and its institutions might have inadvertently made things worse for themselves in the long term. As Fassett County had so poignantly demonstrated, the unfeeling gaze of a distant throne could result in many woes.

 

“I think I understand why Countess Jezne so feared the judgement of the Royal Court now,” Ludmila said.

 

“Don’t get me wrong,” Lady Shalltear’s gaze returned to her. “When everything is operating within expectations and no unwelcome surprises occur, Albedo is an unquestionably supreme administrator. She is also just – within the framework of her own values, anyways – and everything that passes across her desk is subject to the same exacting scrutiny that her capabilities bring to bear. She is still insufferably haughty, though.”

 

“Then this rivalry that you maintain is not a result of some past incident?”

 

“Oh, no,” Lady Shalltear leaned back and waved her hand dismissively. “This was just what I thought you actually wanted to ask. Was I mistaken?”

 

“Your intuition was correct, thank you,” Ludmila replied. “I thought I might gain some insight on Lady Albedo’s character, so your answer was exactly what I was trying to ask about. I’m even more curious now, though: what is your rivalry about?”

 

“All of His Majesty’s servants are rivals in the sense that we compete with one another to render the greatest achievements in his service,” Lady Shalltear said, “so you should keep that in mind even when considering the others. The additional rivalry between Albedo and myself that you perceive has to do with our ongoing competition to see who is most qualified to be Lord Ainz’s primary wife.”

 

Ludmila took an extra moment to register what had just been said. She was happy for Lady Shalltear who loved His Majesty so purely, and admittedly a bit jealous as a woman that was behind on that front. The answer made perfect sense, all things considered, but, after being exposed to the grand scope of the Sorcerous Kingdom, she felt that it fell conspicuously short of her expectations for some reason. Maybe it was because Lady Shalltear’s feelings were already something she had long realized.

 

“Ehm…this is probably a silly question, but how do you plan on getting ahead in this rivalry for His Majesty’s affections?”

 

“Well, it is ultimately Lord Ainz’s decision, so forcing the matter would be inappropriate. I heard that the gorilla actually pushed him down once, but it only resulted in her confinement. I can only do what I can, and hope for the best…but even if Albedo does temporarily gain his favour, I will never give up.”

 

“That seems normal enough.”

 

“Yes, normal – unlike that brute.”

