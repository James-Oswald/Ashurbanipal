\chapter{Ludmila Zahradnik}

With their discussion concluded, Moren Boer bobbed his head again several times as he backed away and out of the door. Ludmila turned to Nonna, who had stood by silently throughout the audience.

 

“What do you think of this proposal, Nonna?” Ludmila asked. “Did he bring up any similar topics with you while I was away?”

 

“No, he did not,” Nonna answered. “He has outlined the potential benefits of his proposal clearly in your discourse, however. If these mages can be incorporated as productive members of the Sorcerous Kingdom who abide by the regulations set forth for their work, I have not noted any apparent conflicts with current laws or policies.”

 

“What about non-apparent conflicts? This organization’s admitted history leaves me with a number of worries.”

 

“They are from beyond the realm, and subject to the incomprehensible behaviours that you Humans display. Understanding this, it stands to reason that anything could happen.”

 

The uncertain answer was not very encouraging. Moren Boer presented his associates as a persecuted and weary community seeking sanctuary in exchange for a number of substantial benefits, but the sheer attractiveness of their offer made her more than a bit wary. The Sorcerous Kingdom was supposed to be a nation where its people could find security, harmony and prosperity, but how far could they extend this offer?

 

Her initial openness to the proposal was tempered by the fact that she was someone who was desperate for migrants and skilled labour so she could develop advanced industries, and Moren’s associates appeared to have targeted this precisely. What if their proposal just set the stage for some other, nefarious plan? There was also the idea that the Sorcerous Kingdom might come to be seen as a haven for international criminals. It was something she would have to watch very closely.

 

“Yes…anything could happen,” Ludmila muttered. “Seeing that they’re so invested in Necromancy, maybe they might spend all their time trying to touch you?”

 

The endless scrawling of Nonna’s pen stopped, and the Elder Lich looked up from her work to stare at her.

 

“It was just a joke,” Ludmila said lightly. “Mages don’t do strange things like that, do they?”

 

Nonna remained silent and Ludmila shifted uncomfortably. Surely they didn’t.

 

Ludmila cleared her throat before shifting to a related topic.

 

“Speaking of the things that mages do,” she said, “there are many Elder Liches being utilized as administrators throughout the realm. The current demand for magic casters has not been alleviated by the availability of them…why is this?”

 

“Our repertoire of spells is structured for combat purposes,” Nonna replied. “The current shortfalls stem from the demand for utility magic in the citizens’ daily lives, as well as the fashioning of various magic items for civilian purposes.”

 

“Lady Shalltear appears to have magic items in abundance,” Ludmila noted, “so it should mean there is some sort of industry that produces them elsewhere in the Sorcerous Kingdom.”

 

“The means by which we produce magical equipment are unsuited for this place,” Nonna answered, “nor does it produce any items of relevant common interest. For this reason, local methods are being employed.”

 

“I see…have there been any projections made by the administration in regards to this industry? Is it worth pursuing here, or is it only a temporary issue due to the recent changes in the region?”

 

“The projections from the central administration have determined it will be an ongoing issue for the indeterminate future. The ratio of magic users in this duchy is disproportionately small, and associated industries will be suffering a prolonged shortage as a result. Imports of magic items from the Empire consist of nearly all of our new supply.”

 

Ludmila breathed a sigh of relief. It was another industry she could promote in her demesne, and the first real industry that could provide her with revenues not directly contingent on the goods produced by Undead labour…provided that things worked as advertised. She had no idea what Moren’s associates migrating to her demesne meant in practical terms for magic item production, so it was something that would need to be explored in depth later.

 

Standing up to collect her things scattered around the floor, she shouldered her bag again. The shadows had already started to lengthen, and Ludmila needed to complete her tasks as quickly as possible so she could return to the manor and review the information on the aerial transportation network.

 

“I should get going,” she told Nonna. “The logging camp is next on my list. You may continue your work here – I cannot imagine anything you would be needed there for.”

 

Leaving Nonna to her work in the Lichtower, Ludmila stepped out into the bright daylight. The foreman rose as soon as he spotted her coming out to speak to him.

 

“I believe I have addressed all of the household petitions here,” she said. “Were there any additional matters on your mind?”

 

“Construction’s been steady,” he replied, “Farmers haven’t said anything and I haven’t heard much else from anyone. Should be on track for Feast Day celebrations – wall and all.”

 

“That is reassuring to hear,” Ludmila nodded. “Speaking of Feast Day, is there anything you would like to see that we do not have here? I might be able to have it imported, if it is available.”

 

“You’d do that? I mean, uh…well, if it’s for the feast…the food is great here, but water and tea are all we have for drink. Some sort of liquor would be grand, my lady.”

 

She should have seen that coming. It was customary for her father to import several casks for the village from the winter markets.

 

“Is there anything else?”

 

“Don’t really know, to be honest,” the foreman replied. “I’m from a place where people come by the roads, but here you bring in everything by ship. Normally we didn’t have to ask – asking didn’t matter, anyways – and nobles didn’t think about it at all. The merchants and peddlers going around from place to place had it all figured out.”

 

Given the run-down state of the roads, relying on merchants over land was out of the question. Even if she fixed the bridge somehow, her demesne was well out of the way of any well-traveled routes, and not populous enough for merchants to justify the detour. She would need to check over the cargo manifests from over the years to see if she could figure out what was needed.

 

“Well, we still have several more weeks until I finalize my orders for the Feast Day provisions,” she told the foreman. “I should be in the demesne on a regular basis until the harvest needs to be delivered. Let the other tenants know that they have the opportunity to ask.”

 

“Will do, my lady,” he tipped his hat and nodded. “You’re headed out now, then?”

 

“The logging camp is next. Once I am done there, I will be back through here to see how construction is progressing.”

 

With that, she turned away and left the village, walking back down the ramp leading to the main road. A pair of Death Knights, transporting a massive fir trunk between them, passed her in the opposite direction as she continued her way up the road to the northwest. They were followed by three carts, filled with bundles of branches, pulled by Undead beasts.

 

The logging camp was a makeshift collection of large, olive-coloured, tents roughly five kilometres from the village. The large shelters had been woven out of fabric, extracted from the rushes from the marsh, which had then been weatherproofed. Due to the short distance from the village to the uncleared land, the camp was only recently set up, and would slowly be moved to follow their progress.

 

Short piles of tree trunks, trimmed of branches, awaited transport near to the road, divided by the types of wood they would yield. The woodcutters felled more trees than the Death Knights could transport in the same time, but the backlog would be cleared away by the next morning while the labourers rested. Six large Iron Golems were slated to arrive in three weeks, which would drastically speed up the process. After the farmland for the next village was cleared, Lord Mare would arrive to shape the terraces and raise the foundation for the next farming outpost.

 

A dry, rasping squawk turned her attention away from the nearing camp and she saw a Bone Vulture descending from the sky with a basket. Ludmila reached up to receive it, and the Bone Vulture rose to circle far overhead. Her pace picked up in anticipation of the meal, and she arrived at the camp shortly after. The woodcutters were all out and, amidst the scent of fresh pine tar from the tents, she sat down on one of the long logs arranged around the central campfire.

 

Lunch was a recipe from Terah that Aemilia had picked up at some point: the sandwich was similar to what her housekeeper often packed for when she was out and about, but, as the ingredients went from the cultivated foods delivered to E-Rantel to the Barony’s foods that were harvested from the wild, it resulted in a markedly different meal. The vegetables were foraged from the forests and the marsh, the fillets of broiled trout from the river. The variety of sauces and spices from the city was also absent, so the sandwich relied entirely upon its main ingredients for flavour.

 

She set into her meal, wondering when she could start training Rangers to improve on the variety of foods available to the fief. Not only that, they needed to see to creating stockpiles of medicinal herbs. As the migrants were from the inland territories, it was no surprise that there were no Rangers among them, so Ludmila was the sole individual performing all the work that the residents of Warden’s Vale had normally shared between them. Ambitions surrounding training Rangers for the Sorcerous Kingdom’s armies aside, she needed them desperately for her own demesne.

 

Finishing her meal, Ludmila rose and called down the Bone Vulture to carry the basket back to Aemilia. There was a distant, telltale stomping of metal boots as a second pair of Death Knights came to retrieve another trunk from the piles nearby. A murmur of men’s voices came from beyond the tents to the west, and they all froze as they entered the camp.

 

“Lady Zahradnik,” the lead man said. “We didn’t realize you were here. How long…”

 

“I just sat down for lunch,” Ludmila filled in as his voice trailed off. “Roscoe, was it?”

 

“Ah…yes,” he bobbed his head, “Lyndon, it is. Surprised you remembered, m’lady.”

 

“Just your last name, to be honest,” she smiled slightly. “It comes from being a noble, I suppose. I had families hammered into me all the time by my parents.”

 

“Family…yes. Family’s what’s important.”

 

There was a rustling from behind one of the tents, and the same girl Ludmila had seen carrying the Slime through the farming village padded out into the open. This time, she was holding a covered wicker basket that was nearly identical to the one that Ludmila had just sent away.

 

“Time for lunch?” Ludmila asked as the girl handed the basket over to Lyndon.

 

“You got it,” Lyndon replied, and he lifted the cover to hand out meals wrapped in cloth. “Er, this is my son, Nash. My daughter here is Jelena, m’lady.”

 

“Jelena?” Ludmila took note of the out-of-place name between the three.

 

“Sounds strange, doesn’t it?” Lyndon patted his daughter’s head, “Actually maybe not. Your name feels similar; my wife Jelka has the same eyes and hair, too.”

 

Ludmila’s gaze went between the family members as she studied them. Nash clearly took after his father and, if Jelena took after her mother, it was possible that she shared the same ancestry as the people of Warden’s Vale that Andrei Zahradnik had gathered around him.

 

“That is quite interesting, actually” she mused. “Jelena seems quite adventurous as well: I saw her with a Slime she found from somewhere outside the village.”

 

“She is,” Lyndon agreed, “and she gives us scares all the time too. Was more curious than afraid coming off of the boat, and she wandered all over the place after moving in. Certainly doesn’t take after us – maybe it’s a tell of the blood, like the teachings of The Six say.”

 

The teachings Lyndon referred to concerned the strength of bloodlines passed down through familial lineages. It was one of the tenets of the faith: the same tenets that encouraged its members to take strong, talented partners in an effort to ensure that humanity slowly became stronger from generation to generation. Though it was not specifically mentioned in the scriptures of the Six Great Gods as Lyndon had claimed, ‘tells of the blood’ was a cultural idea – referred to in differing forms, depending on where one was – that was widespread amongst the faithful.

 

It referred to a stark contrast in an individual relative to the other members of their family. Whether it manifested as more pronounced traits from one’s bloodline or something rare and unexpected, those attributed with this label had strength and skill beyond ordinary Humans, often with personality borne on a spirit to match. They were the crystallization of generation after generation following the doctrines of the faith: a precious jewel that would become the pride of a family – proof that their perseverance had been rewarded.

 

Looking back to Lyndon, Ludmila realized that the woodcutter was appealing to her for help. There was no priest in her demesne to consult with, and he had no idea what to do with the budding blossom that had blessed his humble household. Duty to one’s subjects came in many forms: ensuring that these rare individuals did not go to waste was perhaps the most prominent one to followers of the Six Great Gods.

 

Ludmila looked Lyndon in the eye.

 

“Will both your children be following you in your trade?” She asked.

 

“T-that’s…Nash’s taken to it: he’s got it in him to continue my trade. Jelena doesn’t like that we’re clearing the forest though.”

 

“How old are they?”

 

“Nash’s fourteen, Jelena’s twelve.”

 

“Have they received any schooling? Can they read and write?”

 

“Uh…me and my wife can’t write or read, m’lady,” Lyndon’s gaze turned downwards. “Kids can’t either. No temple school where we came from.”

 

“I see,” Ludmila said. “Every village here will have a school, so every child in my demesne will learn how to read, write and work with numbers. Perhaps all the adults that have come here will eventually as well. After the children have completed their basic education, they will either follow in their parents’ trade, or move to the port to learn a new vocation under the people there.”

 

“Schools…for everyone, m’lady?” Lyndon looked up at her, “My boy helps me here with work, and he’s learning what he needs to…when will there be time for school?”

 

“Well, ideally they’d start younger in their schooling, but, for the time being, you’ll have to set aside mornings for them when lessons start. In a few weeks there will be additional labour coming from the capital, so you need not worry about falling behind.”

 

“Is that so? Well...I’d be happy if they could learn. This whole place you have here, my lady – everything is so different from where we came from, so I really didn’t know what to expect.”

 

“How do you mean?” Ludmila was curious how her demesne differed from the others in the eyes of a tenant.

 

“Uh…I’m no noble so maybe I don’t know much,” Lyndon balled up the cloth in his hands after finishing his meal, “but if one word describes the feeling, it would be control.”

 

Ludmila had an inkling that the people might feel that way from her conversation with Ostrik the previous day, but she never imagined that it might become a problem.

 

“Is this a good thing,” she asked, “or a bad thing?”

 

“I-it’s just how it is, I guess?” Lyndon seemed to shrivel up from her scrutiny, “m’lady – I’m sorry m’lady. Maybe I said too much, you being so friendly and all. N-not that I’m blaming you. Never had a noble talk to me so much before…never even seen one till I got here, actually…”

 

The tanned woodcutter wiped his forehead with the cloth, scattering crumbs across his face. He didn’t move to brush them out of the furrows in his brow.

 

“I was just curious,” Ludmila said softly. “With so many new migrants, I thought that maybe there are some things to consider beyond how I am accustomed to doing things. What did you mean by control?”

 

“It’s just…the overall feeling, I guess,” Lyndon said. “Normally, lords don’t leave their manors and towns to hold court in farming villages: village chiefs and such deal with all that. Their retinues usually never leave them, either, and farming villages don’t have militia. Merchants come and go to sell us what we need…otherwise, we’re just given a free hand to do whatever. The lord claims that the lands are his, but the only time it seems to matter is when he’s raising a levy or when the tax collector comes with his escort. Here, it’s pretty much the reverse – you don’t even need to say anything, and without a single doubt everyone knows everything is yours.”

 

Ludmila supposed that this would indeed be how a commoner saw things: especially in small, poor fiefs. While the borders of a demesne were defined by lines on a map from a legal standpoint, the ability to exert control over one’s demesne relied almost entirely on a noble’s capabilities. The ability to project martial power to keep the lands secure; the wealth to improve facilities and infrastructure; the administrative acumen that allowed one to firmly grasp what was going on in their own lands and what to do about it…all these were the purview of the noble that ruled their land. A poor demesne could not maintain costs for extensive security or development, and talent in administration was not guaranteed between members of the aristocracy.

 

A wealthy demesne with a legacy of excellent administrators, such as that of House Corelyn, could afford a well-funded militia and talented retinue to address all but the most serious of problems. The cycle of prosperity that they promoted resulted in a well-connected and secure fief in which the needs of its people were adequately addressed. A fief without these advantages would struggle to keep even the immediate area around their administrative centres secure: problems like bandits, Demihumans and monsters would mount until Adventurers could be afforded – if at all.

 

Re-Estize had a minor stipend in place which acted as a subsidy to mitigate the costs incurred by Adventurer teams, but it was insufficient for even low levels of security. If anything, it was a last, desperate measure to keep the lands from descending into ruin from those threats.

 

“The protection of the Sorcerous Kingdom’s armies encompasses the entirety of the duchy,” Ludmila said. “The level of security enjoyed as a result of the Undead forces here should be representative of all the lands now. Goods should flow more freely once we become attractive enough for merchants, but, for now, we are entirely reliant on the capacity of our river transport. While I do effectively have complete control over the flow of goods, it is entirely a practical consideration to ensure basic needs are met. The current tax rates are also due to similar considerations when it comes to development and investing in future growth.”

 

“I uh...didn’t get most of that, but did you say taxes would change? We’re clearing a lot of land right now, so it sure looks like we’re pulling in a lot of timber, but we reach the end of this strip and that’s it, right?”

 

“Yes – that’s why I have limited the number of woodcutters in the demesne to the families that you are working with now. After we’re done here, you will be selectively harvesting throughout the valley and setting up stands of trees to farm. This is the plan for the time being, anyways – tax rates will be adjusted with the shifting prices of commodities, and once I get a better idea of what else we can produce here.”

 

“Makes sense, I guess?” Lyndon scratched his head, “As long as we don’t starve.”

 

“Trust me, it is very hard to starve here. As for everything else, I hope that I have not been too stifling…by the way, I’ve heard something to the tune of the people being…lost. Directionless. Is that true?”

 

“Maybe,” Lyndon furrowed his brow. “Things are so different here that you can’t help but be lost when you arrive. After the new folks come to the farming village, though, I think it goes away pretty quick. They see what’s going on around them; what their work means. Even after the village is done, the next village will be coming up. There’ll be plenty to do for a long while yet. After that...I guess what you heard might be right. Everyone is so used to struggling to make it from season to season. Without that hounding you on…I don’t know.”

 

We do not have everything we need for when we have everything we need…

 

The statement in Moren Boer’s strange proposal took on an entirely different meaning. When the struggle for immediate survival was no longer a motivator in the people’s lives, what could she provide to fill that unexpectedly unwelcome void?

 

“You have my appreciation nonetheless,” Ludmila smiled. “Perhaps after I have spoken to others and see how things go, I will figure something out.”

 

“I guess that’s why you’re the lady, m’lady.”