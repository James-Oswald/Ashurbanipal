\chapter{Ludmila Zahradnik}

Lyndon’s children finished their meals well before the discussion ended, and they now looked bored and restless. After the woodcutter and his son left the collection of tents, Jelena peered up down the road to the village.

 

“The next pair of Death Knights should not be back for over an hour yet,” Ludmila told her. “I am going to be inspecting the work here – would you like to come with me?”

 

From what Ludmila had seen so far, Jelena appeared to be quite resourceful. Where most might consider it risky for a girl to travel alone on a road surrounded by fields of tall grain bordering the wilderness, Jelena had simply followed the Death Knights as they went back and forth delivering their cargo. As they had orders to defend the fief and its people; any opportunistic creatures that made an attempt to attack the girl would have to deal with two nation-destroying Undead, which happened to specialize in defence, first.

 

The girl looked up at her, dirty blonde hair whipping in the wind. After a moment, she nodded and they walked together up the road towards the slowly receding forest. The land around them had been cleared of trees and valuable materials, and it currently looked quite ugly. Elements of the undergrowth accustomed to shady, damp conditions lay drying under the open skies, turning the landscape into a bleached and desiccated-looking scar.

 

“Your father says that clearing the forest displeases you,” Ludmila said.

 

Jelena nodded beside her wordlessly.

 

“Did you know that I dislike it, as well?”

 

Jelena slowed her pace to look up at Ludmila again with a puzzled expression.

 

“My ancestor came to this vale, over one hundred years ago,” Ludmila continued to scan their surroundings as they made their way. “He fell in love with this place, and settled here with the hope that his descendants would grow and thrive surrounded by nature and all of its blessings. Everyone that grows up in the vale loves it for what it is: including me.”

 

“Then why are they cutting down all the trees?” Jelena finally spoke, “You’re the noble…that means you’re in charge, right?”

 

“That is correct,” Ludmila replied. “This is my demesne, and I ordered the strip of land here cleared all the way to the other end of the vale – even though I dislike the feeling that it gives me. It is probably the same feeling that you have.”

 

“This feels terrible. Why would you do something that makes you feel so bad?”

 

“Because my demesne needs to be strong enough to support itself. It may look like a lot of the trees are being cleared away, but, in reality, it is not even a tenth of Warden’s Vale. By developing this land, I will be able to feed the people that come here and pay for the upkeep of the Undead. In this manner, we will be strong enough to protect the rest of the land that remains untouched. That is the law of nature; a rule of our world: if you wish to protect what you hold dear, you must be strong enough to overcome all the adversaries that would try to steal it away from you. The strong survive – the weak, perish. Do you understand?”

 

Ludmila remained silent as Jelena seemed to consider her words. They reached the treeline, where the remains of the old road disappeared into the dense forest that had overgrown it in generations past. Heading west a safe distance away from the trees, she slowed her pace as Jelena picked her way through the rough terrain.

 

“I think so,” she finally replied.

 

“Do you know what you want to be when you come of age, Jelena?”

 

Jelena shook her head.

 

“Well, if you love nature and get along with the forest creatures, what about becoming a Ranger like me?”

 

Jelena stopped and looked up at Ludmila again.

 

“You don’t look like a Ranger,” she said. “You look like a fancy lady.”

 

“There is no rule that says Rangers cannot look like fancy ladies,” Ludmila let out a laugh. “I suppose your idea of a Ranger is an Adventurer with a bow or something like that?”

 

“They’re not?”

 

“Well, they can be that,” Ludmila admitted. “I am registered as an Adventurer, but my duties as a noble take priority. I can use a bow, but I can use other weapons too. Demihuman Rangers can use slings or even just throw rocks. They can use their claws, teeth and other parts of their bodies as well as I can wield a spear, if not better. Being able to fight, however, is only a part of being a Ranger. If anything, fighting is the least of what we do out of everything – most of my work as a Ranger can be done even while wearing this fancy lady dress of mine.”

 

“Then…what do you do?” Jelena asked.

 

“Rangers are masters of the lands that they watch over and protect,” Ludmila answered. “In the marshes, forests, hills and mountains, the Rangers of Warden’s Vale are raised and trained to be as comfortable in the outdoors as they are in their own homes – because the outdoors is our home. You may find Rangers anywhere: from deserts to oceans to caves…I have even heard of Rangers becoming denizens of vast cities, which are whole environments of their own. We are experts at survival, tracking and navigating the lands: familiar to them in ways unachievable by any other – we can even get along with the creatures within them that others would consider too savage to befriend.

 

“Rangers understand the nature of things and, with this understanding, we view the world through a lens that many others may not even consider. We think and act in ways that may seem incomprehensible to everyone else in our efforts to promote natural balance. Do you think this sounds like you? Does it sound like something you would like to do?”

 

Jelena nodded quietly, and Ludmila smiled warmly.

 

“Well, it just so happens that Rangers are in demand here,” she said. “Once you have finished with your schooling, I will train you myself. How does that sound?”

 

“Will my mom and dad let me?”

 

“Your father keeps looking this way, so how about we ask him right now?”

 

They completed their trek to where Lyndon and Nash had resumed their work. Two Skeleton labourers were working a crosscut saw through a pine tree that towered at least 50 metres above them. Lyndon walked out to meet Ludmila and Jelena where they were standing out at a distance, while Nash remained to supervise the work. The other teams were further along the edge of the trees. A single Death Knight stood watch over them from nearby while a pair of Bone Vultures soared overhead.

 

“I spoke with Jelena,” Ludmila answered the unasked question on Lyndon’s face. “She told me she would like to train as a Ranger.”

 

“...a Ranger?” His expression froze, “Y-you mean like an Adventurer?”

 

“There are several avenues that a Ranger may pursue,” Ludmila replied. “For the time being, she will be serving in Warden’s Vale until she has received sufficient training. After that…it will depend on the needs of the fief.”

 

“But how will she train? I didn’t know there were any Rangers here.”

 

Ludmila held the frown forming on her face at bay as she pointed to herself.

 

“Eh? You’re a Ranger? I-I thought you were a Noble, m’lady.”

 

“I am beginning to think that I have all sorts of preconceptions attached to me,” Ludmila muttered. “I am a Noble and a Ranger. House Zahradnik was founded by an Adamantite Ranger, and all of his descendants were Rangers as well.”

 

“Beg your pardon, m’lady,” Lyndon lowered his head. “If she’s to be a Ranger, does that mean she’ll be fighting?”

 

“You will be learning what a Ranger does through her eventually: some of it you may even find familiar. As for fighting…yes, I will make sure she is trained for combat. She already shows clear aptitudes as a Ranger. If she is truly exhibiting ‘tells of the blood’ as you suspect, she will be strong. But that is only something that we can discover with time.”

 

“Of course,” Lyndon said. “Will you be taking her away with you?”

 

“Not right away,” Ludmila replied. “She has to finish her schooling first. After that, she will move to the harbour for training. You will have plenty of opportunities to visit each other – the harbour is not so distant, and her training will see her all over the demesne: including your village.”

 

“I see. That’s a relief. It’s all a relief. Thank you, m’lady…for taking care of my girl. For everything, really.”

 

A tear trickled down Lyndon’s cheek and Ludmila frowned. She did not think it was something that was enough to cry over, though she supposed that she didn’t really know how long Jelena’s family had agonized over the matter. The woodcutter wiped the moisture away and scrubbed his nose, standing awkwardly in front of her for a few seconds before clearing his throat and speaking again.

 

“You’re a good lord–er, lady, Baroness. Truly.”

 

“I only did my duty, Lyndon. Any liege would–”

 

Ludmila abruptly fell silent at her own words. Though it was something she had fervently believed all of her life, recent events had definitively proven that not every noble took the same approach to their duties, or even upheld them at all. Before she could think of a better way to frame her response, Lyndon spoke again.

 

“I was thinking back on what I said over lunch when you asked me those questions,” he said. “I said it simple, and after working for a bit it really sunk in just how rare this is. A place that’s safe for our families, where we ain’t scared for food or our lives, where the people don’t look down on you for worshipping The Six. Before living here, there was only one place I knew that was like this and that was Corelyn Barony. That was where we were hoping to go – where all the other folks at the guild and the temple wanted to go – before the priest suggested we come here instead.”

 

“That should gradually become the norm, I think,” Ludmila replied. “As I said, the armies of the Sorcerous Kingdom see to the security of the realm now. Undead labour and magical assistance are improving productivity across the realm substantially. Even the faith of The Six is growing, so more and more of the citizens will come to see things our way.”

 

“That may be so,” Lyndon said. “I mean, if m’lady says so, it’s probably true…but that’s not everything. I used a word back there – control – and I think I didn’t quite put it the right way to you. I talked about how strange it was; how it was the opposite of what we were used to, how the places we came from before mostly left us to ourselves. What you said after…I think I should try again.”

 

Lyndon looked down as he shifted his weight and was silent for several moments, then he faced her to speak further.

 

“When new folks arrive, it is strange here – it’s like nothing we’ve seen before; nothing we’re used to. It makes a man feel lost; unsure about everything…at least for a while. That’s how most of the villagers are right now: they got that sort of empty, lost look to ‘em, but that’ll pass. Now I can’t speak for the people working out in the port, but the rest of us don’t stay that way for long. I said that, without a doubt, everyone knows everything is yours: but I didn’t mean to make it sound like you were some sort of iron-fisted tyrant. Maybe instead of control, I should have used order.

 

“Now I’m just a woodcutter, so I don’t have many fancy words to tell things, but even a guy like me has eyes to notice what goes on around them. It’s like we can see your hand moving over the land, your words and will becoming reality before our very eyes. Where we came from, that hand only appeared to take away our men for the levy, or our livelihoods for taxes – leaving us with aching hearts and hungry bellies. It never gave us anything in return, leaving us to the wolves and bandits and monsters.

 

“I said it was the opposite here, and I mean it. Here, we see that hand every day. Instead of only appearing to take away, it gives to us first. Everywhere we look, we see it gesturing; pointing. Here are the homes you’ll sleep in. Here’s the village you’ll raise your families in. Here are the walls that will shelter you. We look to the fields and see how there’s so much food coming that we could never even hope to eat even a small part of it. The Death Knights tromping around and those bony buzzards flying overhead, making sure nothing comes in and gets us. We see the sail of your boat going up and down the river, and the shipments coming in without fail – even that dry old granny Lich making sure everything’s workin’ proper…it’s like you’re saying that you haven’t forgotten us, even if you’re not there with us.”

 

“I see…” Ludmila said softly, “Thank you, Lyndon. The idea that the people will eventually see things that way is a great relief.”

 

“Oh, there’s one last thing,” the woodcutter said. “I guess after all that maybe I should’ve used it from the start to describe how I felt.”

 

“What’s that?”

 

“It is a word you use: I hear it all the time, whether from you directly, or something that’s read to us. Tenant.”

 

“Because that’s what you are?” Ludmila furrowed her brow in confusion.

 

“We are,” he agreed. “And more’s the point. Where we come from, the lord and his men; even the merchants...they all call us peasants. The city folk, too. When we came to E-Rantel and holed up in whatever cheap place we could afford, they had half a look and said the same. Even though we’re all freedmen just like them. Even a simple man like me understands that words have power – names have power; they change the way people see things and think.

 

“Now I know it’s true that tenants that cultivate the land like us are peasants, but peasant means something else to the people that throw that word around. Peasant’s like something you stick in a bucket, along with the night soil and rotting garbage that you want to throw out quick. It’s a word that tells you that you’re below, like serfs and chattel. You hear it from anyone and it already tells you that, in their hearts, they’ve dug a ditch to throw you in, or built a wall to keep you out.

 

“Tenant means you’re a freedman under a lord; that there’s a deal between the two of you: labour and rent for land and protection. Duty and obligation. Lords with all the land and power can break that deal whenever they want, and all we common folk can do is keep our heads down and try to keep our families alive, or try to move somewhere else if we can – if the lord don’t try and bully us into staying. Maybe that’s why they started using peasant: because they saw us as powerless nothings; that they can do whatever they want, and leave us to our fates.

 

“Whether a noble uses tenant or peasant is the difference between heaven and hell: it tells you whether they’ll try and keep up their end of our deal if they can, even when times get hard. Here, in your lands, we see you doing that every single day.”

 

“Pa!” Nash’s voice drifted out from the edge of the trees, “This one’s ready to go down!”

 

Lyndon turned his attention towards his son, making an exaggerated gesture of some sort.

 

“A-anyways,” he said as he turned back to speak to her, “that was about half a year’s worth of words…so uh, back to work. Don’t let my wife know I talked your ear off like that – she’ll probably strangle me.”

