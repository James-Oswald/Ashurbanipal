\chapter{Ludmila Zahradnik}

Ludmila continued her inspection of the village’s construction in silence, uncertain what to say. The Volkhv, for his part, appeared content to follow and observe the goings-on around him. They crossed over the bridge into the common area of the village, where he drew the looks of everyone they passed. The scent of meals being prepared wafting from the various homes prodded her silently onward: after walking around all afternoon, she looked forward to returning home for dinner.

 

After checking with Nonna one last time to see if there was any business to handle that had arrived since she had held court earlier in the day, she was back on the road headed to the harbour.

 

Though the evening sun had already disappeared behind the valley’s western ridge, its light still painted the cliffs across the river in a fiery evening glow. One of the pairs of Death Knights came up the road from the port village, returning to the logging camp to pick up their next shipment. The Volkhv tested the air as the row of Undead beasts that followed passed. After the sounds of the Undead procession faded in the distance, Ludmila took a deep breath and decided to broach the subject of the missing Ranger.

 

“The Ranger of yours that went missing,” her voice came out carefully, “I am fairly certain it is dead.”

 

“Yes, that’s what was reported,” the Volkhv replied. “She remained behind to provide the opportunity for the others to escape, and was last seen fighting one of these Death Knights. She was a good friend who I had known since I was a lamb. Scrying for her did not turn up anything, so I considered it highly likely that she was no more.”

 

Awkward…

 

“You have my apologies,” she said. “I was not present when it happened, but the Undead defenders here are under my orders. It is my responsibility.”

 

“This is your place, Warden,” he said. “Stating the obvious is unnecessary. Apologies are also unnecessary. In this land, you maintain order and balance – if this is a result of that, then that is what it is. There is no reason to apologize for taking your place in the world; no reason to explain.”

 

“You are not angry that I killed your friend?”

 

“Anger…” The Volkhv seemed to mull over the word, “Our kind is not prone to anger. Her passing is unfortunate, but what is, is. If it makes you feel any better, if she was not slain, then I would not have come when I did. If I had not come when I did, then I would not have discovered the existence of a Warden here. Druids and Rangers serve a greater order, and to all things there is a time.”

 

Why was she the one being consoled instead? Again, Ludmila understood what he was saying, but the Volkhv’s calm acceptance of events seemed strangely fanatical in its own way. Ranger or Druid or otherwise, one still felt grief and anger and pain that rose in reaction to death and loss. At least if they were Human.

 

“Being that it is what it is,” he continued, “you will accede our request to become a part of this place?”

 

“I do not believe it would cause any issues with the land itself,” Ludmila replied, “so, personally, I am inclined to agree.”

 

“There is a shadow that lies over your words.”

 

“There are some things that need to be arranged in order for your people to dwell in my territory for the long term,” Ludmila told the Volkhv. “You are right that, as a Ranger, I serve the natural order of the world, but I am also a noble who serves another order.”

 

They continued forward for a short distance as the Volkhv stared absently into the sky. The quick, loping gait that he assumed to match her stride made the antlered ball of fluff bob up and down in rapid rhythm, but he showed no signs of fatigue. The eye on his forehead focused on her again.

 

“Noble…this name is not expressed in any concept that is familiar to me.”

 

Ludmila wondered if this was something that happened often when speaking to nonhuman races. The Undead seemed to understand her perfectly fine. She thought for a moment, wondering how to convey what it meant.

 

“You are aware of many races,” she said, “do the Dwarves you mentioned not have this? The Humans of Re-Estize to the north of your home certainly do.”

 

“We watch the world as it passes through magical means, yes,” the Volkhv said, “but it is mostly visual observation that does not convey language.”

 

“I see…then, among my race – Humans – it is a type of leader,” Ludmila told him. “Demihuman societies also have something similar: individuals that rise to power, who may oversee and direct many. You called me a Lord of my race…it is very similar to that, as I have recently discovered.”

 

As she explained what it meant to the Volkhv, Ludmila wondered why she had never considered it in such terms before. The abilities she was made aware of and explored in the past months certainly fell in line with those observed of Demihuman Lords. Considering how common the recognition of Demihuman Lords was, even amongst Humans, the fault of this blindness was probably the result of Human culture. Even without the faith of the Six Great Gods, it now felt that the Human cultures that she knew of promoted Human exceptionalism almost uniformly. Humanity was held distinct from ‘Demihumans’ and ‘Monsters’, and this perception may very well be the thing that prevented nobles from consciously realizing abilities that lay dormant.

 

If one framed Humans along the same perception that Humans had of Demihumans, Ludmila was quite literally a Human Lord: stronger than the average Human, and possessed of abilities that revolved around her leadership role. If not for being coaxed into this realization by Lady Shalltear, a nonhuman being, she may have just lived her entire life not understanding this fundamental similarity.

 

“You were born as this, in the fashion of the Zern?” The Volkhv asked, “Or is it something that manifests later, like with the tribes of the wilderness?”

 

“I don’t know anything about these Zern, but I was born a noble,” Ludmila answered. “Most nobles are born, though it is not unheard of for those that distinguish themselves to become them as well.”

 

“Hm…that is most peculiar: it is usually one or the other amongst the myriad species we have observed. Then, as a noble, what is this additional order that you serve?”

 

“Nobles serve under their liege – another noble – who, in turn, serves under their own liege. The sovereign reigns over all, and it is through those that serve the sovereign that the order of the realm is established. I am a noble of the Sorcerous Kingdom, my liege is Lady Shalltear. Lady Shalltear’s liege is the sovereign: His Majesty the Sorcerer King, Ainz Ooal Gown.”

 

“The existence of similar arrangements is known from our recollection of other peoples,” the Volkhv said. “It is not something we ourselves have, however, as we are all born with an understanding of the greater order of the world. I suppose to those races that do not, such individuals would manifest to manage them. What is it about this order that precludes us from becoming a part of your place?”

 

“The order of the Sorcerous Kingdom is mostly established through sets of laws that address appropriate conduct and limit or stop certain activities if they are perceived as undesirable. There are carnivorous species which exist here, such as Ogres, but none share this same sort of mutual relationship that the Krkonoše do. Due to concerns that some species might predate on others, a law was recently passed regarding this: the subjects of the Sorcerous Kingdom are prohibited from devouring other subjects.”

 

“But we would no longer be ‘subjects’,” the Volkhv noted, “just corpses.”

 

“I do not think the law makes any such distinction between one and the other,” Ludmila said. “There are laws regarding corpses, as well. I do not think it should take very long to accommodate your people’s unique relationship – in the meanwhile, you may explore my territory to find a place suited for your people.”

 

“We have already located such a place,” the Volkhv replied, “but we would have continued onwards if I had not encountered you here.”

 

“You have?”

 

“Yes, we have divined the path before us using magic – we know the layout of lands ahead before we actually arrive. This land was occupied by many strange things, which is why we sent Rangers ahead to investigate.”

 

“Alright,” Ludmila said, “where is this place that you wish to settle?”

 

“Across this river, there is a place of many high meadows. Beyond, more people resembling your own in the green riverlands. Beyond that…a land where life and death are overturned, and the living feed on the Undead. This place of high meadows is a part of your land?”

 

Her gaze turned to the towering cliffs beyond the Katze River. It technically was, but, due to the nature of Human civilization, her claim was nominal at best. Holding land was not as simple as marking lines on a map. How much an administrator could control depended on whether they could project power over an area, and whether it was worthy of development and the accompanying security costs were what actually determined whether one could assert de facto ownership over territory.

 

It was extraordinarily rare for a noble to control every acre of de jure land that was marked as theirs on a map. Before the advent of the Sorcerous Kingdom, only House Corelyn, Wagner, Gagnier and Vintner – whose territory had been abandoned after Katze – could claim such a feat in the duchy. They were all small, fully developed inner territories with the additional benefit of bordering the duchy capital.

 

Perhaps the most famous example of this aspect of territorial control, regionally, was the Katze Plains. The Kingdom of Re-Estize, the Baharuth Empire and the Slane Theocracy all made de jure claims over the vast territory which – if not for being an cursed wasteland dominated by the Undead – should have been a fertile river basin which provided access to the inland sea in the southeast. Yet none of these nations was capable of retaining their hold on the area, nor were they able to cleanse the cursed lands: not even the Slane Theocracy.

 

What resulted was an empty and barren plain that each of the three nations had marked as theirs on each of their own maps, though no one actually held it. All that the ridiculous claim amounted to was a point to rattle swords over in diplomacy; often used as a convenient excuse to hold the annual skirmish between Re-Estize and the Empire.

 

The mountains between Warden’s Vale and what was now Corelyn County came with a set of challenges which resulted in a similar situation. Re-Estize had claimed the range as a part of their kingdom on the maps, but even as it was passed along to the Sorcerous Kingdom with the rest of E-Rantel, access was extremely limited from both sides. It was a wild, alpine, area – inhospitable to Humans without exorbitant investment – so none actually lived there.

 

As House Zahradnik was the closest frontier territory, it was nominally in her sphere of influence, but no attempts at projecting power over it were made even at the height of Re-Estize’s expansion in the region. It was uncontested by the Theocracy, so if the Krkonoše occupied the area as her subjects, then it would be both de jure and de facto a part of her land. With their assistance, some development and exploration might be possible as well.

 

“I will extend my management over the range if you wish to move there,” she said after thinking on her reply. “How long would it be until your people arrive?”

 

“Once we have your leave to become a part of this land, Warden, they should arrive over the course of two or three months.”

 

“Will they be in danger from this evil that drove you from your land?”

 

“We have scryed the places we left behind,” the Volkhv replied. “Of any pursuit, there is no sign.”

 

“Then what of the other peoples that fled?” Ludmila asked, “You mentioned that there were many affected by this ‘evil star’.”

 

“They would not enter the mountains,” the Volkhv answered. “It is faster to flee through the hills and forests to the east.”

 

Ludmila frowned worriedly at his words. While the Abelion Wilderness was often thought of a savage and uncivilized place, the truth was it was home to probably millions of Demihumans of various species. Any strong displacement between the tribes would result in widespread consequences both within the wilderness and to the nations that bordered it. While the Theocracy might have the strength to handily weather such an event, others might not.

 

As the defender of the reach, House Zahradnik was charged with guarding the realm against any of the waves arriving at the border as a result of this disturbance. In the worst case scenario, she would be facing a mass migration that boiled down through the upper reaches and over the passes. If the Demihumans were aware of the strength of the Theocracy and opted to avoid a confrontation with them, this might very well be the case.

 

“How long do you think it will take them to come to the range running south that you passed by on the way to this place?” She asked.

 

“That is difficult to say,” the Volkhv answered. “They are a multitude of tribes of various forms – there are even small confederations. If one group brushes into another, they may both flee, or they may fight if one side does not recognize the threat coming from the west.”

 

He had a point: the potential numbers that would arrive at her borders might not be as great as she thought. While those who had been directly affected by whatever had befallen the western portions of the Abelion Wilderness might share the same fears, those in the way who were unaware of it would have varying reactions to the intrusion upon their lands. Some might listen and flee as well, or at least allow those fleeing to pass. Some might see it as a challenge to their territorial claims and war would result.

 

Rather than being one huge migration, the waves would likely become erratic ripples which arrived over a much longer span of time. This would still be a problem for her to deal with, but at least she thought it would be manageable. While she had few doubts over whether the forces of the Sorcerous Kingdom could defeat these Demihumans in open conflict, tracking down so many and keeping them from spilling over into the realm was another thing entirely. If they arrived slowly, and in limited numbers, it would be much easier to intercept and contain them.

 

Ludmila had looked forward to relaxing during the evening while she absorbed the information about the new aerial transportation network but, now, she had a whole new thing to worry over.

 

“My liege should be visiting within a day or two,” Ludmila said as they descended the slope to the bridge. “I will speak with her about your request and, if she believes there will not be any issues, we can have your people begin their move across the river. Are you able to stay overnight? It would be prudent to have you stay for her arrival, just in case she has any further questions for you.”

 