\chapter{Ludmila Zahradnik}

Ludmila left Ostrik to his work, stepping out from under the makeshift forge and into the light of the blazing sun overhead. It was plainly uncharacteristic weather for this side of the border ranges, especially in the transitions between winter and summer, which had the tendency to be overcast and wet. It wasn’t any mystery as to what was causing it, however.

 

Due to the slow start on the spring planting season, Lord Mare had altered the weather slightly throughout the duchy in order to ensure the crops grew and ripened on time. Well, it was slightly altered when it came to the lowlands, but the weather in Warden’s Vale had changed by quite a lot as a result of being downwind from everything. Ludmila’s territory only represented a small fraction of the land in the duchy, so she couldn’t reasonably ask Lord Mare to change everyone else’s weather on her behalf.

 

Between Lord Mare’s control of the weather and the magic he had cast over the fields once the crops were sown, her oats were easily on their way to meeting the harvest schedule with previously unheard-of yields. The implications were abundantly clear to all who bore witness to what would have been considered a miracle in their former nation of Re-Estize. Though magic being used to assist in agriculture was not a novel idea by any measure, the sheer scale at which it was employed in the Sorcerous Kingdom was unheard of.

 

Continuing her way up the village lane, she looked down towards the pier where Nonna was still overseeing the loading of timber. Once the summer harvest was ready for transport, Ludmila was not sure if her single vessel would be enough to transport all of the grain before the next harvest in the early winter.

 

The most straightforward solution was to build more ships, which was something Clara and Liane were looking into: a part of Clara’s long term plans for Corelyn Harbour was to include a shipyard to construct more vessels for the river trade. Another option would be to restore the bridge to reconnect her fief to the rest of the duchy by land. Doing so would be expensive, however, and nowhere remotely near as cost-effective as having more ships.

 

Voices on the terrace below drew her attention: the families that had arrived were being guided by villagers that had volunteered for orientation duties. She watched them silently from above, thinking of the issue that Ostrik had brought to her awareness.

 

Ludmila wasn’t sure if he honestly did not know what he was attempting to describe or simply trying to be polite, but what he had said amounted to there being a lack of purpose and leadership in the fief. In addition, despite its necessity for the time being, her control over the flow of goods made people feel powerless when it came to the smaller elements of their daily lives. With her continual absences from her own territory, Ostrik noted that new subjects were becoming more and more directionless. Thinking ruefully to the words she conveyed to Momon during their trip to E-Rantel, she supposed that it not only applied to kings, but leaders in general. She still wanted to thank the Adventurers of Darkness for their help, but she had not seen them since.

 

As for what she could do about the current state of her subjects, several solutions came to mind, some of which would probably manifest on their own at some point – such as the arrival of merchants when the population grew large enough. Ludmila’s long-term goals, however, were still relatively undefined. Her personal duties to Lady Shalltear did not really suggest what she should be doing with her own fief, and matters regarding taxes and other mundane contributions to her liege would need to be discussed before delivery of the harvest was completed. For the time being, she had settled on finishing the extension of farmland along the valley but, once the golems being used in the capital were free to lease, the schedule would be pushed forward rapidly.

 

She mulled over her options. Seeing that the boat had nearly finished loading, she strolled over to the manor entrance where a Death Knight stood sentry, as well as…something else. On the opposite side of the doorway there was a giant Demihuman with feline features, which was nearly four metres in length. Ludmila looked from the cat-thing to the Death Knight and back again: given that it wasn’t attacking and looked decidedly Undead, it should be something she could ask Nonna about when she received her reports. Giving it one last look, she shook her head and entered her home.

 

“Aemilia,” Ludmila called as she entered, “did you see that cat…thing…outside the door?”

 

“I did, my lady,” her maid replied. “It was just standing there, and the Death Knight didn’t seem to care so I just went in.”

 

“Did you hear anything about it on the way up here?”

 

“No, my lady. I just came in from helping the Linum family move in next door.”

 

“I see…how are they faring?”

 

“Lluluvien and Wiluvien are ecstatic – they already love the Vale. As for their mother…she is unchanged, as far as I can tell.”

 

Aemilia rounded up her Skeleton assistants and led them out of the door. Shortly after her lady’s maid had departed, Nonna appeared at the door. The Elder Lich placed several documents on her desk as Ludmila seated herself.

 

“Is there anything that requires my immediate attention?” She asked.

 

“Not at the moment, no,” Nonna replied.

 

“Then…what is that cat outside the door?”

 

“It is a Squire Zombie,” Nonna said. “Beings that fall to a Death Knight are raised as Squire Zombies under its control. Beings slain by a Squire Zombie will be similarly raised as Zombies under the Squire Zombie’s control.”

 

Rather than a cat bringing in its prey to display before its owner, the Death Knight had brought in a cat. Ludmila wondered if it would keep happening.

 

“I…didn’t know they could do that,” Ludmila said. “I assume this has something to do with the report of Demihumans encroaching on the border?”

 

“That is correct,” Nonna replied. “Several Demihumans of this species appeared on the borders of the fields several days ago. A pair of Bone Vultures intercepted them before they intruded too deeply, but they were destroyed. A Death Knight then arrived and dispatched one while the other Demihumans fled into the forest. This Squire Zombie is the result.”

 

“I don’t suppose anyone asked what they were doing here…can we ask the Squire Zombie?” Ludmila said.

 

“They approached in the middle of the night,” Nonna replied, “and the servitors were ordered to guard the fields. Death Knights utilize the corpses of the slain to raise Squire Zombies; they are not the same individual that was slain.”

 

Ludmila tapped her finger on her desk idly. The Undead Demihuman outside her door was not of a race she had ever seen before, so it was difficult to speculate as to why they had appeared. Still, the security of her demesne – and the border of the Sorcerous Kingdom – stood as a priority: one of her most basic duties.

 

“Then they performed their task as instructed. As requested by your report, I submitted the order for new Bone Vultures before I left the capital, so they should arrive at any time, if they haven’t already…were there any more encounters?”

 

“None,” Nonna said. “Though the Bone Vultures are currently the best detection assets we have deployed on patrol duties – they were unable to sense the approach of these Demihumans until after they left the trees and exposed themselves on the fields.”

 

This problem was something that Ludmila was already well aware of. Lady Aura’s rough outline of the Sorcerous Kingdom’s forces came to mind once again: there was a distinct shortage of forces that were capable of reconnaissance work beyond direct observation of subjects with poor concealment abilities. Trackers and other advanced detectors were relatively few, and Ludmila had resolved to train Rangers to make up for the shortfall in her own demesne – perhaps the nation’s armies would find a place for them as well. For the time being, however, she was the only Ranger in her fief.

 

“Was the Royal Court informed of the incursion?” She asked, “What did Lady Aura have to say?”

 

“The report is pending further investigation,” Nonna replied. “Would you like to inform the Royal Court immediately?”

 

“No, I believe you have the right idea,” Ludmila decided. “I don’t want to waste the Royal Court’s time – minor incidents are my responsibility to handle, anyways…before we move onto regular business, I have one more thing to ask: why are there tents outside?”

 

“Several Humans arrived without the proper authorization to migrate into the territory,” the Elder Lich answered.

 

“Without proper authorization?” Ludmila wasn’t sure what Nonna meant, “Did they stow away onto the ship in the past few weeks somehow?”

 

“They arrived overland yesterday. As they appear to be citizens, I thought it best to wait for you to render a decision on their...disorderly conduct.”

 

It barely took Ludmila a moment of thought to understand what had happened. She closed her eyes and sighed – yet another oversight.

 

“Indeed,” Nonna said. “Why must Humans move without orders? Such senseless and wasteful behaviour.”

 

Ludmila didn’t bother correcting the Elder Lich’s misinterpretation of her expression, as it was actually accurate in a twisted sort of way. The sensitive balance she was keeping in Warden’s Vale depended very much on the idea that people would only appear when she could provide for them. She had no expectations of people randomly arriving overland simply because it was close to unthinkable for common folk to make their way through the previously dangerous routes that led to her fief.

 

Under the rule of the Sorcerous Kingdom, travel within its borders was fairly secure unless one went somewhere they weren’t supposed to go – at which point security happened to you. His Majesty’s Undead armies had cleared away all of the hostile elements that could threaten its people along roads both new and old, so independent travel without personal protective measures was something that the people would eventually become accustomed to and expect.

 

“I’ll speak to them after we’re done here and find out what I can,” she finally said. “Hopefully this won’t become as large of an issue as I think it could be...”

 

The door opened, and Aemilia appeared with her skeletal assistants. Seeing her mistress at work, her lady’s maid quickly instructed the Skeletons to empty their buckets of water before going to the kitchen area to prepare tea.

 

“Speaking of immigration,” Ludmila looked back to Nonna, “there are several things that should be kept in the ship’s hold, just in case the new tenants are not prepared for the journey.”

 

Several minutes passed as she outlined improvements to the sparse accommodations on board the ship, as well as updates to be relayed to the manor in E-Rantel to deliver to the cathedral and Merchant Guild.

 

“This will reduce our cargo volume,” Nonna noted.

 

“It should only be a moderately sized crate, yes?” Ludmila replied, “A little bit of hospitality on the journey will help towards acclimating our new arrivals. Speaking of which, how goes the transfer of citizens to the village?”

 

“Due to the size of the residences ordered, each home takes roughly a week to build. Between them, the construction crews are raising homes at an average rate of one per day. The waiting time for new immigrants has remained stable as a result…do they really need such large accommodations? These buildings are far beyond the standard cottage size of the other rural fiefs observed in the duchy, not to mention the difference in material and construction quality.”

 

“I have a lot of space and not enough people,” Ludmila said as she shuffled through the summaries piled through her absence. “I am also competing with every other territory for skilled labour in this insane race to stay ahead of the deflating commodity prices. Being so far from civilization puts me at a marked disadvantage. If I want to bring civilization here, then civilization must look like it belongs here.”

 

Between the dusty dreams piled up in the manor archives – which consisted of what was in the small locked cabinet where the demesne accounts were held – and her own understanding so far of the strengths and weaknesses of the Sorcerous Kingdom’s forces, the layout she had plotted for the farming villages was ambitious: at least if one considered the duchy’s previous state as a territory of Re-Estize. With the availability of cheap Undead labour, all one needed was land and resources, which was something the mostly untapped expanse of her fief had plenty to spare. What resulted was a strange concept far removed from the traditional appearance of rural villages anywhere in the region.

 

Instead of a manor to house a member of the gentry or some sort of other Human administrative agency, Ludmila had instead come up with what she in the end had settled on calling a Lichtower. It consisted of several sections: the main, central part of the complex was an administrative office of two floors where she would station its namesake. The first floor would be used to service the villagers, while the second would house the village archives. Two wings were attached to store reserve Undead for the village if required, as well as provide space for future government facilities. The tower itself was twelve metres in height, providing a commanding view of the village’s surrounding farmland for sentries. Attached by a drawbridge, which provided access over the main road, was the warehouse office where parcels from Bone Vultures would be received and either transferred to the storage buildings nearby or picked up by awaiting residents.

 

The village itself was divided into two raised sections, which lay on their respective sides of the main road running through the farming terraces. On one side was the warehouse area, while the other housed the residents, services and essential facilities of the village. A small market square existed as well, but would probably not find any use until trade established itself. Both sections were enclosed in their own respective walls, which were connected to each other through gatehouses that straddled the road between the two parts of the village. Each entrance from the road into those sections also had a small gatehouse of its own.

 

When finally completed, the haphazardly constructed hamlet with its buildings strewn over a large area would be gone; in its place would stand an orderly and defensible farming outpost which rose over the fields. The village would house roughly 200 to 250 residents in its fortified enclosure: a stronghold which stood safe from sudden raids.

 

In any other place, it would be absurd to call it a village, and insanely expensive to construct. The fortifications were intended to shelter her subjects and the village’s goods, allowing the powerful Undead forces stationed there to focus on eliminating attackers as quickly as possible on the field. With the raw martial strength sufficient to handily destroy major cities in other nations being deployed by a single village, Ludmila deemed it more than enough to deal with anything short of a large army – which would probably become a matter of national security rather than something a farming village was expected to deal with.

 

The village was not without issues, though. As Nonna had stated, construction times were markedly slow in comparison with the much simpler construction that rural villages usually saw. Then there was the recent ordinance sent out by the central administration to harden crucial points in national logistics against attacks from both land and air. She also needed to somehow make them defensible against magic casters of the fifth tier. The latter was still something she was studying: slowly developing a grasp on the possibilities and potential applications of various forms of magic.

 

The initial concepts which went into the village were clearly only drawn up with defence against attacks from land in mind. Currently, with as little knowledge as she held in magic and aerial combat, her only solutions involved using brute force to overcome such attackers: throwing what she thought might be effective against various types of intruders. Hopefully, she could consult with Lady Shalltear – or perhaps Lord Mare – over these problems at some point. Beyond this, however, she was fairly confident that it would do what it was supposed to, and word would spread that her fief on the border was a prosperous and safe place to live as the seasons came and went.

 

“Besides,” Ludmila smiled lightly, “you agreed that the village layout was far superior to those of the inner territories, yes?”

 

“For all intents and purposes, before the new guidelines came in for fortifications,” Nonna affirmed. “Considering the tendency of you Humans towards disorderly conduct, however, I am dubious that the additional long term purpose of these settlements will work as you describe.”

 

The additional purpose Nonna referred to was essentially enforced by the limitations of each farming outpost. Each could normally support and house a set population within its walls, so the excess would be encouraged to migrate. Each would hold one of the future schools that Clara had described, as well as facilities to train apprentices in all the trades that went into supporting a small settlement. Ideally, it would raise successive generations of skilled workers with a basic standard education. A substantial portion of these new generations would move out due to lack of space, and the capital of her fief would by then be ready to welcome them.

 

“I do not expect everything to work exactly as planned,” Ludmila replied to Nonna’s doubts, “but it should work well enough. Our systems will always be subject to refinement, so we should always be on the lookout for ways to improve on the development of the demesne.”

 

Aemilia appeared from the back of the manor with tea, placing the simple wooden cup on the desk.

 

“I suppose that now is a good time to see to all of the petitions around the village,” Ludmila said. “Nonna: prepare a list of all the villagers in this settlement that have requested an audience in the time I’ve been away – I’ll hold court in the farming village tomorrow. Aemilia, check if the Linum sisters are settled in: if one of them is available, I’ll have her deliver my summons.”

 

Taking a careful sip out of the steaming cup in front of her, Ludmila settled in for a long afternoon of work.