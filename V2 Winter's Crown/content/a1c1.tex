

\chapter{Ludmila Zahradnik}

“‘Twas a chill night, wasn’t it miss?” A cheery voice piped up beside her.

 

Ludmila looked to her side: the owner of the voice was a young woman who occupied the space in the hold beside her. Both her auburn hair and light spring blouse fluttered loosely in the wind as she juggled her nursing child from one side to the other. Ludmila’s mantle shifted slightly as she turned, and she offered a smile as she responded.

 

“Yes, the river is always chilly,” she said, “but we’re well into the spring now, so it won’t be so cold in the Vale.”

 

The sound of men and women beginning to stir from the shivering groups that they had huddled into for the night rose as the morning sun finally crested the ridge of peaks to the southeast. On the other side of her, Aemilia was putting away the blankets they had brought for the trip – which had ended up being lent to the people around them instead. The woman, along with her husband and child, had received Ludmila’s.

 

“Isn’t your first time, then?” The woman asked, “Sounds like you know how things are like out there.”

 

“I was born and raised there; I’ve just been away for a couple of weeks.”

 

“The gods must have sent you to help us, then,” her free hand moved in a reverent gesture. “I can’t imagine how things would have been if we hadn’t met you. Thanks again for lending us your cover.”

 

“It’s no problem, a good mantle is enough for this time of the year. I was worried more for you and your baby – it seems most of the passengers here weren’t ready for the weather.”

 

“Aye, seems so,” the woman agreed. “Don’t think we got any warning of it.”

 

Ludmila nodded lightly before she turned away, after which her smile slowly faded. Though she had been entirely occupied recently, she thought that she had adequately covered her responsibilities as an administrator. Now that she was on her way home and able to witness her work personally, however, she noticed various oversights; things she could have done better everywhere she turned – even before she actually arrived in her fief.

 

As the young mother had indicated, the migrants to her territory were ill-prepared for the journey; after a brief period observing the other passengers, Ludmila figured that they were ill-prepared for life in the highland valley as well. She had experienced the issue once already when Aemilia first came with her to her home, but for some reason it had never occurred to her to warn of the conditions that the migrants would experience before summer fully set in.

 

Watching her hopeful new subjects shiver in the cold as they huddled together between the cargo in the hold that night made her feel woefully inadequate. That she had even erred in such a simple matter felt outright neglectful in her responsibilities as their liege.

 

She felt a nudge at her arm and looked down to see an early lunch being offered by Aemilia. It was a meal prepared by Terah: between the food she set out for dining in the manor – which seemed standard for the nobles of the city – and the food that she packed for travel, it seemed that her Housekeeper was far more skilled at making the latter for some reason. It was so much better than the standard fare that she often requested her sandwiches and baskets for regular dining over the local cuisine.

 

Ludmila accepted the offered meal with thanks, and continued to observe the ship and their surroundings. Beyond Aemilia, the Linum family sat together in the hold: Lluluvien and Wiluvien with their mother between them. There was no sign of Ilwé’s condition having improved since being separated from Count Fassett. The dozen or so other passengers were all in the forward areas of the ship, putting the cargo in the hold between the Undead crew and themselves. Ludmila instructed her maids to not act in a deferential manner towards her on the trip, and had them all dress in common clothing to blend in and interact with the new migrants.

 

Three of the families were from the list of prospective tenants provided by Bishop Austine, while the fourth family was that of a journeyman weaver who had been contacted through the Merchant Guild. There were two farmers, as well as a cobbler, and the four families would eventually join the farming village undergoing reconstruction as more homes were constructed. For the time being, they would stay in the administrative village, using the vacant residences in the hill. There, they would receive orientation for life out on the frontier, as well as familiarization with Undead labour.

 

A low murmur rose as they sailed out of the gorge and the vista of Warden’s Vale opened up before the passengers. Spring had fallen fully over the verdant highland valley, and blossoms filled its flooded marshes and slopes. Ludmila, as usual, was looking out for changes rather than solely appreciating the natural beauty of her fief. To even a casual observer, her gaze would have probably held a tangible edge; Demihumans from the wilderness had been reported encroaching on the land, and she focused her attention fully with this idea in mind.

 

A conflicted feeling had risen from within her at the news: duty had called her far to the west and, shortly after answering that call, other duties demanded her attention in the south. While she relished in her solidified sense of purpose, at the same time she could not be everywhere at once. There was the Adventurer Guild as well…she wondered just how far behind she had fallen, and how they would treat her extended absence from her commitments with them.

 

A little over two hours later, she spotted the first problem as they pulled into the harbour. Rather than damage incurred by Demihumans or anything overtly threatening, this problem took the form of two tents on the flats where her ever-growing glut of timber stockpiles were laid out. It was something entirely unexpected, and she could not for the life of her figure out why the tents were there. They appeared to be stitched together from a patchwork of rough, undyed fabrics. A few men and women sat around a campfire as the boat sailed by.

 

Ludmila’s puzzled gaze lingered over the group before turning to the pier as the vessel made its landing. Nonna awaited at the shore with a clipboard in hand; the tome that she carried everywhere was placed into a small satchel slung over her shoulder. The points of crimson light in her skull did not appear to follow Ludmila as she hopped onto the wooden planks and approached, instead intent on watching the migrants preparing to disembark.

 

“Report to the manor after you’re done here,” Ludmila said as she walked by the Elder Lich.

 

As much as she wanted to ask about the unexpected arrivals right then and there, Nonna had her own work to do and Ludmila was loath to interrupt the exchange of cargo. Her single vessel was not nearly enough to keep up with the goods awaiting transport, and she didn’t want to personally make things any worse.

 

She made her way up the hill, stopping to check on the homes along the way. Upon finding vacant dwellings, one of her fears was put to rest. Her initial thought was that she had somehow committed an error while issuing her orders from E-Rantel, which had resulted in a shortage of accommodations for incoming migrants. Nonna would most likely be able to explain what was going on with the people at the tents. Continuing on her way, she stopped at the warehouse.

 

“Ah, Lady Zahradnik,” Jeeves greeted her as she appeared in the entrance. “Welcome back.”

 

The small Skeleton in his curious outfit bowed in greeting beside his lacquered black box. Within the warehouse, the aisles of shelves were partially filled, but the absence of dust overall indicated that things were being moved around regularly.

 

“Thank you, Jeeves. How are things going here?” She asked, “Have you identified any problems?”

 

“Inventories are flowing according to the schedules that you’ve outlined,” he replied. “There have been no additional demands beyond the projections…though our exports are being severely throttled – we still have plenty of space for storage on the flats below, but I am uncertain if it will be sufficient for the long term.”

 

“The supply of timber is solely from the clearing being done for the areas slated to be developed into farmland,” Ludmila said. “Once we’re finished there, the amount you see coming in should be drastically reduced. I don’t plan on stripping the valley bare, though I do intend to set aside an area for growing and harvesting trees at some point.”

 

“I understand,” Jeeves nodded. “If that is the case, then there should be no problem. Aside from that, a villager who came by informed me that a few strangers had been going around procuring cloth from the residents…I am uncertain what for.”

 

“That must be how those people below got the material for their tents – have you heard anything about what’s going on with them?”

 

“Unfortunately, I do not,” the Skeleton shook his head. “They have not come to interact with the village inventories at all. As far as I know, they have only been dealing with the Human residents.”

 

“I see…” Ludmila took one last look around the warehouse, “You’ve done well so far, Jeeves: don’t hesitate to inform me of anything you think requires my attention.”

 

Jeeves straightened at her words and smiled. At least she thought he smiled: without flesh it was impossible to tell for sure. His overall image, however, seemed to convey the idea that he was pleased.

 

“It is an honour to be of service, my lady,” he bowed, “have a pleasant day.”

 

Exiting the warehouse, Ludmila continued on her way. Following the ring of hammer on anvil, she soon found herself looking over the space where Ostrik Kovalev continued his labours. The smith was still working out of the portable forge which he had brought with him to the barony, but the workspace around him had changed drastically. A makeshift roof had been constructed over a length of the terrace, providing a warm and dry area to work under in relative comfort. Rows of crates filled with charcoal and bog iron were stacked in the spaces leading up to the forge itself, while beyond several bloomeries continued to burn as they smelted iron. There was also something else in the rear of the bloomery area which she had no idea about.

 

Four children scurried about the area – two boys and two girls – and one of them finally noticed her watching the whole operation as he came out to retrieve a basket of charcoal. He stared up at her with big blue eyes before running over to pull on the edge of Ostrik’s shirt.

 

“Whaddya want, kid?” He said absently as he continued to work.

 

The boy tugged more insistently, and the smith finally turned away from his work with a baleful gaze.

 

“Now listen here, you: I thought I–”

 

His voice cut off as he noticed Ludmila standing nearby.

 

“Lady Zahradnik,” he made to bow, wiping his hands on a cloth nearby, “I–”

 

“There’s no need to interrupt your work,” Ludmila said. “I just dropped by to see how things are going here.”

 

The smith nodded as he turned back to the forge, and Ludmila circled around to stand across from him nearby. Over the ledge of the rampart, she could see that nearly all of the piles being burned for charcoal were gone. The children continued about their tasks, tending to the clay bloomeries and keeping various parts of the workspace well stocked and clean. From her vantage, she could also see skeletal labourers pumping bellows attached to the furnaces.

 

“You’ve adjusted quite well here,” she noted. “You’ve even picked up a few apprentices.”

 

“With construction prioritizing the farming village,” he replied while waiting to reheat the piece he had been working on, “I figured I should make myself comfortable here. As for the kids, well…with as much as you seem to be planning on doing so far, I took as many as I thought I needed. I can’t say I could’ve taken any more though, we’ll need a bigger space for that.”

 

“I appreciate your initiative,” Ludmila said. “It was something I was going to approach you about sooner or later. Once the first village is done, there should be some time to construct a forge here while we wait for the woodsmen to clear away the next set of fields.”

 

“Ah, then, about that apprentice thing, my lady,” Ostrik said slowly as he resumed working. “There’s quite a bit of confusion as to how you plan on organizing your tenants.”

 

“How do you mean?”

 

“The way you’ve been running things here so far...how do I say it…” he said, “it’s uh, very controlled. For the time being, you’re providing for all of the needs of your subjects: food, clothing, sundries, fuel – everything. Nonna goes to visit the farming village twice a day, the villagers put in their requests, then Jeeves has the orders delivered via Bone Vulture or through a cart with those Undead Beasts if whatever they want is too heavy. Don’t get me wrong, I think it’s great: it’s all very orderly and clean and convenient, but it leaves your living subjects quite...detached, if I were to describe it.”

 

Ludmila furrowed her brow at the word. The system that she had devised was what she thought would be the best way to organize things based on the means she had at her disposal. Lady Shalltear had noted that it was a unique way of employing her resources, but Ludmila did not think she meant it in the way that Ostrik described. Her smith shifted uncomfortably and turned his gaze back to the forge.

 

“It’s, well…maybe it’s something like this,” he said. “The people are used to seeing shortages of things, or the price of services and goods to figure out how their little piece of the world is doing. It’s like that basically everywhere I’ve been. Here, though…here, you ask for something, and it arrives within an hour or two if it’s available. You put up goods for delivery, and something comes to pick it up and it disappears into the warehouse for Jeeves to take care of. They see all the timber harvested get carried by their homes and the ship goes up and down the river…but they really have no idea what is going on. Everything they do seems to vanish into an unknowable void, and that same void provides everything that they need.”

 

She thought over his words again, wondering what she could do to satisfy the lack which he described. Ostrik continued after completing the piece he had been working on, placing a new bar of iron into the forge.

 

“It was like that with the apprentices I picked up too,” he said. “Normally kids follow after their parents’ trade, and if their family has a feeling that they might not be able to survive doing the work for whatever reason they’ll try and send the kid off to apprentice in something that they think will work out.

 

“I was willing to take in those kids because I knew that you at least need this many – probably more – new smiths working eventually, but even so the things I have to figure out: how to provide for them, what to pay them when as they become more skilled in the trade, finding places for them to work…I really have no idea. Their room and board is provided for by you, so is their food and so is everything that they work with. This sort of formless future leaves people really wondering what’s ahead of them, and it can be quite frightening to some.”

 

“You understand that things are currently arranged the way they are due to the availability of goods and resources, yes?” Ludmila said, “My ability to provide for the population is limited at the moment, as there is only one ship transporting goods along the river – I need to keep our imports balanced to ensure the needs of the people are adequately addressed, so I can see how it can be perceived as my being…controlling. Even if I were to open a market in the village, however, the selection would be quite meagre.”

 

“Yes, I had a feeling it was like that, my lady,” Ostrik replied. “The people might understand that things are as they will be for a year or two as well. It might be because they’ve just moved in, or they’re being exposed to new things and getting used to life here…but they have the look of being lost beyond their immediate occupations. The things that they’re used to having as a gauge for how they are doing don’t exist, and there isn’t anything to show them where they fit within the framework of everything.”

 

“I see,” she said. “I believe I understand what you’re trying to say now…are you sure you aren’t some sort of Sage from somewhere?”

 

Ostrik laughed, shaking his head at her words.

 

“I’m no Sage, my lady,” he smirked. “Just a traveller.”