\chapter{Ludmila Zahradnik}

After completing the schedule for her petitioners, which had been awkwardly drawn out using the seat of the small stool as a writing surface, Ludmila came out to hand it over to the foreman in the yard. He subsequently started calling in the names on the list as their concerns were attended to in order.

 

The last time she held court in the farming village, it was before the events of Fassett County. The villagers were shocked and nervous over the fact that their Baroness had come personally to attend to their petitions at all, and the fact that it was in what was a decidedly less than formal setting made Ludmila feel that they were cringing away from her the entire time. All they could prepare was a sawed-off log to sit on where the Lichtower was to be built in the future, and Ludmila had brushed it off before sitting down and going about her business.

 

With the sun beaming overhead through the uncharacteristically warm spring day, Ludmila thought that her interactions made her seem personable to the villagers despite their uncomfortable reactions. Aemilia, however, discovered all of the splinters in her skirts when she returned home and flew into a panic about her image after what had felt suspiciously like a guilty confession over what she had done. Hopefully the stool here now would be acceptable…well, probably not, but at least it wouldn’t leave splinters to confess over.

 

Today, each household sent a representative, most having the regular concerns she was used to addressing in her years of experience in managing a village. Immersed as she was in not only attending to their petitions, but trying to more concretely define what Smith Kovalev had described to her previously, the morning and early afternoon passed swiftly. Each audience actually lasted at least twenty minutes, however, while some were twice as long. As she was – settled into her familiar routines – it wasn’t until the short figure of Moren Boer appeared through the empty door frame that Ludmila was reminded there was at least one irregular piece of business that required her attention.

 

The strange, pale, man was no longer pale at all: the long hours helping out his fellow villagers as well as his own work outdoors bestowed a dark tan that gave him a look much more in line with a person who laboured outdoors. The way he carried himself was still mostly the same, however, so Ludmila supposed that strangers might still offer him a wide berth if they could help it.

 

“Good morning, Lady Zahradnik,” he bowed two too many times before straightening himself again.

 

“Good morning, Mister Boer,” she replied with a slight smile. “I hear that you have been instrumental in getting the other farmers accustomed to their work here.”

 

“Ah, I’m not sure if I deserve so much credit,” he shifted uncomfortably at her words. “They were already quite curious after they settled themselves in, so what followed seemed a matter of course. It’s not as if I had to chase them around.”

 

“I see…still, I think things might have been far worse if I was not lucky enough to have you come to my demesne first. Were there any issues or improvements over the past few weeks that you would like to bring up?”

 

Moren licked his lips and rubbed his beak-like nose, glancing about before he leaned slightly forward, as if he was preparing to convey some great secret. Rather than whispering, however, he spoke in a clear and entirely normal tone of voice.

 

“Yes, actually,” he said. “I’ve been watching the goings-on around the village carefully and I believe that we’ll need more magic casters soon.”

 

“This is something that I was planning on looking into eventually, but is there any particular reason you are bringing it up now?”

 

“Hmm…how does one put it…” Moren frowned, “The thoughts are in my head, but I now find that articulating them is not something I’m so good at. My lady, you’ll excuse me if I sound a bit strange?”

 

“I would say that we are well past the point of being strange, Moren.” Ludmila replied, “Our life in the Sorcerous Kingdom is far removed from what we were accustomed to in Re-Estize.”

 

“Why yes,” he blinked several times, “I suppose that’s true. Life in your demesne is so peaceful that one tends to forget – none of the more usual hazards of life in Re-Estize exist here, that I’ve seen. No brigands, no crime, no fear of famine or plague…one day just very much flows into the next uneventfully. Well almost, but even the Demihumans that showed up a few days ago seemed nothing more than an interesting topic for discussion.”

 

“Were you in the fields when it happened?” She straightened in her seat, “What did you see?”

 

“I wasn’t, no,” Moren shook his head. “It was well into the evening so the entire village was, well, in the village. There were a few relaxing around the edge of the rise that saw what happened, though: apparently the Demihumans stepped out into the field where a pair of Bone Vultures raised the alarm and swooped down to intercept the intrusion. The Bone Vultures were destroyed shortly after, but a Death Knight arrived and scattered the Demihumans back into the forest.”

 

“Did anyone recognize what sort of Demihuman it was?”

 

“Every single one of the villagers here are from the inner territories,” Moren replied, “so our knowledge of Demihumans doesn’t go much beyond Goblins and Ogres. You’re not familiar with what came, my lady?”

 

“I saw the Squire Zombie made from the corpse,” Ludmila told him, “and it was not a Demihuman species from this part of the wilderness. The borders of the various tribes will change as the seasons pass, but it is usually the same ones fighting over the same territory with their boundaries shifting as a result. A completely unknown tribe arriving is rare, and I need to learn as much as I can if we have a new neighbor.”

 

“I-is that so?” Moren said with a thoughtful look, “Being a follower of The Six, I figured you wouldn’t care beyond chasing them off or exterminating them.”

 

“I feel I am getting that a lot these days…”

 

“I meant no offence, my lady,” he lowered his head several times in succession. “It’s just that followers of The Six have a notoriously zealous reputation.”

 

“Have any of the migrants expressed such sentiments?”

 

“Now that you mention it, no.” Moren replied, “There was very little in the way of a reaction when the Demihumans showed up the other day. I half expected them to grab their farming tools and dash off into the woods to chase after the rest.”

 

“It seems that we are about as enigmatic and unknowable to you as those Demihumans beyond Goblins and Ogres,” Ludmila smirked. “Anyways, you were talking about needing magic casters soon?”

 

“I was? Oh, yes…magic casters,” Moren cleared his throat. “Hm…alright…I suppose it would go like this: the Barony appears to have everything we need, but we do not have everything we need for when we have everything we need…”

 

Ludmila glanced to Nonna, who was standing along the wall nearby. The Elder Lich seemed to shrug and offered nothing.

 

“And you believe that magic casters are the solution to this…need?”

 

“Why, yes,” Moren replied. “It just seems to be the way of things, does it not? Once populations become large enough, they develop a taste for the world that magic casters bring. Magical services, conjured goods; magic items that provide utility and convenience…considering the remote nature of Warden’s Vale, it’d be prudent to have magic casters in residence rather than constantly rely on imports that are expensive, inconsistent and subject to the fees and tariffs that come with their delivery.”

 

Ludmila frowned slightly. The flow of thoughts uncharacteristic for a Farmer – even one that practiced some magic – stuck out like a sore thumb.

 

“Does this have something to do with the communications you have been receiving in the past few weeks?”

 

“Ah – I recognize that sharp look from before…it was that obvious, huh?” He sighed, his eyes darting around again before they met with hers, “Truthfully, most of this proposal was formulated by several of my former…associates.”

 

“You mean the ones you spoke about from your younger days?”

 

“The very same,” Moren nodded. “I haven’t been in contact with them for nearly a decade. One of the ones in my village discovered that I did not return from Katze and, on a whim, he sent me a Message. After he discovered what I was up to…well, that was it.”

 

Moren chuckled nervously, seeming to recall some memory. Pulling out a cloth, he wiped the sweat off of his face before continuing.

 

“A-anyways, I guess there’s no beating around the bush with you, my lady,” he said. “Have you heard of Zurrernorn?”

 

“No.”

 

Moren was taken aback at her prompt answer. He scratched his cheek as his gaze turned inward.

 

“Ah…um, hmm…usually people at least recognize the name, if not precisely what they are. I suppose being out on the frontier, it’s not really a concern – there’s little reason for them to appear here. Plainly put, Zurrernorn is a secret society that primarily delves into Necromancy. Due to certain past events, they are generally unwelcome in Human lands and operate as an underground organization for the most part.”

 

“What did they do?”

 

“Oh, a few things here and there…” Moren fidgeted with the damp cloth in his hands, “The membership is quite varied and widespread so there is little regulation. Each group is basically independent of the others, and I don’t even know anyone who has the slightest clue who the leaders are. More famously, the most extreme groups of Zurrernorn have plotted against entire cities in the past in their pursuit of necromantic knowledge.”

 

“So, in this pursuit of necromantic knowledge, they desire to migrate to the Sorcerous Kingdom.”

 

“Yes, that would be the short version, my lady.”

 

Memories of her first discussion with Moren about this organization of Necromancers arose in her mind. She recalled the fact that they had indeed tried something in E-Rantel some time last year: his mention of the event was later confirmed by the story of Darkness, as told by various people in the city.

 

“Do they understand that the Sorcerous Kingdom upholds the rule of law, and that they still can’t run amok doing whatever they wish, even if there are Undead all over the place?”

 

“Oh, I immediately made sure that they understood this,” Moren said. “Still, they wish to come. I think I’m in contact with seven or eight of them at last counting…do you know what happens if you receive several Message spells at once?”

 

“Not specifically, no.”

 

“Neither do I, and I don’t wish to find out,” he laughed weakly. “A-anyways, all they want is a chance to prove that they can fit in here. Many are tired of hiding: in fear of being shunned, or arrested or killed for what they’ve invested their lives into. The fact that Necromancy is embraced in the Sorcerous Kingdom – that the nation itself is ruled by the most powerful Undead being in the known history of the world – makes it nothing short of a beacon of hope to them.”

 

“Should they not just move into E-Rantel then?” Ludmila said. “There are few barriers to entry that I know of.”

 

“Well they are a secret society, so they’re naturally cautious given their less-than-friendly reception elsewhere. It’s unlikely that any of their senior members are a part of this group – it’s more likely that the first that arrive will be used as a canary of sorts. As such, they’ve chosen to come to a place where someone they know dwells to see how things go. If they manage to integrate without major incidents, word will spread and it is almost certain that more will come.”

 

The thoughts presented sounded reasonable. It also meant that this group of Necromancers would be situated in a place far from the inner territories, where they could be observed at little risk to most of the duchy while their worth as productive members of society was weighed.

 

“I suppose that seems prudent from their perspective,” Ludmila said. “If their goal is integration, what do they plan on offering to my demesne?”

 

“Hmm…yes, that’s where their collective proposal would come in,” Moren answered. “Aside from the few combat specialists that become Adventurers or take more militant paths, most casters learn at least a few useful spells and various skills as a means to provide them with an income while they pursue their long-term goals. Many have noted that, since you are effectively building your demesne from the ground up, there is a rare opportunity to establish a community of mages with a strong representation relative to the general population. Elsewhere, both their special needs and contributions may be ignored and drowned out by mundane perceptions: just like how magic is treated by Re-Estize despite all that it does for the people. It’s something that they would have never considered possible of a noble of Re-Estize previously, my lady, but, after seeing how you are advancing with the development of your demesne so far, I believe you possess the vision to create this – not to mention ample land and resources.

 

“There are many possibilities that come with a large population of magic casters. In addition to helping service the wear and tear experienced by Undead servitors, they also offer those means that earn their livelihoods. They can conjure goods such as paper and spices, scribe scrolls for all sorts of spells useful in daily life, and assist with the day to day operations of the fief with their magic. Some of them are even capable of enchanting and will sell what magic items they are capable of creating at the markets here. As the community grows and its members improve, so, too, will the potential for magical industries.”

 

Ludmila examined the points she jotted down throughout the proposal. Barring the stigma that accompanied the organization these mages were from, she thought that it was an extraordinarily attractive offer. As things were, the availability of magic casters was essentially nil due to the demand for them in the city and inner territories. If Moren’s associates adhered to the order of the realm, she could not see any reason why she should refuse them.

 

“Did they make any specific demands?” She asked.

 

“D-demands?” Moren looked up at her worriedly, “No…they wouldn’t even dare lest it result in their rejection. If you’re referring to what they need for their work, that would be up to what each individual member does. To be safe, I would say a modest workspace and accommodations, if it isn’t too much trouble. In the future, you can confer with them directly for anything greater.”

 

“What about their families?”

 

“You would allow them this?”

 

“Of course,” Ludmila wondered why it was even a question. “They would just be professional labourers in the demesne. I do not understand why I would bar them from bringing their families with them if everyone else is permitted.”

 

“Ehm…that’s right, isn’t it? Then should I send word of your approval?”

 

“You may tell them that I am deliberating on the matter,” Ludmila replied. “I need to consult with my liege and ensure that there are no issues with their presence before anything else. It should still be roughly a month before the buildings of the first village are completed anyways, and any workspace won’t be ready until the Lichtower is done. There are new regulations concerning the development of magic and magical devices, so any related work needs to be in a secured space – we’ll be making alterations to the Lichtowers to facilitate that.”

 

“I understand, my lady,” Moren said. “I’m sure they’ll be relieved that they weren’t rejected outright, and ecstatic that you’d be so accommodating.”

 

“Oh, and one more thing,” she asked. “Are they able to teach magic?”

 

“I don’t see why not,” Moren answered. “As long as the student is capable of wielding magic. They even taught a poor farmer boy like me, after all.”

