\chapter{Ludmila Zahradnik}

The sounds of village life greeted Ludmila as she stepped out of the manor and into the shadows of the high cliffs. The steady ring of hammer on anvil punctuated the air, while gusts of wind periodically drowned out all else. To her right, the home where the Linum family dwelled was silent: door closed, but curtains drawn open. Deciding against interrupting their breakfast, she walked down the lane until she reached the warehouse and stepped into its open door.

 

“Greetings, Lady Zahradnik!”

 

Jeeves’ cheerful voice greeted her as she crossed the threshold, and she felt herself smile in response. The miniature Skeleton merchant bowed respectfully and awaited her reply.

 

“How have things been for the past day?” She asked.

 

“Mostly the same, mostly…” He looked up at her and tapped a bony finger on his clipboard, “I did receive several requests for dyes, though. There are none available in our inventories, as far as I am aware.”

 

“You are not wrong,” Ludmila told him, “The requests are probably the result of yesterday’s petitions. If anyone else asks, let them know that the orders will be sent to E-Rantel this morning, so they should arrive within a week or so. Speaking of which, have you seen Nonna? I need her to get these documents scribed in the city manor.”

 

“Yes, she went to review the timber inventories down on the flats,” Jeeves replied. “The ship should arrive sometime in the morning tomorrow, so she had one of the woodsmen come down with the latest delivery from the logging camp to ensure that everything is in order beforehand. That was a while ago, so she should be finishing up any time soon.”

 

“I should take a quick look around the village before she’s done,” Ludmila said. “Do you have anything else that requires my attention?”

 

“That should be everything,” Jeeves bowed again. “Have a pleasant day, my lady.”

 

Returning outside, Ludmila tried her best to remain inconspicuous as she wandered up and down the village lanes. She found that, unless someone came quite close, no one seemed to notice her presence. She wasn’t sure if it was because it was still gloomy outside in the shade of the mountains to the east or if her dress really wasn’t as flashy as she thought it was. On the other hand, Clara’s observation about how sneaky she had become might have had a measure of truth to them.

 

People occasionally came up and down the path. Those that seemed to be on errands had Skeletons following them. Several homes had groups of people gathered about, chatting amongst themselves. Unless they had been slated as permanent residents of the harbour, most of the migrants would have nothing to do aside from attending to family chores or getting used to life in her fief until they moved out to the villages. The waiting time until they moved into their new homes in the farming village was a week at most so, to Ludmila, it didn’t seem too long or short of a period to become accustomed to some of the new ideas presented in the Sorcerous Kingdom.

 

Following several minutes of wandering about, she found herself at the crest of the hill, standing amongst the stones of the village shrine. The space was kept well-maintained by the migrants who came to offer their supplications, and the surface of the stony hilltop was kept as clean as could be expected. Her gaze paused at each representation of the Six Great Gods, and Ludmila wondered how she was doing in their eyes.

 

According to Bishop Austine, the nobles were in the best position to influence matters of the realm and secure a place for Humanity in the Sorcerous Kingdom; both she and Clara thought that their activities in the past two months had helped to achieve those ends. For a noble of Re-Estize, the lines were clear and understandable, thus easy to observe and work within. As a noble of the Sorcerous Kingdom, however, she was constantly aware of the notion that the lines themselves could shift and change with little warning.

 

A quiet battle had started with this realization: one to ensure that any changes in the Sorcerous Kingdom did not have any adverse implications for humanity and her faith. For the time being, the faith of the Six Great Gods and their emphasis on the collective destiny of humanity resulted in many values which overlapped with those of the new administration: regardless of their station, everyone had roles in which they strove to excel in, and all roles held inherent worth in a vast system that drove growth and progress. The continued alignment of these values was crucial to survival in the Sorcerous Kingdom.

 

After performing her devotions, she shouldered her bag again and turned back around to find Nonna waiting below.

 

“Have there been any issues with outbound shipments?” Ludmila asked as she handed over her orders to the Elder Lich, “If there are any improvements that should be considered, we should be experimenting with their implementation before the harvest cements our schedule.”

 

“We’ve long maximized the usage of cargo space in the ship,” Nonna replied. “Obtaining new vessels for transportation along the river, or somehow improving on existing cargo capacity are the only apparent options from here. The coming harvest will present…difficulties if nothing is done to improve our tonnage.”

 

Ludmila sighed as they made their way to the village entrance. The problem of cargo capacity was a huge one for her fief: once the harvest was in, it would take their sole vessel over three months to deliver all of their excess produce to Corelyn Harbour. This meant that any other exports through the river effectively ceased unless they could construct or secure another vessel somehow.

 

She wasn’t the only one experiencing issues with transporting the harvest – if anything, the fiefs with expansive agricultural development had it far worse. The explosion in productivity meant that the existing logistics were woefully insufficient. Even with the availability of Soul Eaters to draw freight nonstop day and night, the vehicles for said freight were still limited in capacity and could only handle slow speeds. Liane had been buried in orders for new wagons, and she didn’t have enough raw materials or labour to keep up with the demand.

 

Even the addition of Frost Dragons were a pittance in comparison to the sheer volume of produce that would be harvested. According to Liane, it would take a year or so until they had enough vehicles to properly transport each harvest, provided that additional lands were not cultivated. For Ludmila’s part, she needed a dozen or so new vessels once all of the new villages were working.

 

“Maybe with the border to the Empire open now,” she muttered, “we can find a shipwright from the north…I have no idea when I would have the time for that sort of trip, though.”

 

Winding her way back down the lane, Ludmila decided to try hopping down the nearest terrace. It turned out to be remarkably easy, so she did so the rest of the way until she was at the base of the village and headed down the hillside towards the bridge. Making her way across, she spotted several sections of newer wood where the more dubious portions of the old structure had been replaced. It was a temporary measure, though. The entire bridge would need to be torn down and a stone structure built to handle the increased traffic it would eventually see.

 

The fief had seen a remarkable amount of change in the past two months, yet it always felt that the change was not fast enough. ‘Not enough capital, not enough labour, not enough time’, seemed to have become the tune she was constantly made to dance to ever since she started on the development of her demesne. On paper, she knew that each successive village completed would lead to a larger labour pool available to work on the next stage of development, but she always had a sneaking suspicion in the back of her mind that there was something she had missed, or something she could do better. That she was regularly made aware of shortfalls and oversights did not help much with her confidence.

 

She covered the ten kilometres from the bridge crossing to the village at a leisurely jog, arriving within fifty minutes. Considering she had done so in a dress – though admittedly it was clearly designed for her activities in mind – it felt nothing more than astonishing, and she didn’t even feel taxed by the end. In her childhood, Ludmila was always amazed that her father and mother could perform such feats, but it seemed that she had finally caught up with them.

 

The new village loomed ahead, its two parts built straddling the road. The road remained level with the surrounding field, carving a twenty metre wide gap between the common and warehouse areas. The entire area was formerly level with the road, and the village foundation an impossibility but for the assistance of Lord Mare. After witnessing what he was capable of with the Adventurer Training Area and Clara employing him to reshape the terrain for her new harbour, Ludmila had come up with a similarly fantastical plan for many parts of her demesne.

 

Lord Mare was quite interested in what she was doing to reshape the Vale, so he had readily agreed to help in his curiosity over the outcome. He raised the base of the new village from seemingly nothing; even fashioning the sewer system that flowed beneath. Once he was satisfied with the result, which ended up taking a whole afternoon of alterations, he promised to return when he was needed again and to see what the end result looked like. He appeared to have a great passion for shaping and building things, and his zeal in related activities was readily apparent.

 

So far, the buildings of the village rose on what resembled a miniature plateau standing four metres over the fields surrounding it. The stone walls of the two forts that straddled the road would rise an additional four metres, creating an imposing barrier that would be extremely difficult to assail in a conventional manner. The walls, with their battlements and towers, would come after the buildings in the interior had been mostly completed, so, for the time being, it was a curious-looking collection of buildings rising on an equally curious-looking outcropping of stone.

 

She ascended the long stone ramp to the common area, passing by the Death Knight standing guard at the entrance, and was greeted with the sight of people busy with the village’s ongoing construction. Men, women and children were busy at work in the market square, which was currently being used as a construction yard. Masons adjusted blocks of stone for fitting while carpenters fashioned frames, paneling, rails, furniture and other items. A kiln had been set up, and pallets nearby were filled with cinnabar-coloured roofing tiles. Labourers helping to put together the stone homes that lined the village streets appeared occasionally to cart away a load of one thing or another to their respective construction sites.

 

Ludmila thought she could sneak past the market area to look around unharassed, except nearly everyone noticed the more-than-conspicuous Elder Lich following behind her which, in turn, drew their attention to their liege. The foreman overseeing the yard stopped what he was doing and quickly walked up to them.

 

“Good Morning, Lady Zahradnik,” he said with a slight bow. “You’ve been away for a while…just come in?”

 

“With yesterday’s arrivals,” she replied. “There is a lot of catching up to do, so here I am.”

 

“So you are, my lady,” he somehow nodded using half of his body. “First floor of the Lichtower’s part way done, so you won’t have to bake in the sun holding court like last time.”

 

“I will be readying things there, then. Was there anything urgent that you needed addressed while you have my attention?”

 

“Hmm…nope,” he replied. “It’s about as steady as steady gets. Weather’s perfect and there isn’t much happening in the way of distractions. There was that nasty-looking Demi that popped up the other day, but a Death Knight took care of it real quick. Didn’t even know what happened until the entire thing was over, so work in the village didn’t even stop.”

 

A young girl walked past them with an Emerald Forest Slime on her head. They were a common sight in the fief, and Ludmila had issued instructions for one to be delivered per household and deposited into the sewers for waste management.

 

“...is that really safe?” The foreman said as his gaze followed the girl and her slime.

 

“I did that when I was a little girl too,” Ludmila replied. “They are monsters, but not the sort of monster that people have in mind when they think about them. They are quite placid; as long as they do not think they are being threatened – I would say that they are even more docile than most of the livestock people keep. If you communicate your intent to them, they will let you know whether it is agreeable or not.”

 

“Communicate…with a Slime,” the foreman’s voice was flat.

 

“The laws of the world that allow us to speak with others will work regardless of species – if they are capable of comprehending the meaning of what you say, that is. Despite their simple appearance, the Slimes here are also quite intelligent in their own way.”

 

In the bright sunlight, the Slime looked like the shimmering jewel that was its namesake. The blob jiggled along with the girl’s steps, seemingly content with sitting on her head. They disappeared behind the corner of a house, and Ludmila returned to the business at hand.

 

“I hope I will not be interrupting anything major once I start pulling people in for their petitions,” Ludmila said to the foreman.

 

“We’re your subjects, my lady,” he shrugged. “I ain’t never seen no noble worried about ‘interrupting’ anything before.”

 

“It is going to be busy for a very long time,” Ludmila said, “and small interruptions have a way of adding up. I will put together a schedule for you to work around with in advance.”

 

“Alright, just lemme know when you’re ready.”

 

Leaving the foreman to his duties, she walked into the nearby entrance of the Lichtower. Though it was to be the largest building in the village’s common area, it shared the same, simple style as the other buildings around the market square. The portion of the first floor constructed so far amounted to the basic structure of the office and the wings attached to it. There were still no floorboards or furniture, though someone had placed a single stool in the space for some reason – maybe they had put it there for her? The base of the tower and the stairwell which spiraled up the inside of it were also there.

 

“Does the office area look large enough to work with?” She asked Nonna.

 

“It is much more than sufficient,” the Elder Lich replied. “I have mentioned that buildings in this village seem overly…excessive. The administrative office is no exception.”

 

“Once they furnish it, it should look a lot smaller.”

 

“All one requires is a desk,” Nonna said.

 

“It does not hurt to have a comfortable, open space to work in. Besides, making everything tiny would create a meagre impression for the administration – it is hardly excessive.”

 

When it came to her own workspace, Nonna did not care for anything beyond the bare minimum of functionality. Be it a corner of a warehouse or just standing out on the pier, comfort and appearance were non-factors. Considering that most of the citizens which she helped to administer certainly did care for appearances, it seemed an awkward blind spot to have for someone in her position.

 

“Considering how you still feel this way about a lot of things,” Ludmila said, “I am worried how the new administrator here will get along with the villagers.”

 

“Our priority is a smoothly functioning bureaucracy,” Nonna sniffed. “Getting along can come later.”

 

“At least you recognize that it helps: the Elder Liches coming in will probably be freshly trained with no experience. They will be counting on your guidance to optimize their performance here.”

 

“I hardly see what that has to do with anything we’ve just discussed.”

 

“I think you have improved on this quite a bit yourself, Nonna,” Ludmila looked over at the Elder Lich from her stool. “You’ve changed quite a bit from when I first met you; perhaps that will be made apparent when the others arrive.”