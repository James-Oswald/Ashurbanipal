\chapter{Ludmila Zahradnik}

After deliberating for a few minutes, Ludmila decided to only bring a portion of her equipment along, fastening the hatchet to her hip beside her dagger; fastening her bracers and pulling on her gauntlets before grabbing her spear. With as many Undead as there were with the ongoing evaluations, she highly doubted there would be any Demihumans or monsters brave enough to stick their noses into the fields to cause trouble. Wild animals were generally even more timid, and all of the Undead activity would also keep them at bay. On her way out of the village, she saw that the woodcutters had sorted themselves out. One of them ran his gaze over her armament as she approached them.

 

“Why the weapons, my lady?” He asked nervously.

 

“Things may look nice and peaceful right now,” Ludmila smiled, “but this is still the frontier. Demihumans, monsters and everything else should be smart enough to stay away with all of the Undead around, but one can never be too sure. You’ll be working to clear up fallow land around the hamlet nearby for the next week or so. The Undead watching the area will have a clear view of your surroundings, so you should be safe: just run to the nearest Death Knight in the event that something scary comes along. I’ll be headed into the woods for something else after I get you all started.”

 

Ludmila rounded up the unclaimed Undead and led her hodgepodge of tenants and labourers across the bridge to the beginning of the fields. Being the closest to the harbour village they had seen the least neglect, yet still had new shrubs and trees intermittently sprouting up from where rows of grain were to be planted. She stopped on the road and turned around to address the woodcutters.

 

“All of these terraces, including the portions of the windbreaks encroaching on the fields, need to be cleared so they can be prepared for planting. There are teams of Undead working their way out from the hamlet that you’ll eventually come across, so all I need for the time being are the stands of trees removed and the timber sorted out for transport to the village – the teams of Undead labourers will take care of the rest.

 

The Death Knights that I leave with you will help as well – if there’s anything that requires heavy labour, such as stubborn stumps or roots, they should be able to take care of it. They’ll also be taking turns transporting timber to the village as it becomes available, so let them know how you’re going to be organizing our inventories down on the flats. There will always be one standing guard nearby, but pay attention to your surroundings once in a while just in case something comes along.”

 

The Bone Vultures she had sent ahead of them were already arriving to deliver the axes, picks and shovels that she thought they would need. The men took the equipment, fanning out into the fields, and Ludmila continued on her way. A few of the Undead flew overhead every few minutes, following her instructions to deliver pieces of wood removed from the fields to the smith. There was a veritable wall of debris running along the road that stood higher than her waist and three times as wide from all the clearing that had been done so far.

 

The original team of Undead had cleared the third field and was now working on a fourth, while the two sets of Death Knights working the ploughs were working through the second field. She stopped them as they passed by to see what their blistering pace had done to the equipment. In addition to the damage from their first run, more had accumulated besides. It clearly would not survive at this rate.

 

Ludmila instructed them to continue their work at half the speed. Until she could get her hands on something more durable, she didn’t want to deal with the delays involved in the replacement of something that could currently only be purchased in the city – if there were even any left. Walking a short distance further, she stopped where Nonna was watching over a second team of Skeletons working to clear a field on the other side of the road.

 

“What are you doing here?” Ludmila asked.

 

“Collecting data for a control sample,” the Elder Lich did not turn from her observations.

 

“What does that mean?”

 

“It is a baseline used to compare sets of experimental results to. In this case, I am collecting information to be compared to Undead teams commanded by you, and the group that will be controlled by this new Farmer.”

 

“I see,” Ludmila said. “Then I suppose I should leave you to your work.”

 

“Yes.”

 

“Actually, one thing: how long did the plough teams take to finish the first field?”

 

“Roughly 26 hours,” Nonna replied. “The measurements of the fields are imprecise; there was some overlap in the work.”

 

At their former rate, it meant that the two teams were ploughing roughly 9.5 acres per hour. It was an order of magnitude more than what a team of draft animals using the same equipment could achieve, not factoring in that the Undead were untiring and could work day and night. She figured that, even at half speed, the Undead she had initially requested would have only taken roughly a month and a half between clearing, ploughing, sowing and harrowing. There was still enough time to achieve two harvests this year, and something that could easily be accomplished if the preparations for the first sowing had been done in the winter. Of course, with four times the labour, it should be done in a little over a week. Ludmila still couldn’t quite believe it though – even as it started to play out before her eyes.

 

The two remaining teams of Undead for the farms had ordered themselves along the road going through the centre of the hamlet. She allowed Moren Boer time to pick one of the abandoned buildings to move into and took one of the groups of Undead awaiting instructions to get them started on their work. When she returned, the short farmer was standing over the well, looking down into it.

 

“Is there water?” Ludmila asked, realizing she hadn’t checked.

 

“Yes...I can see the sky’s reflection from below.” He answered as he straightened himself.

 

“The last group of Undead here will be under your direction,” she walked over to where they were waiting. “The other tenants seem to be able to handle four without much difficulty, but I’ve heard that excessive numbers can result in various problems. As you’ve probably heard me explain to the others, these skeletal labourers will carry out your instructions as you envision them but they cannot carry out tasks that are too complex. They will continue with their orders until they complete them or are instructed to do otherwise, so you may be able to manage them by assigning them to tasks that take long spans of time or may be repeated many times. I’m rather new at this myself, so as a Farmer, you may be able to achieve superior results with your expertise.”

 

Moren walked over to the formation of Undead, slowly examining them row by row. He gave an experimental order, and a group of four of them stepped forward. He tried again with the same results, looking over sheepishly at Ludmila.

 

“It seems four is my limit,” he said apologetically.

 

“Try what I mentioned earlier,” she suggested. “Set those four on their task and come back for more.”

 

As he moved to follow her advice, Ludmila headed over to the barn. The Undead Vultures that she had set as sentries were still watching over the building, and she spotted several dead rodents in front of the door. Just inside, there was a pile of broken tools and she used her foot to sift through the damage. There were a few broken hafts, but most of it appeared to be damage to the iron itself caused by their indiscriminate use by the tireless Undead.

 

She instructed several of the Undead that had followed from the village to load the pieces onto the nearest cart: she would take them on her way back to the village to see if they could be repaired. Nonna had reported that 15 percent of the equipment had failed, and they had completed ploughing roughly the same percentage of the fields so, even if Ostrik couldn’t effect proper repairs, they should still be able to complete the work with all of the equipment that the new teams had brought with them.

 

Moren had finished setting half of his team to task by the time Ludmila finished looking through the barn. She walked out to the field he was working on, and found that the Undead that had started clearing it were working slower than the ones that she commanded. Was it because of her ability? Or was it because Moren was still new at directing the Undead? The Farmer returned with the next group – his instructions seemed shaky, but he seemed to understand what needed to be done. She didn’t think one’s confidence would affect the results of their directions, but that was only a hunch.

 

He blew out a heavy sigh as he set the last group to work. Going back and forth between his Undead labourers, he would pause to issue corrections frequently. Maybe there was another reason for his slower pace.

 

“How often do farming tools break?” Ludmila asked him as he continued with his task.

 

“Ones like these?” Moren replied, “Hmm…they should last a year or two, at least. Longer, if you’re careful and take good care of them.”

 

A year or two? She was losing them at a far greater rate than that.

 

“Nonna told me that I’ve lost fifteen percent of my tools in the past day alone,” Ludmila said. “Do you have any idea why this might be the case?”

 

“They shouldn’t be breaking that often,” Moren said. “If the tools are from the same place…well, who gave the instructions for them?”

 

“That would be me,” Ludmila replied.

 

She looked over to the remaining Undead awaiting orders and commanded them to start clearing. Unlike Moren Boer, who could only direct four at a time, the entire group went to work immediately. The Farmer frowned.

 

“Is there a trick to that?” He asked.

 

“I’m not too sure, myself,” Ludmila answered. “You’re the first Human I’ve had to compare to, but it already appears that there are a few notable differences between us.”

 

“That might be just a bit of an understatement…” Moren chuckled, “Does that mean the other Undead out in the fields are also following your instructions?”

 

“Aside from the group being controlled by Nonna, yes.”

 

“Well…nobles are used to ordering around lots of people,” Moren mused, “maybe it doesn’t matter what they are.”

 

Lady Shalltear’s words to her about the differences in vocations and ‘classes’ came to Ludmila’s mind. When she had first heard them, extraordinary skills and feats filled her mind…but even things that she considered mundane appeared to be affected as well. If that was the case, she would have to consider things more carefully than she had initially thought she needed to.

 

“My Skeletons seem to be acting out their orders in a different way than yours are,” Ludmila said. “Could you take a look at them and tell me what you think?”

 

Moren walked over to where her Skeletons were clearing the field. It wasn’t long until he looked back up towards her.

 

“I think I know why your tools are breaking, at least,” he said.

 

Ludmila came over to join him, but she couldn’t see anything wrong with what the Skeletons were doing.

 

“Why is that?” She asked.

 

Moren looked at her strangely for a moment before his face lit up in comprehension.

 

“A few years back,” he told her, “we had some Adventurers come by our village for a job – Platinum rank ones. They used the village as a base for their work for about a week or so, and one of them got curious about our work. Eventually he joined some of us working out in the fields, wanting to try his hand at the whole ‘farming thing’. The man was strong and fast, but in the end what he was doing was basically the same as what these Skeletons were doing. We had to politely convince him to stop before he started damaging our property too much.”

 

“What was he doing wrong?”

 

“Well, rather than doing something wrong, he just didn’t know how to do it properly,” Moren glanced at the spear in her hand. “I guess it might be easier to use weapons as an example, since you’re a Frontier Noble? A regular guy will look at a spear and think that sticking someone with the pointy end is all there is to it. Then you watch someone who knows what they are doing and it looks anything but. Just like a spear is a tool for fighting, you have tools for everything else. If you haven’t learned the trade, then all you might think is that a blacksmith is banging metal with a hammer, a teamster is just riding along with their wagon, or a farmer is just digging holes or cutting grass. Sure, you might get the job done in the end doing what you’re doing, but there’s always a better way to do it.”

 

Ludmila was already feeling embarrassed halfway through his explanation. Especially considering it was her own explanation, just delivered in a more specific context. The work of Undead labourers relied on the expertise of their director: a Farmer would obviously know how to do their job better than she would, and thus be better at directing Undead labour used for farming. Even as she was making requests for various professions to migrate to Warden’s Vale, she only knew how things roughly fit together through her upbringing – she didn’t have an intricate understanding of how each piece of the puzzle worked.

 

“Er…I meant no offense, my lady,” Moren started to back away. “My mouth just ran off with me.”

 

“No, it’s alright,” Ludmila cleared her throat. “I’m just a bit embarrassed over how I’ve instructed you on how to direct the Undead, yet somehow the meaning of what I’ve said hasn’t sunk in for myself. You’re grasping these new tasks quite rapidly – when I first put up the posting asking for people willing to immigrate to my demesne, I didn’t think anyone would take to their work this quickly.”

 

Moren let out a shaky laugh at her words.

 

“A part of me thinks this is still some sort of dream,” he said. “I wasn’t sure what would happen.”

 

“Are you going to be alright all alone here?” Ludmila asked, “There will be more tenants coming in eventually, but you’re all that there is here until I make my next trip to E-Rantel at least. Being surrounded by so many Undead must feel unsettling for someone unaccustomed to it.”

 

“I don’t mind,” he said. “I find the Undead quite fascinating, actually.”

 

Ludmila looked at him sharply. That was definitely not something someone would normally say. Perhaps she could finally discover the root of the unease that he had generated in her maids.

 

“I think you should explain what you mean by that.”

 

Moren’s expression changed as he realized what he had confessed in his idle chatter. His mouth opened and closed several times before he settled on what to say.

 

“When I was younger, I had a…phase, of sorts,” he said nervously. “I took an interest in the Undead for some reason – they were something dark and mysterious that was far beyond my dreary life at my family’s farm. I got deeper and deeper into it until I fell in with a group of shady folks in the nearby town. I did things here and there for them: cleaning and other chores; delivering messages and such. They taught me a little bit of magic in exchange. I was too much of a coward to learn any Necromancy, though. I was raised under the ways of the Four Great Gods, and there was always that little voice tugging at my conscience that kept me from learning anything more than a few basic spells. I just grew out of it after that, but there’s still a part of me that finds their work interesting.”

 

“So you’re a mage, then?” Ludmila asked.

 

“Of a sort,” Moren answered. “Didn’t save me from the levy, though. The lord’s men came through and I got swept away alongside a bunch of other people from my home village. They marched us off the E-Rantel, where I got a spear stuck in my hand and a sergeant that hated everyone unconditionally. I barely survived Katze.”

 

Now that he mentioned it, his unstable presentation and pale look did resemble some of the survivors that returned to Warden’s Vale after the battle. Ludmila felt bad for him and for the disparaging comments that had been made by her household. The idea that he had fallen in with a group of Necromancers still made her wary, though. Such practices were taboo in Re-Estize out of concerns that the practice might lead to the rise of more Undead. It was an ultimately superficial sentiment, however, given the current state of the Kingdom’s former territory. As a noble commanding hundreds of the Undead; serving an Undead vassal of an Undead King, she made his past activities seem a pittance in comparison.

 

“But you still made it out alive,” Ludmila said. “Why didn’t you return to your family?”

 

“I came from the heartlands – a village under a Baron on the other side of Pespea’s territory. With the army shattered, it was every man for himself. A lot of the territories along the way home aren’t as well-managed as the ones in E-Rantel, and even this duchy had the signs of various problems that I could see on the way in. Many of the nobles don’t care or can’t afford to keep their lands secure. Even worse are the ones that are in league with the thugs and bandits that run amok in their lands. It was safe enough to come in with the Royal Army, but unless you’re able to pay for their ‘tolls’ or have an armed escort like the merchants do, you’re more than likely to end up dead or a slave if you try to travel alone. Knowing that, I decided to take my chances staying in the city.”

 

Ludmila’s eyes narrowed as he spoke. She was appalled at the idea that any noble could willingly suffer those conditions in their demesne; she was equally appalled by the idea that their own respective lieges could even allow it to happen. It was a clear breach of noble contract in Re-Estize to allow such lawlessness, never mind participate in it – how was it that such nobles were not immediately sanctioned or even stripped of their titles so they could be managed by more capable lords?

 

The lord of a fief governed autonomously, but that autonomy came with obligations to both land and liege. The story of the Linum sisters floated out of her memory: she had been unwilling to believe that such a thing could have happened, but if the rest of Re-Estize had places where these practices had become commonplace, then perhaps the Duchy of E-Rantel had been influenced by it to some degree as well.

 

“Anyways,” Moren continued, rubbing his arm in a nervous-looking manner. “I came across your posting the other day and thought that I fit your requirements in a strangely fateful way. I was a Farmer, and I wasn’t scared of the Undead. I thought maybe I could put everything I had learned to good work for the lord here – maybe learning some Necromancy would be useful?”

 

Ludmila had no immediate answer to his question as she had no experience with it. Considering their new reality, it was a certainty that the laws and customs surrounding the practice of Necromancy would be changed. It didn’t guarantee that one could do whatever they wanted, however…there should be regulations concerning its use, but such regulations would need to be laid out by those who were experts in its applications.

 

“Do you think there are more groups around like the one you fell in with?” Ludmila asked.

 

“Oh, I have no doubt about that,” Moren said. “In fact, I heard there was some incident in E-Rantel where some crazy fellows tried to do something in the cemetery. They were stopped by Adventurers, but where there’s one, there’s probably more – there were a few in this case.”

 

Moren stopped speaking to correct some errors in what the Skeletons were doing that Ludmila still could not discern. After observing their work for a minute, he returned to speak to her.

 

“Even if there aren’t,” he said, “word will spread about a nation that openly uses the Undead, ruled by a powerful Undead King. It will draw them like ants to honey.”

 

Ludmila frowned. That the Sorcerous Kingdom would become a popular destination for crazy Necromancers was definitely not something that had ever crossed her mind. She hoped she could deal with it when the time came.