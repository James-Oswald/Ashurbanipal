\chapter{Ludmila Zahradnik}

At the gloomy outburst, Ludmila furrowed her brow in confusion and felt a hint of distaste well up within her. Frontiersmen were a subdued and practical lot – the men and women of Warden’s Vale were not prone to bouts of melodrama like some unskilled tavern bard.

 

“Lost? How?” She pressed Milivoj, “You should have barely arrived.”

 

For years now, the Kingdom of Re-Estize and the Baharuth Empire regularly held a large skirmish around the autumn harvest season. Neither side would purposely commit to pitched battle and the outcome was similarly indecisive. The casualties inflicted could barely be called damage: so tiny they were, compared to the nation’s population of over nine million. Each year, the great houses would play their games, and their vassals would try to figure out how to deal with all the logistical shortfalls it would cause.

 

In line with the rather late declaration of war by the Empire, the King had raised his banners over a month previous. Since it was an expected occurrence by this point, Baron Zahradnik had already long prepared his quota of men and supplies and left that same week. Going by Milivoj's statement and the unexpected return of the men, this outcome should have occurred only a handful of days after their arrival at E-Rantel – barely enough time to finish setting up their camps and preparing for the initial skirmishes.

 

Ludmila couldn't accept this claim. With both sides putting minimal effort into the battle, it should have been a standoff that played out over many weeks – well into late winter, perhaps even spring. That should have been the point, after all: the longer the Empire lingered at Katze, the longer the Kingdom’s levies would be prevented from participating in the regular activities of the season. The Autumn harvest had long been completed by the time the call to arms had been sent forth, so it seemed that the Empire had chosen to both stifle the busy winter trade and delay the spring planting season.

 

It represented an aggressive shift in the Empire’s strategy: delaying the bustling winter markets in the region would have meant supplies would not reach the fiefs that needed them. The planting season – if postponed for long enough – would require that crops and livestock be organized in a drastically different fashion to account for lost weeks of growth, if they could be at all. To Ludmila, this sudden move to throw the Kingdom off balance indicated that the Empire was about to change the way they waged war some time in the near future, but the idea that they would face such a drastically different result in the same year was beyond her ability to imagine.

 

“Milivoj, answer Mistress Ludmila’s question,” Bohdan spoke gently to the man, trying to coax him to speak further.

 

The priest leaned forward, laying his hands on the haunted-looking man. Ludmila did not hear any spells being cast, but Milivoj seemed to visibly gather himself and started to speak again in a clear and regular tone.

 

What followed was the account of a nightmare. It could be only that.

 

Milivoj’s eyes regained their focus as he finished his recollection, looking between Ludmila and Bohdan as he ended his account. He seemed oddly calm, as if sharing his memories had distanced him from his experience somehow. Ludmila shifted uncomfortably. There were so many points where she thought she should have questioned, but it was all so fantastical and horrifying that she had refrained from doing so as the young man spoke. Even now, she was at a loss as to how to proceed.

 

The tale was long, and the use of divine magic to eliminate Milivoj’s fear and force lucidity upon his mind resulted in an eerie, distant account that spared them none of its gruesome details. Some of those, she could not even begin to understand...or perhaps words were simply insufficient to do so. Tens of thousands of men, falling to the ground dead as if a giant scythe had swept through them like so much grain. Colossal globes of liquid darkness that rained over the field, birthing colossal, nightmarish creatures that could only be described in loose relation to more common things. The appearance of an Undead caster who possessed the might to control those unfathomable monsters.

 

She couldn’t even guess at how many lives had been lost in the carnage. It was beyond the comprehension of a teenage girl living out on the edge of civilization with a scion’s education and limited experience outside the borders of their isolated fief. Beyond what any of them knew. Even those villagers who had spent decades defending the border against the myriad of Demihumans and monsters from the wilderness had never recounted anything so bizarre, horrifying and deadly.

 

But…Bohdan might know. The venerable priest had lived long and seen much, undergoing his training as an Acolyte in the Theocracy. A mentor and friend who had long served the barony and its people with his divine magic and generations of wisdom. Ludmila turned to see if she could lean on his knowledge and experience, but her words caught in her throat as when she saw the ghastly expression painted on his worn face.

 

“...spare us.”

 

The elderly priest was the first to break the silence that followed, with words barely audible as they passed his pale, thin lips.

 

“May Surshana spare us,” his voice rose as he invoked the grace of the god of death. “The army of the Kingdom, destroyed. The Empire, fighting on the side of abominable horrors and evil Undead!”

 

His quivering voice continued to rise and Bohdan turned to face her. The feverish gleam in his eyes was so intense that she felt herself take a step back.

 

“Mistress Ludmila,” he said, “we must leave.”

 

Ludmila’s mouth opened and closed, and she looked back and forth between the two men. Motioning for the priest to hold, she turned to Milivoj, who remained leaning on the wall of his home.

 

“Milivoj, please get some rest,” she touched him gently on the shoulder. “I’m sorry for asking you to remember all of that.”

 

Nodding in appreciation as she helped him from the wall, the young man turned to disappear back into the tiny earthen home, closing the door softly behind him. She wasn’t sure how much longer the enchantment cast upon him had left, but she suspected that she had just condemned the man to the state that they had discovered him in yesterday, despite appearances.

 

Stepping back onto the village path, she found the priest pacing back and forth on the narrow road, muttering to himself. Seeing her approach, he repeated his words.

 

“Mistress Ludmila,” Bohdan said quietly, “we must leave.”

 

“What do you mean by ‘leave’?” His unexplained insistence was beginning to wear at her patience. “You are the village priest. This is the seat of House Zahradnik. We can’t just simply ‘leave’ this place.”

 

“No!” His shout punctuated the morning air, then his voice returned to its normal volume, “No. Do you not understand, Childe? The Royal Army has been routed and broken. E-Rantel was being abandoned! The Empire, with their new unholy allies, will sweep over these lands – employing the very devastation they have wrought. When they come, those horrors will shatter our minds and bodies just like those men that returned home. The Undead show no mercy to anyone! Only death at the hands of this great evil awaits us if we stay.”

 

“But the Baron–”

 

“The Baron and his sons are unaccounted for,” he continued to press her. “This is a decision that you must make on his behalf. They would be powerless against what comes anyways: all of us would be. Everyone must leave – not just you and I – every man, woman and child must flee this place before it’s too late.”

 

Ludmila felt that the lingering dread that had hung over them since the previous day had, for the old priest, manifested into the same fear that came from the men in the boat. Ludmila swatted away the same feeling creeping out of the corners of her own mind as she tried to calm Bohdan.

 

“Perhaps it will not come to that,” she said. “It would be reasonable for the Kingdom to concede some territory in light of such a defeat, so it would not make sense for the Empire to cause such wanton devastation in their future holdings.”

 

Normally, when clashes happened between rulers, a decisive defeat would result in the concession of titles. The nobles managing the lands within those titles would owe their fealty to a new liege lord and continue to administer their fiefs under a set of obligations similar to the ones they had before. While she understood that recent changes had occurred within the highest levels of the Empire’s administration, it’s territories were by and large run the same way as the Kingdom’s: through a hierarchy of aristocrats and capable proxies who managed their respective territories.

 

Practically speaking, it meant very little to the common citizen, and for the nobles it would mostly mean that their taxes flowed to a different lord. No action would be taken against them as long as they administered well, upheld their end of the noble contract and observed the laws of the realm. In their case, E-Rantel and its surrounding duchy – being the claim which provided the casus belli for the annual confrontation – would be ceded, but life for the people would remain mostly unchanged.

 

Bohdan remained unconvinced, shaking his head.

 

“The risk is too great. Never in all my years have I even remotely heard of such a thing. This is no normal war…”

 

There was a short pause as the priest collected his thoughts, seemingly inspired by something. Ludmila waited patiently while scanning their surroundings: so far, no one had taken note of the discussion.

 

“This sort of magic that Milivoj described...it is far beyond that of even great heroes. I have heard that the Empire has a legendary magic caster – Fluder Paradyne. Only one such as he could have possibly done such a thing.”

 

Ludmila plied her memory, trying to recall what she knew of magic. When her family paid their infrequent visits to the duchy's capital of E-Rantel, she would associate with the other noble ladies that were in the city. The informal discourse of luncheons, afternoon viewings and evening events would cover topics ranging from frivolous to fruitful. As a young girl following her mother, she had soaked up everything like a sponge as children did. In recent years, while she still enjoyed the same topics as other young noblewomen, she was increasingly drawn to more practical subjects that would help her manage her family’s demesne.

 

Regardless, discussions on magic were cursory at best. They were either related to divine magic – dealing with the state of the land or the well-being of the population – or a fanciful spice added to titillating tales of adventure spun by minstrels hired for their entertainment. Even then, she was probably more knowledgeable than her peers in this area. In his twilight years, Bohdan had become capable of casting divine magic of the third tier, so she was more familiar as to its use in a fief than most of the other ladies of the regional nobility. If he himself held this legendary caster in such estimation, she could hardly refute him.

 

Bohdan’s continued ruminations jarred her from her thoughts.

 

“What…what if the Undead caster that Milivoj saw riding that aberration was Fluder Paradyne? It is said that he has lived for an unnaturally long time – I do actually recall his name being spoken of even when I was a boy studying in the Theocracy. Perhaps he has given himself over to evil magics to become an Elder Lich? But that would mean the Empire has fallen in league with the enemies of the living: Fluder Paradyne has been both mentor and advisor to generations of Emperors…”

 

Ludmila didn’t know what an Elder Lich was, but the rest did sound genuinely dire. The Baharuth Empire was once a part of Re-Estize, and had broken away a decade before the founding of House Zahradnik. If Fluder Paradyne had been present, manipulating the courts of the Empire ever since the schism, it could be possible that the belligerence of the Empire was the result of his influence and long in the making.

 

Bohdan had served as a trusted member of the Baron's council since the time of her great grandfather – a faithful servant of the gods and of the people that had come to rely upon him. There was no reason to believe he wasn’t acting with the best interests of the barony in mind.

 

“Then what do you propose we do?” Ludmila asked.

 

“If we flee west to the rest of the Kingdom,” Bohdan answered, “the horrors of the Empire will surely run us down along the way. The Kingdom will be doomed, anyways. We must go south, through the upper reaches. We can escape to the Theocracy – by the power of The Six, the Theocracy will be able to stop this abomination.”

 

“This plan is heedless.” Ludmila’s response was immediate. “You propose that we follow the river up its course to the southwest until we can find a place that is shallow enough to ford, after which we would need to turn southeast through the passes until we reach the Slane Theocracy. That is at least two hundred kilometres of savage wilderness between Warden’s Vale and the border of the Theocracy, then another hundred kilometres to its nearest city.”

 

Rather than the gently rolling fields of the pastoral heartlands that someone unfamiliar with the region might be given to imagine, the southern reaches of Re-Estize were bordered by an ancient mountain range which was populated by a plethora of inhuman tribes. The regions further to the south were a windswept plateau dominated by shrublands, rocky moors and sparse woodland, but the area nearest to their territory and much of the Kingdom's southern border was an imposing barrier range dense with primal forest. Though nowhere near the height of the towering, snow-capped Azerlisia range to the north, the southern ranges were still high enough that their heights lay bare, their rugged peaks standing in stark relief to the deep valleys cutting through them.

 

“Leading the entire village through that will take two weeks at the least,” Ludmila told him, “maybe a month if we encounter significant delays. Being caught and exposed to a late winter storm in the passes would surely be catastrophic. There are also the Demihuman tribes and monsters that are in the area. They might not come after Warden’s Vale very often, but they would not pass up the opportunity to attack a vulnerable caravan of refugees if it suits them. I cannot approve of this reckless course of action.”