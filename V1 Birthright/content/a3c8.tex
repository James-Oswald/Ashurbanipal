\chapter{Ludmila Zahradnik}

As scheduled, the Undead labourers arrived during the night, and she saw them neatly formed up on the flats outside the entrance of the village when she peeked through the curtains of the hall window. Coming down after a quick breakfast, she spotted the figures of Nonna and Jeeves as well. The Undead had assumed much the same formation as they did when they presented themselves for inspection in E-Rantel, and it was because of this she immediately noticed that something was wrong.

 

“There’s one missing,” Ludmila noted as she double checked the formations arrayed before her, “what happened?”

 

“It lost its footing on its way down into the valley,” Nonna answered. “It did not survive. The equipment has been retrieved.”

 

Ludmila frowned at the unexpected news. The makeshift trail up the gulch that connected to the forest road in the south was usually only frequented by Rangers on patrol and not well maintained. It was something she should have warned them about, especially considering that she had just traveled through there a day previous to their dispatch.

 

“They await your orders,” the Elder Lich prompted.

 

Nonna did not even appear to show a sliver of sorrow for the missing Skeleton. It had even gone on to mention that its valuables had been recovered rather than dwell on the loss. Ludmila wasn’t sure what to think about its reaction. Did it simply consider it an expendable servitor not even worth such consideration, or would the Undead servants of the Sorcerous Kingdom treat all of the citizens this way? She wondered what would happen if a Human was injured or killed while working. Would these Undead simply continue about their tasks with little regard to the events around them? The idea that she somehow gave more thought to her missing labourer than one of its fellow Undead was not something that sat well with her, but there was currently nothing to be done about it.

 

“Let’s go, then,” Ludmila said.

 

Rather than taking the main road, which led around the base of the hill, she cut across the slope to save time. It was still early in the morning, with the sun not even shining on the upper slopes of the valley to the west yet, but the trek to the old hamlet that had been built to tend to the fields was over an hour away. Their route rejoined the main road where a fork split off to the west, leading up to the lower slopes of the valley. Near the foot of the slope, there was an old wooden bridge that stretched across the churning floodwaters flowing into the marshy flats to the north.

 

Everything about their journey traced through some of the older parts of the fief. The old wooden bridge led to the old road which passed through the old fields with its old hamlet before winding its way up to the top of the valley where the old suspension bridge had frayed and broken in generations past. It was the original way that people travelled to and from the territory, with the road leading south through the forest and into the neighboring barony. There was not much left of the road, though – the sparse traffic that it still saw from the villagers headed out to forage in the woods was not enough to keep it from fading away and becoming overgrown; eventually vanishing into the forests a couple of kilometres beyond the hamlet.

 

Ludmila tested the bridge as she crossed it, shifting her weight about to see if she could find any weak or rotted sections of planking. The poorly maintained infrastructure of the barony had already claimed one Undead labourer; the last thing she needed was one of the Death Knights carrying the heavy iron ploughs falling into the waters below. She stood to the side after crossing the bridge to watch while the column of Skeletons continued up the traces of the road.

 

The first Death Knight approached the far side of the bridge, carrying the plough which must have weighed half a tonne directly overhead. It stopped, seeming to sense her trepidation over its crossing and carried the farm equipment together with another Death Knight, distributing the weight between them. The ominous creaking of the bridge sent tingles of alarm over her but, in the end, the first plough, then the second made it across without incident.

 

Ludmila jogged back up to the head of the column and they continued up the gentle slope into the terraced farmlands, where the road turned to lead them northwest. As of the previous year, the barony only cultivated a small part of the total farmlands leading up to the old hamlet – producing grain which served to supplement the other sources of food hunted and foraged from the wild. The rest of the thousands of acres had lain fallow since before she was born, and where there were once neatly kept fields on the wide terraces along the lower slopes of the valley, now were wild meadows filled with bushes and tall grass. There were dozens of young groves of trees sprouting everywhere she looked, and the edges of the windbreaks that separated the fields into orderly sections were encroaching into the open spaces as well.

 

The small farming hamlet was in better condition than the overgrown fields, some of its buildings seeing seasonal use by the villagers during the farming seasons. Unlike the homes in the village, the ones in the hamlet were of a more regular style of construction one might see anywhere in the duchy: small cottages with timber frames and panels of wattle and daub which served as walls. A well had been dug and constructed out of stone in the centre of the village, as well as a large communal barn which was used to store equipment and seed. Releasing the lock and latch, she rolled open the barn doors and looked inside from the entrance.

 

Four wooden carts were parked before her in the central space of the barn. She instructed several nearby Undead to roll them out and moved in to inspect the interior. The walls were lined with wooden bins, which were filled with seed grain – oats and barley as far as was written in the village ledger. The upper floor of the barn stored various old farming tools as well as leftover sacks used to carry grain for transport. There was, thankfully, no sign of damage or vermin having made their way in over the winter.

 

She turned to the entrance, calling for Jeeves.

 

“Use this barn to organize and store the equipment,” she instructed him. “We’ll keep the carts outside while the fields are being worked – select four skeletons to help you out with everything. The fields nearest to the hamlet will be prepared first, so we can just use this building directly to exchange broken tools and transport seed from, but as we work our way further outwards, we’ll be using the carts to transport what is needed between locations.”

 

Jeeves bobbed his head in acknowledgement, and four Skeletons immediately joined him as he walked in, carrying his box. The entire column lined up to drop off the various tools they had been carrying into separate piles. Nonna watched all the activity quietly, writing in the book it carried on its person everywhere it went. Ludmila walked back out of the barn, leaving Jeeves to his work. On the road which led through the centre of the hamlet, the skeletons which had dropped off their equipment formed up again behind the Boar-type Undead Beasts and Bone Vultures, which had not carried anything from the city.

 

Seeing the sun beginning to peek above the eastern ridge, she motioned for the Undead – including the Death Knights carrying their ploughs – to follow her to the nearest field. She looked over the blanket of stubble left over from the harvest.

 

Green shoots of grass and clumps of weeds had already started to grow despite the cold weather. Ludmila directed the first plough to be set up: each was supposed to be operated by a pair of Death Knights, with one replacing the team of draft animals which normally pulled it and the other driving from behind. Since the crew for the Knarr which had been provided did not initially know how to sail the vessel, she had been wondering whether these Undead labourers would know what they were doing...apparently not.

 

She watched from the side as the pair of Death Knights tried to puzzle out the operation of the farm equipment. A short while later the other team had set down their own equipment and joined them. They did not audibly communicate with one another, but would often take turns gesturing to one another and performing actions in the air as they ‘discussed’ amongst themselves. It vaguely looked like a group of middle-aged men that had gathered around a problem, debating over how to address it. Ludmila only knew that it should have been hitched to whatever was pulling it, then the blades of the plough would cut furrows into the field and turn the soil.

 

After several vague suggestions from her and a series of actions that seemed far too complicated for simple farming equipment, the team looked ready to go. The two other Death Knights withdrew to stand beside her, watching with their arms crossed as the first team started to move experimentally. The blades of the gang plough bit deeply into the soil, turning it as the pair slowly drove in a straight line along the edge of the field. Seeing results from their effort, they started walking faster. The walk turned into a jog, then into a run until lines of sod were being thrown up into their wake. As they continued their blistering pace, loud clanking sounds could be heard all the way from where Ludmila was keeping pace alongside them from the road.

 

It wasn’t until a rock the size of her head flew up and landed with a thud near her feet that she realized what the noise was.

 

“Woah!” She shouted while waving her arms wildly, “Wait! Stop!”

 

It was too late. With an awful noise somewhere between a clang and a thunk, the plough finally hit something it couldn’t toss aside. The Death Knight driving the plough was lifted over a metre off the ground as it held onto the handlebars before the entire thing dropped back down again. The one pulling the chains hitched to the equipment ended up with its face in the dirt. Ludmila ran down to them as they recovered themselves. While the legendary Undead were unharmed, the same could not be said for the decidedly not-so-legendary farming equipment.

 

One of the chains used to pull the plough had snapped somewhere and, when they turned the equipment on its side so she could examine it, the full extent of the damage was revealed. The leading edge of the gang plough had been thoroughly mangled: the foreshare had broken off and the head of the mainshare bent at a right angle. The mouldboards were dented and scored from all the large rocks they turned aside at the high speed the Death Knights had been running at. The culprit that finally ended their run was a boulder that was buried just under the surface of the field. She could see the chip where the impact had occurred but, besides that, the boulder seemed smugly intact. She turned to one of the Death Knights.

 

“Head back and get replacement parts from Jeeves,” she ordered, “and get the other labourers out here while you’re there. Have the Skeletons bring their spears and a dozen shovels as well.”

 

Ludmila walked around until she found a stone that would fill the palm of her hand. Picking it up, she returned to the road and waited. When the Undead labourers had gathered, she held the stone out in front of her.

 

“Sweep this field,” she instructed. “Bring any stones you find of this size and larger to the side of the road. Any other debris as well. Use your spears to probe the ground in case there are boulders lurking under the soil and mark where they are. Those with shovels – dig up this boulder.”

 

The Skeletons formed two wide lines, one following the other as they slowly picked their way across a stretch of the field while the Undead Boars rooted ahead of the Skeletons, overturning suspicious sections of soil that they came across. The rocks uncovered by the Undead were left exposed on the ground, where a Bone Vulture would fly down and snatch it up to be delivered to the nearby roadside. The labourers equipped with shovels were busily digging around the boulder while the Death Knights were once again standing around the gang plough puzzling out how to replace the broken pieces. The base of the foreshare was easily detached, but they had to unbend the ploughshare to work it off the mouldboard. One turned around with each of the pieces in one hand, holding them up as if asking Ludmila what to do with them.

 

“Pass them to a Bone Vulture,” she said.

 

Ludmila waved one of the flying Undead down to pick up the damaged pieces of equipment. She walked over to pick up the broken-off part of the foreshare which had been set aside near where the Skeletons were trying to dig up the boulder and tossed it over to the Bone Vulture waiting nearby.

 

“Take those pieces to Jeeves – maybe he can repair them somehow.”

 

With two pieces of the broken blade each talon and one in its beak, the Bone Vulture flew off in the direction of the barn. Though Ludmila was somewhat doubtful over her own statement – Jeeves had needed tools to mend the torn cloth and the hamlet did not have a smithy – it was at least worth trying. Turning back to the work on the boulder, she saw that the Skeletons had dug a half-metre deep trench around it but, as they went deeper, it only seemed to grow wider.

 

About a metre down and ten minutes later, a clunking sound could be heard coming out of the hole as a few of the Skeletons were met with rocky resistance. She instructed the Skeletons to give the rock an experimental push from one side and it appeared to be solidly attached to the stone below – apparently what she had thought was a boulder was actually a protrusion of bedrock. Ludmila put her hands on her hips and let out a sigh, wondering what to do.

 

She turned to the sound of the Death Knights approaching; it seemed that between them they had finished replacing the broken parts of the plough and the first team was ready to go again. Looking at the others walking up to see what was going on with the boulder, a reckless thought came to her mind. Motioning for the Skeletons to leave the trench, she addressed the group of Death Knights.

 

“Are you able to break off the top of that boulder and carry it over to the road?”

 

It sounded ridiculous even as she said it and stepped out of the way, but the Death Knights did not hesitate at all.

 

The first Death Knight stepped forward and drove the sole of a dark sabaton into the side of the offending boulder. The first strike had no discernible effect, but that was no deterrent. It kicked again and again, with chips of stone coming off the surface until finally a long crack appeared. Several more kicks were delivered for good measure, then the Death Knights dropped into the trench to detach it from the base to carry it off the field. It had split off diagonally, so Ludmila ordered the jagged end to be broken off and carried away as well. The two broken stones together must have weighed as much as the plough if not more. Filling the hole took several rounds of delivering soil, packing the dirt and adding more soil before it seemed solid enough to stay roughly level with the rest of the field.

 

Nearly an hour had passed since the incident, and Ludmila felt that her initial expectations of how well the planting season would go were already off the mark. Looking down the field in the direction of the skeletal labourers, she spotted where several shields had been propped up on the soil to mark where their probing had apparently struck more stones underground. Dividing the shovel-wielding Skeletons into two teams, she had a pair of Death Knights per team follow them out to repeat the process of removing whatever obstacles lay beneath the soil, leaving the ploughs where they were until all the hazards were cleared from the field.

 

“We’ve just started and there are already major delays,” Ludmila sighed.

 

She scanned the field one last time.

 

“Let’s get back to the hamlet,” she said to the remaining Death Knight.