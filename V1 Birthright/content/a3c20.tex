\chapter{Ludmila Zahradnik}

Leaving Moren to his work, Ludmila left the hamlet with her questions for the man having led to dozens of other questions. She made her way northwest along the old road, at the forefront of a group of Undead, pondering new perspectives and information; trying to apply it in ways that might work with her current knowledge and rethinking how she would go about laying the foundations of her demesne.

 

After roughly half an hour, she came to the thickly forested slopes of the valley beyond the fields, where the road disappeared into the undergrowth. It had finally come time to test out some of her own ideas. After Lady Shalltear's presentation of the various types of Undead available for lease, Ludmila had requested a group of Undead Beasts in the form of boars with the idea that they, like the wild boars that lived out in the wilderness, could sniff out roots, mushrooms and the many other edible fruits and plants from the forest.

 

As she picked her way through the woods, she came across a cluster of broad white fungus that was favoured in the village’s ubiquitous stew growing out of a tree trunk. After a few minutes puzzling over how exactly she could issue the instructions, she sent the Undead boars out into the forest. Ludmila watched with some alarm as they immediately sprung into action, crashing noisily through the woods.

 

She caught up to the first Undead Beast, which had stopped several hundred metres away: it had found some mushrooms, but besides looking vaguely similar to the ones she sent it out to look for, they were deadly if eaten. Out of the first set of Undead she had sent out, only one had found what she had instructed them to find.

 

Ludmila tried improving on her commands with the same group, then sent them out again. This time, none of them were able to locate what they had been sent to find. There were very clearly other patches of the same mushrooms around – she occasionally stopped to put them into a basket as she followed the Undead beasts’ rampage – but they were somehow unable to recognize them, despite their director clearly being able to. What they did do, however, was create an extraordinary amount of damage in their search: bowling over small trees, uprooting medicinal plants and destroying flowering bushes that would have produced summer berries. While they appeared to be like boars, the only quality that these Undead Beasts seemed to convey was the destructive image associated with one.

 

The first of her ideas not outlined in the materials provided by the administration appeared to be a catastrophic failure; she wondered if it had already been attempted and left out due to the same findings. It would have been nice for them to leave a note of some sort if this was the case.

 

She descended down the valley towards the flooded marshes, continuing to forage fresh shoots and other spring edibles along the way. Three of the wicker baskets carried by the Skeletons in the group were filled by the time she reached the soggy edges of the floodplain. Ludmila immediately spotted a cluster of broad, arrow-shaped leaves standing out of the water and waded over with a Bone Vulture after pulling off her boots and tucking the hem of her gambeson up into her belt. She couldn’t find a tuber near the base of the plants by feeling around with her toes, so she had to reach into the knee-deep water, feeling around until she came up with an arrowhead tuber which fit in the palm of her hand.

 

The Arrowhead plants grew all over the valley, and their starchy tubers had become the staple of the village in her generation as the population of the barony dwindled to the point that the farmlands could no longer be properly cultivated. They were similar to the Potatoes that were grown in the Empire, albeit much smaller and, after she had been served the fried strips by Yuri Alpha, she had wanted to see if these could be prepared in a similar manner.

 

With the tuber in hand she issued directions to the Bone Vulture, which was completely submerged aside from its head and long neck poking out of the water. After sending it out into the marsh, she continued harvesting her own as she awaited the results.

 

There was not much in the way of recommendations provided by the administration’s materials for marshy terrain, so this was another idea that she had puzzled out for herself. It could be difficult for Humans to navigate the deeper parts of the marsh, but a Bone Vulture could get around easily by flying…or so she thought. Looking out across the water, she saw the head of her test subject still poking out above the surface as it slowly waded through the marsh. It wasn’t until it found a sandbar which was shallow enough for it to raise most of its body above water that it shook itself dry and rose into the air. Well, at least as the water levels receded, their mobility would improve.

 

She directed more Bone Vultures, each sent out to harvest a different type of plant and, as they either succeeded or failed in their tasks, Ludmila developed a loose understanding as to what would work and what would not. For the Bone Vultures, if the target of their efforts had very clear and distinguishable characteristics – such as the Arrowhead plants with their large, arrow-shaped leaves standing well above the surface of the water – success was nearly guaranteed. Bulrushes and other reed-like plants they could vaguely recognize by appearance and, since the different plants were all useful for one thing or another, Ludmila also considered those a marginal success.

 

The more a plant resembled the growth around it, however, the more their success rate approached that of the Undead Beasts in the forest. Sometimes she would receive a bundle of watercress, sometimes she would get a bundle of weeds or a bunch of moss. In several trials, a Bone Vulture would make such a gross misidentification that it returned with what one of the others had been sent to collect instead.

 

After a couple of hours gauging the results, she decided that they were worth using to harvest only a handful out of the many useful plants that grew in the marsh. If the prices for produce fell too far, she would have to figure out some other worthwhile use for them. Ludmila sent them to collect Arrowhead tubers for the time being in order to increase the dwindling stocks of food in the village. Each Bone Vulture gripped the strap of a shallow basket which could be floated on the water's surface as they landed to find food. When the Human population of her demesne increased, she would have to raise new Rangers to help with foraging activities in both the marsh and the forest.

 

With one disastrous failure and a sort-of-success, she scaled the slope back up to the farming hamlet with the remaining Undead. She crossed the field Moren was working on first, and the Farmer nodded as she approached.

 

“How are things going?” Ludmila asked.

 

“It’s an interesting challenge,” he said, leaning on a stick he had picked up from somewhere. “Since they will continue working with their current orders until they complete them, it makes it possible to control many sets of Skeletons as long as the time it takes for them to work is long enough for you to set new instructions with the others. That number shrinks dramatically if I try to manage more precise tasks, however. Four seems to be the number – at least for me – if the goal is to maintain the same quality of farm work as a Human with each one.”

 

“Make sure you let Nonna know,” Ludmila told him. “This is just the sort of information she’s looking for.”

 

The ratio Moren presented meant that roughly 20 farmers would be needed to direct the 80 skeletons working the fields for one hamlet. Accounting for shortfalls in labour due to illness or injury, she thought 25 would be more reasonable. On the other hand, once she was able to bring in a new priest, the problem should be next to nonexistent. While it was still dependent on the actual results, her estimation that a single Farmer could easily provide for their entire household meant that the hamlet would flourish with the abundance, freeing up labour for other industries. It would quickly become a village and a base from which to extend the fields further along the valley, after which she could found a new hamlet that would develop in a similar manner to continue expansion.

 

With plans for five more farming villages with a similar allotment of land Ludmila became hopeful that, with the solid foundation established by the chain of future settlements, she would be able to transform the harbour village into a town filled with the specialized labour and industries that would be needed to stay ahead of the changes that the Sorcerous Kingdom’s Undead labour would bring to the regional economy. She would certainly have food aplenty – enough grain to feed the entire population of the city of E-Rantel – but more specialized development would take time.

 

All of the timber harvested by clearing land for the fields could be exported to meet the future demand that Lady Shalltear had spoken of to Gareth Boyce, and the proceeds would be used to help to push the development of the Warden’s Vale. Perhaps she should see to building the mill soon in order to improve the productivity of her demesne. After that, she would need to hire prospectors and other specialists to explore her lands for other resources to develop.

 

The vision of unprecedented growth filling her head put Ludmila into a good mood, and she hummed happily as she made her way back to the hamlet. As far as her plans for the day went, she would be returning to the village with the broken and damaged equipment while checking on how the woodcutters were doing along the way. Directing a few Skeletons to add the additional damaged tools that had appeared while she was away to the cart with the rest, she then instructed one of the two Death Knights in her escort to pull the vehicle along. As it leaned over to grasp the tongue of the cart, Ludmila had a thought.

 

“Actually – hold on a bit, I have a different idea.”

 

She had struggled to think of what to do with the Undead Beasts after finding them unsuitable for what she had intended them for. Churning up the fields for clearing would only be a short term use, and she wanted to find something more permanent. Ludmila motioned for two of the boars to come forward and she secured them to the yoke attached to the tongue of the cart. Sitting in the seat, she ordered them to circle around the well in the centre of the hamlet.

 

Finding that they were able to accomplish the task without issues, she went to attach two boars to each to the three remaining carts – there were still free Undead Beasts left, but there would be more carts built in time to meet the demands for transport. If they worked for the long term, using Undead Beasts would be much more cost-effective than Soul Eaters for pulling these smaller vehicles.

 

The small train of carts travelled back eastwards along the road towards the harbour village, stopping periodically for the Death Knights to load a tree stump or other pieces of debris that had been too heavy for the Bone Vultures to deliver. The procession stopped when they neared the end of the fields, and Ludmila hopped off to see how the woodcutters were doing. They came in from their work to gather around her and report their findings.

 

“We lost four skeletons in all so far, Lady Zahradnik,” the man submitting their report seemed a bit apologetic and more than a little miffed. “Most villagers learn how to play it safe around Skeletons and Zombies that pop up now and then, but it turns out they’re just as mindless when they’re working for you. Each of our groups lost one felling our first tree – they have no sense of self-preservation, so they just stood still to get crushed. Then another one was lost when we forgot to tell it to get out of the way some time later. Four makes it too crowded to work; it’s hard to look out for the other woodcutters and these Skeletons at the same time.”

 

“How many do you think you can work with, then?” Ludmila memorized his report to relay to Nonna.

 

“Two to a team, probably. Three if the third is just moving things to and from the site.” He scratched his head through his cap, “Don’t get me wrong – they’re good for what we’re using them for, but if there’s too many running around it will just cause more accidents. It’s dangerous work, hacking down trees is the least of it; we have to prepare and plan ahead to make sure everything goes smoothly.”

 

“I understand,” Ludmila nodded. “Once the fields have been restored, I plan on clearing stretches of the slope in sections about the same size as the ones here. How many more woodcutters do you think I can bring in before the numbers become a hazard?”

 

“How wide will the fields be, my lady?” He asked.

 

“A five kilometre width between the edge of the marsh and the valley slope – about the width you see at the hamlet over there. The fields will follow the old road up to where it leaves the valley.”

 

“If we’re clearing through in a line, one team every 100 metres should be safe,” he said after some thought. “Well, it also depends how much work you’ll have for us after all this is done. I’d feel bad for the ones that are sent away if there’s no work to be had after.”

 

“I’ll keep that in mind,” Ludmila said. “So a team is two Skeletons and…”

 

“Two men, my lady.”

 

“Will there be any problems working out of the hamlet?”

 

The man exchanged glances with the other woodcutters, turning back to her with a shrug.

 

“If we can take our families with us,” he said. “I don’t see any reason not to – especially since it’ll be saving us a long walk.”

 

Most of the buildings in the old hamlet were run down. She would need to find tradesmen capable of fixing them up and building new ones if she planned on expanding it into a village. At this rate, the population of the farming settlement would become larger than the harbour village, as it would take more time to attract the advanced industries that she wanted to base out of the future town.

 

“Were there any other problems?” Ludmila asked.

 

“Not really, no,” he answered. “With the Undead labour, we get through everything a lot faster and there’s not much of a need for breaks. We haven’t felled any trees worth shipping out as building material yet, but how do you want them moved?”

 

“Before,” Ludmila said, “our villagers had to divide into lighter sections on site before transporting the timber to the village, but with the Death Knights here, we might be able to keep them whole. Once the field work is done, we can add the plough teams as well. I’m not sure exactly how much they can handle, but it should be quite a lot. You’ll want to put them to the test before deciding what’s safe to transport.”

 

“That sounds reasonable, my lady,” the man said. “If we get any trees that are too heavy, we’ll cut them lengthwise into sections that they can manage if they team up, unless you want to keep them bigger than what that ship of yours can carry.”

 

“It should be fine, for now,” Ludmila said. “I believe we’ll be storing the timber on the flats near the harbour. Are there any preparations that you need to make for the stockpile before deliveries start arriving?”

 

“Uh…yes. We can get that done quickly when we get back this evening.”

 

“Also, there will be a mill at some point – I would like to start construction as soon as possible if I’m to bring in more woodcutters.”

 

The man nodded at her words.

 

“A mill will help you pack more onto that ship of yours and bring in more coin since you’ll be doing a lot of the work sizing the lumber here. The rest I have no idea about – you’ll have to find a carpenter or something to help with construction. We’ll carry on here, my lady. If you can spare them, we could use a couple more Skeletons on standby just in case there are more accidents.”

 

“Of course,” Ludmila said, “I’ll leave a few of the available ones here on the road for when you need them.”

 

“Thank you, Lady Zahradnik,” the man said, and the group bowed in an uncoordinated way, “for everything. We’ll be getting back to work, then.”