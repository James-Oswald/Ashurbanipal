\chapter{Ludmila Zahradnik}

With Vilette Jezne’s words, the casual atmosphere of the room turned weighty. There was a loose rustling over the air as chatter dispersed and the various nobles around the lounge settled into their seats, readying themselves for the business at hand. When the motion around the room ceased and all heads turned towards her, she continued.

 

“I trust that, by now, everyone is familiar with the terms of E-Rantel’s annexation?”

 

As her gaze slowly swept over the nobles in attendance, she was met with many nods and other affirmative gestures. Ludmila had not seen these terms, but remained silent as she did not wish to interrupt the proceedings. Perhaps the following discussion would clue her in on what they were.

 

“I share in the opinion that they are generous indeed,” Vilette Jezne said as she finished looking over the assembly, “His Majesty has decided that the laws of Re-Estize are to be retained. This includes the Crown Laws, as well as the ducal laws set forth by Lord Rettenmeier on behalf of our former liege where they are applicable. As such, we have also retained our titles, rights and obligations as nobles of this realm. The laws of Re-Estize protect nobles from unjust actions by other parties, including His Majesty himself, and so too are we similarly protected by these laws that have been adopted by the Sorcerous Kingdom.”

 

Ludmila nodded to herself, quietly absorbing the content of the noble matron’s address. That the legal framework of Re-Estize had been kept intact was in line with what Momon suggested when they had first met. It meant that rather than uproot what had been established in favour of his own methods, the Sorcerer King saw fit to build upon the existing legal and administrative systems.

 

“His Majesty’s representative, Lady Albedo, has also maintained that the appointment of executive offices remains the sole right of the Crown. The legislative branches of the government shall similarly remain unchanged – the royal, regional and local courts will function in a manner similar to how we’ve conducted our affairs in the past.”

 

If the few remaining nobles standing in the clubhouse lounge represented all that remained of the aristocracy, Ludmila felt that the old arrangement would be somewhat superfluous. The upper and lower courts of Re-Estize were attended by a noble’s vassals, as well as any foreign petitioners, and at each level functioned in a similar manner. The Royal Court was attended to by the King’s vassals, who represented the interests of their own courts. The courts of the King’s vassals, in turn, would be attended to by their own vassals, who held their own courts. In this way, only the issues which impacted the realm as a whole – such as war and large scale disasters – would filter up to the royal court. Disputes between farmers, matters of maintenance and local justice would not make it past the nobles who ruled directly over the land, or their appointed proxies.

 

Since E-Rantel was a duchy belonging to King Ramposa III, and was appointed a Royal Provost in the form of Lord Rettenmeier, the King had been their representative in the Royal Court of Re-Estize. Now, however, the regional court of E-Rantel was the Royal Court. Ludmila wondered if that meant that they simply planned to just keep the Royal Court as a council of appointed advisors and ministers. Either that, or the Sorcerous Kingdom had further plans for expansion and the creation of new titles…

 

“More relevant to recent events,” Vilette Jezne’s voice carried on, “as the Sorcerous Kingdom has a powerful standing military, the need for levies should be next to nonexistent. As with everything else, the laws surrounding them have remained unchanged; I believe this to have been left in place for only the most dire of emergencies. The highest court of justice will remain as it is, reporting to the Crown. Details on how local laws are enacted and justice is enforced remains under deliberation for the time being: if nothing changes, it should remain the jurisdiction of the local administration.”

 

At the last, voices from around the room began to murmur. After many consecutive years of having their manpower tied up for months due to the annual confrontation with the Empire, it was a welcome piece of news that they would no longer be expected to raise armies to fight for the King short of an emergency. The statement on local authority raised all sorts of questions, however.

 

“Does that mean we can no longer create laws that address the issues unique to our fiefs?” A noblewoman’s voice arose from Ludmila’s left, “If we are still allowed to do so, but lose our right to enforce those laws, are we to depend on His Majesty to act on our behalf?”

 

The low murmur rose as the nobles around the room discussed the points presented. It was indeed something to consider. Different fiefs had different industries and regulations associated with those industries, so it was concerning that the disparate territories might not be able to create local laws to address their specific needs. It would also be extremely odd for a minor noble to wait for the sovereign to enforce laws in their own small territories when it could just as easily be accomplished with the retainers or militia on hand.

 

Vilette Jezne cleared her throat, and the lounge settled down once again.

 

“I believe that these details are a matter to be brought up in the future once the regular proceedings of the courts have resumed. In similar vein to our concerns over local laws, the…Guardian Overseer has also expressed that amendments to the Crown Laws are not beyond consideration, and that new laws will be created to meet the country’s needs as the nature of our Sorcerous Kingdom is unlike that of the Human society which we have been raised in.

 

Once again, I must remind you that these terms are already beyond our best expectations, and that we should count ourselves fortunate. If this duchy had been annexed by the Empire, they would have torn apart whatever did not suit them to establish their own order, regardless of preexisting laws and vassal arrangements. Do not mistake this generosity for tolerance, however: the fate of Count Fassett should be proof enough that the Royal Court will not suffer fools.”

 

As the aged noblewoman continued to speak, Ludmila felt increasingly out of touch. What happened to Count Fassett? What in the world was a Guardian Overseer? It seemed to be some sort of official position within the Royal Court, an unfamiliar title that begged description.

 

Count Fassett ruled over the lands on the western border of the duchy, straddling the main highway leading to the rest of the Kingdom. This made him one of the most prominent nobles in the territory, and he took advantage of his station in many ways. Her father had mentioned that he was a fairly aggressive man that leveraged the incomes and advantageous position of his territory to influence local politics, but the Royal Provost, Lord Mayor Panasolei Greuze Day Rettenmeier, had always kept his political aggression well in hand.

 

At the thought of the rotund man, she recalled that he and the administrative staff of the city had fled. Even if the nobles could get their territories up and running again, how was the central administration to function without clerks and accountants? Finding educated people that were qualified for these positions was not something that could be done in a matter of weeks. Unfortunately, Vilette Jezne was not so accommodating to her ignorance as to answer her unspoken question, moving directly onto her next order of business.

 

“In the meantime,” Count Jezne told them, “I believe that we should answer the Sorcerer King’s trust in our capabilities by restoring productivity to the territories. Though…I suppose our first order of business should be to sort out the damage done to our houses and reorder the ranks.”

 

She looked around and let out a derisive noise.

 

“The lot of you look like some peasants at market about to start fighting over the day’s scraps. Nothing will get done until you fix this.”

 

Ludmila, already feeling a bit lost as the noble matron jumped between topics already familiar to the rest, stared blankly until the last piece fell into place with the aged woman’s barb. She looked around the lobby again and finally understood why she initially thought the arrangement of attendees seemed off. The few noblemen in attendance were mostly too young. Everyone else was either a daughter of a landed noble or the wife or mother of one.

 

Her mind went back to Milivoj's testimony. Though she had acted on the news and evacuated her people, Ludmila held onto the hope that she would eventually find her own family in the city. In hindsight, she should have known that the horrific events detailed in the account of the traumatized villager would have exacted a toll on everyone that had been at Katze.

 

Vilette Jezne was no longer simply the Dowager Countess Jezne: her entire family had probably perished and the title had fallen back to her – Countess Jezne would now need to find some distant relation or adopt a suitable successor, as she was well past childbearing age. Ludmila looked at each of the attendees: faces, names, houses; reconstructing the web of the new political reality of the realm.

 

Most of the women present she had seen before on previous visits to the capital; either through interactions between houses or at the various social events and venues that had served as a pleasant backdrop to the dealings of the ladies of the nobility. The young girls and women now in attendance were the eldest unmarried daughters of each house. The older women were the wives or mothers of noble houses with no qualified successors.

 

As for Countess Jezne’s scornful remark about fighting for scraps, there was a tremendous problem. The number of eligible consorts in the realm had become disproportionately small compared to the single, childless noblewomen that had suddenly inherited all at once. She now understood the nature of the strange groups of nobles arranged about the lobby: each had one or two boys in it, pressed in on all sides by women who were now in competition over a scarce resource.

 

She looked to the lordling that she had brought in with her from the front desk: he was still timidly hovering around her. Several ladies nearby were sizing him up like a tender morsel, but none would dare directly challenge a Frontier Noble. Looking around, she saw that a Countess had even gone so far as to claim the twin sons of one of her own baronies.

 

It’s not like I marked this kid for myself or anything...

 

He wasn’t too far off from her own age, actually. She stole a glance at the boy again: he was probably three or four years younger.

 

The sigil of one of the inner houses was embroidered onto his oversized coat. The inner nobles were the members of the aristocracy that held titles close enough to E-Rantel to base their primary residence in the city or close to it. They tended to have the most developed fiefs and owned most of the industries around the city, as well as have the greatest access to trade and various connections…

 

The boy put on something of a hunted look – as if he had suddenly realized his current place of refuge might have not been as safe as he had initially thought. Ludmila was too busy calculating to notice, though, idly tapping her chin with her index finger.

 

“Baroness Zahradnik.”

 

The cogs in her mind froze at the sound of a familiar voice behind her. She turned to see the tall, beautiful maid that had led the weeping Baroness Corelyn to a room waiting a respectful distance away.

 

“Yes?”

 

“Your presence is required in the council chamber,” the maid told her, “by order of the Royal Court.”

 

The summons had come sooner than expected, but in an entirely unexpected fashion. Using a maid to deliver a royal summons was entirely unheard of by any nation in the region. Ludmila glanced out of the corners of her eyes to gauge the reactions of the other nobles, but they only stood by quietly: watching her in turn.

 

Ludmila had not yet moved into a residence, nor had she changed from her travelling clothes. However, she could not delay the audience. If this maid was delivering the will of the Royal Court, then she also represented the authority of the Sorcerer King. This being the case, there was only one answer. Quickly arranging her common appearance as best as she could, she lowered herself in a deep curtsey before the maid.

 

“I am at His Majesty’s service.”

 

The maid smiled gently and turned to walk towards the exit of the clubhouse. Ludmila followed her out of the clubhouse, taking one last look back at the lounge: three other noblewomen had already converged on the boy that she had left to their predations, and the gathering of nobles had gone on to discuss other matters. With the summons taking precedence over the nobles’ session, she hoped that she wouldn’t fall even further behind.