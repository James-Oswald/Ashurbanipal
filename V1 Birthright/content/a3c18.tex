\chapter{Ludmila Zahradnik}

Nonna’s haunting words followed Ludmila into bed, causing her to rest fitfully through the night. They were still on her mind when she stirred the next morning. Her attaché’s uncharacteristic outpouring of words had made her ponder all of the experiences she had with the servants of the Sorcerer King so far. Even the words of Lady Shalltear, who was both zealously dedicated and reverent in her service to her liege, seemed to express the same tone from time to time. When she reviewed her memories with the context provided by the Elder Lich, Ludmila could almost sense the shadows of that same creeping doom from some of their discussions. A desperate prayer composed of unflinching service and unbending loyalty, in hopes that an unfathomable fate would be kept at bay.

 

No one desired to be cast aside. No one enjoyed being unappreciated. But, in Ludmila’s mind, as long as one’s life lay ahead of them there would always be new opportunities and purpose to grasp in the future. Even as a noble who came from a house which had always faithfully served their liege, she felt that she could never fully understand the full depth of what Nonna had described. The dark cloud of her thoughts followed her even as she prepared for the day and Aemilia hovered around her nervously, sensing her mistress’ troubled mood.

 

“Is something the matter, my lady?” She asked worriedly as she helped tie up Ludmila’s hair.

 

Looking at Aemilia in the reflection of her small mirror, Ludmila asked a question.

 

“Do servants live for their masters?”

 

“Well of course they do, my lady,” her maid replied matter-of-factly.

 

Aemilia finished with Ludmila’s hair and reached for the mantle draped over a stool nearby. Her hands slowed as she inspected the treated cloth for blemishes and debris.

 

“To an extent…or perhaps it would be better to be described in degrees? It depends on what sort of servant they are, and the relationship that they have with their master.” Her hands started to move again as she helped to fasten the heavy mantle around her mistress’ shoulders, “Even as a lady’s maid, I suppose that one day I will want to have a family of my own, and my loyalties would be divided after I do. Not in any sort of improper way, of course.”

 

That sounded about right. A very normal response for a Human servant – far removed from the unfathomable doom described by the Elder Lich.

 

“I guess there really are some irreconcilable differences between us and His Majesty’s servants,” she murmured.

 

“My lady?”

 

“Make sure to give your Undead assistants lots of work to do,” she replied. “They’ll probably be happier that way.”

 

Ludmila rose from her seat with a slight smile at her maid’s befuddled expression and headed out the door. It was still early, and the morning mist lay thick across the floor of the valley. The villagers appeared to still be getting ready for their day, so she decided to wander around for a bit. The ring of hammer on anvil drew her to the western edge of the village. Ostrik was already at work – or at least it looked like he was working. His forge was lit and there was a dark bar of metal being heated within.

 

“Mornin’ Lady Zahradnik,” he greeted her as she approached. “You expecting a fight?”

 

Ludmila was wearing the gambeson she had purchased from the city, the rest of her armour and weapons were still in the manor, but she planned on picking them up before leaving.

 

“Good morning, Mr. Kovalev," she replied. “I’ll be spending some time in the woods today: you never know what might pop up – even inside of our borders. Did you still need to speak with any of them before we leave?”

 

“Nah, spoke to ‘em yesterday. They know what I need.”

 

“What do you need?” Ludmila was curious.

 

“Once I’m done with this new file here, I’ll be almost out of fuel,” he explained. “We’ll be building mounds to make charcoal with the wood that’s harvested.”

 

“Will any kind of wood do?”

 

“Yep,” Ostrik nodded, “there’s no need to be picky about it for now – I just need to keep things going here. Anything that’s worth more as lumber or whatever you’ll probably be better selling off as is. Once I’m done here, I’ll start looking around for ore while the woodcutters do their thing.”

 

“You mentioned that yesterday,” Ludmila said. “Just where exactly do you plan on looking?”

 

The smith extended his hand northwards, pointing out towards the flooded plain.

 

“That’s a marsh,” she stated flatly.

 

“Yes, my lady,” he agreed with her description. “If you haven’t had a smith here since ever, there’s probably enough iron ore out there to help build a small city. Or equip an army. Well, if you’ve never seen it before you can come back later and I’ll probably have collected quite a bit by then.”

 

“Well, now you really have me curious. Would you like to test out some of the Undead labourers while you’re at it? I can send a handful your way.”

 

“Seeing that I wasn’t eaten in my sleep,” Ostrik said. “I suppose I could give it a shot. How does it work?”

 

“You tell them what to do, and they’ll try to do it.”

 

“That’s it?” Ostrik furrowed his brow.

 

“A better way to put it,” Ludmila said, “is that they understand what you want them to do when you order them around rather than follow your words exactly. If you do something like ask them to learn by watching you, or tell them to do things that you have no idea about yourself, it won’t work. This means that your own expertise is crucial when directing them. Just don’t expect to be putting your feet up as they forge equipment on their own – the Skeletons are only capable of straightforward tasks.”

 

“What happens if I order them to do something too difficult?”

 

“They just do not do anything,” Ludmila said. “Miss Luzi was able to get them to do all sorts of menial chores; you should be able to figure out what they can do when it comes to tasks revolving around your own work.”

 

“Huh,” he grunted. “Well send ‘em on over then. I’ll see what I can do with ‘em, my lady.”

 

The smith returned to his work, and Ludmila turned back in the direction from which she had come. A smile appeared on her face as the village girls came down the hill with Skeletons in tow to fetch more water. A few of their brothers were doggedly following them, begging for a turn to command their new Undead helpers. It was truly a novel sight – one that she would have never dreamed of seeing just a couple of weeks ago. She followed behind the raucous group of children until she reached the flats where the remaining Undead still waited from the previous evening. She instructed four Skeletons to head over to help the smith, then sent half of the Bone Vultures to deliver the wood which was probably piling up from clearing the fields to a spot outside the village.

 

Recalling Ostrik’s words, Ludmila walked over to the edge of the marshes nearby. Looking down, she could see the various plants that were harvested by the village in the past. She could identify the different types of terrain and vegetation that marked different depths of water, as well as the paths that would safely lead through the flooded lands. She definitely did not see anything that she thought resembled iron ore. Several geese came over hopefully, but were disappointed when she turned away without feeding them anything. In doing so, her eyes fell upon the harbour to the east, where the Knarr was moored with its crew.

 

Nonna’s words had made her a bit paranoid about making sure all of her Undead had some work to do, lest they spiral into some sort of suicidal depression. Perhaps she was thinking too much about it, but she still wondered if there was anything useful for the crew to do when the ship was not transporting anything. She looked down from the pier and the crew looked back up at her. After seeing so many Skeletons, she felt that the ones in the ship were noticeably stronger than the regular labourers; their equipment was different from the simple round shields and spears of the others as well. Instead, they had round metal shields and long scimitars as sidearms.

 

Looking over at the captain, she came upon an idea, heading back up into the village again. After ten minutes, she returned with a fishing net and a small basket. Since it was armed with many throwing weapons, she thought it should be able to cast a net as well. The captain gave her a look as she handed the net over.

 

“Take the ship out on the river and use this net to catch fish,” Ludmila instructed. “You may use the whole length of the vale, so take the opportunity to practice maneuvering the ship as well. Keep anything edible around the length of my forearm. When the basket is full, return to the harbour and drop it off in the warehouse.”

 

The crew cast off immediately once she stepped back onto the pier – at least it was something, and the crew did not appear to balk at the task. Hopefully, the fish she had brought to mind when issuing the instructions would be what ended up in the basket when she returned. Back at the village, Ludmila found the woodcutters standing about with their sons. Moren was standing with his things piled about him, looking down the way towards her.

 

“Ready for work, Lady Zahradnik,” the same man that had spoken for the group before announced as she walked up to them, “provided that there are tools. We checked the warehouse and there wasn’t much.”

 

“Everything is out in the hamlet,” she explained. “You will be starting just across the bridge and working your way there though. Your equipment will be delivered to you. Mr. Kovalev said that he spoke to you about a few things that he needs?”

 

“Yes, my lady,” he nodded, “we’ll be marking the cheap timber for charcoal. The rest will be sorted out according to value.”

 

“That sounds good. I will be stopping by the manor for a bit. Head down and pick up four Skeletons each while you wait.”

 

She walked past them without waiting for a reply, assuming that the boys from before would lead the charge down to pick up their new underlings. It didn’t take long before she heard their excited shouts coming from below, oblivious to their fathers’ misgivings. Did this count as using one’s children against them?

 

Aemilia was speaking to the village women near the manor, and they all turned to greet her when she tried to walk around them.

 

“My lady,” Aemilia said, “I was just going over a few things with the women. Will you be headed out to the fields now?”

 

“Yes, I will be out getting the men started out there,” Ludmila stopped to speak to them, “they will be helping to clear the fields before we actually get to the forest beyond the hamlet, so they won’t need to use it as a base to work out of until then.”

 

“Good,” one of the women let out a derisive noise, “they’ve been lazing about, surviving on handouts for weeks: it’s about time they did something. Speaking of which, what you have here is much better than the barley we’ve been having recently. You’re gonna spoil us rotten with the Mannagrass alone. Miss Luzi says that all the geese here are yours as well: does that mean we’ll have plenty of fresh eggs and poultry?”

 

“The eggs you’ll have to personally head out for,” Ludmila told the woman, “until we have people specifically managing the flocks. We also occasionally do have poultry to keep the numbers down. Warden’s Vale is surrounded by the bounty of nature so, while I cannot say we can match all of the amenities that a city provides, the food is much better here. You should still be appreciative of the new administration’s efforts to keep the city fed, however. Lady Shalltear has been working every day, delivering supplies to sustain the people of E-Rantel – she is my liege as well.”

 

“O-of course, my lady,” the woman stammered. “I meant no offence.”

 

“As long as you understand that His Majesty’s generosity is not to be spurned,” Ludmila said to the group. “Our lives will all be transformed for the better, in due time.”

 

She didn’t want to leave the women on what she felt was a sour note, so she turned to Aemilia.

 

“The Undead should have freed up quite a few spare hours,” she glanced over the assembled tenants. “Is there anything else that the village needs?”

 

“Yes, my lady,” Aemilia replied. “The homes have a few basic furnishings, so we would like to start weaving mats and coverings and such to make things more comfortable. The warehouse didn’t have anything that I could see – was there something the village used from around here?”

 

“There are stands of rushes all over the place that you may use,” Ludmila replied.

 

“The villagers used rushes, you say?” Aemilia looked over at the vast carpet of plants in the marsh.

 

“Yes,” Ludmila replied. “I’ve only watched them at work, but they were able to draw fibres from the plants. After processing, they were able to make different sorts of cloth for pretty much everything. You should have seen a few examples of it around the village while you were out cleaning the other day.”

 

Aemilia gave her response some thought before answering.

 

“I’ll have some samples brought in, my lady,” she said. “I should be able to figure it out – I’m a weaver’s daughter, after all.”

 

“You can start by trimming down all the plants along the shores surrounding the hill,” Ludmila instructed them. “About knee-length is fine if you don’t plan on using them. You can send your leftovers and refuse to Mr. Kovalev: he seems to know all sorts of things, so he may be able to figure out a use for it. Make sure the shores are trimmed down over the next few weeks, though – I’d rather not have any more would-be heroes popping out to save us again.”