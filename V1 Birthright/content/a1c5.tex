\chapter{Ludmila Zahradnik}

The sound of the bell rolling up from the pier could once again be heard through the walls of the manor. Seated at her desk in the hall, Ludmila did not bother raising her head as before.

 

After they had parted ways, Bohdan doddered back up to his abode, calling out for his Acolyte all the while. She came running down from the shrine at the crown of the hill moments later to see what had the aged priest so excited. Taking his arm in hers, she guided him back to his home while he went on about the important work ahead of them.

 

When they reappeared some time later, he was adorned in the vestments of his station: a white cloth cassock stitched with the religious symbology of the Six Great Gods. Lines of cerulean and sienna coloured the folded hem of his collar and scapular, representing the gods of Water and Earth to whom his and the majority of the village’s worship was primarily devoted to. The grime on his skin had been carefully scrubbed away and even the chaotic wisps of his hair had been inexplicably tamed. He strode back down the lane with his Acolyte in tow, who had similarly put on her formal attire. She wore the humble grey robes of an Acolyte, with sienna and white along the edges of her collar: she was an adherent of Earth and Life.

 

By the looks of it, she still had not fully grasped what had ignited the spirit of her mentor. Their solemn procession down to the riverbank had already drawn the eyes of several villagers, however, so she simply followed his lead with a serious expression on her face.

 

The pair had made their way down to the pier and used the bell to call for the attention of the people. Considering that he had renewed the enchantments on Milivoj many times earlier that day, the voice might not have been magically amplified. Perhaps the Acolyte was maintaining an enchantment on him instead, but it was still a rare thing to see all the same. Priests in the service of smaller towns and villages never knew when their spells would be required to aid in an emergency, so they were usually quite sparing when it came to mana usage since the numbers of temple staff in these settlements tended to be low.

 

It was now over a week since then and, as Ludmila continued her work in the manor hall, her quill paused once again to hear his voice drift all the way up the hillside from the riverbank. The change that had come over her old tutor was nothing short of remarkable. Bohdan was well over a century old, having come to the barony as a young missionary in her great grandfather’s time. Merely a month previous, when the village saw the Baron off with his men, many wondered if Bohdan would survive the remainder of winter to see their lord’s return.

 

Now, he hardly appeared to be the man that was bent and shriveled with age, waiting out the end of his days. The fire in his voice and purpose in his actions made him emanate the aura of a bright and optimistic youth bringing the word of the gods to a world starving for their touch. Such was his fervour that the new life and energy that he projected was tangibly influencing those around him.

 

The villagers, drawn by the peal of the bell and the old Cleric’s message, were initially shocked by the decision but quickly shifted from their regular activities to begin preparations for the journey. In hindsight, she decided that her initial opposition to his advice would have been futile if it had come to open disagreement. The trust in the holy man who had served the village for four generations was far greater than in the teenage girl who occasionally acted as the baron’s proxy while the lord was absent. She had neither the influence nor the personal power to convince them to stay; they would have most likely ignored her decision and ransacked the village stores before fleeing with the priest.

 

Realizing that her wandering thoughts had caused her hands to become idle, Ludmila shook her head and refocused on the task laid out before her. The village ledger once again lay open on the desk; a small stack of paper on hand beside it. The crumpled, coarse-grained sheets had been conjured by the Acolyte around the same time the levy had departed and left under a stack of wooden blocks for days to press them flat so they could be written on properly. The ledger was usually brought out to update the village inventories as they were made use of or replenished, but the work today was far more involved and time consuming than simple deposits and withdrawals.

 

The village’s warehouse was filled with products meant for the winter markets in E-Rantel, but now the decision had been made to seek refuge in the Theocracy. In preparation for the migration, she had the villagers sort out their own households as well as spend the days gathering extra provisions in the fields and forests around them. Between the slow travel time upriver by the broken men of the village and the time that they had spent preparing, she estimated that the defeat at Katze had been over three weeks previous. Since travel by land would actually be faster than slowly crawling upstream without proper sailing, it seemed that the monstrosities described in the battle had not been continually deployed. If that had actually occurred, the armies of the Undead would have broken down the walls of the city and overrun the entire countryside before anyone knew what was happening.

 

Based on what they had brought with them for the previous year’s skirmish, the size of the Empire’s Legions meant it would probably require over two weeks to storm E-Rantel, then possibly another week to redeploy for their next advance over the countryside. With the time it would take to ready the village for their journey, she had sent four Rangers inland immediately after the decision to evacuate had been decided upon. She only dared send them to check on the farming villages in the closest baronies, instructing them to fan out and investigate the surrounding countryside nearby. As each had returned over the course of the last day, however, they all reported the same thing: the villages had been abandoned; there was no sign of Human life to be found in any of the areas that they had covered.

 

Ludmila decided that it was approaching the point where the siege might conclude since the Kingdom did not appear to be sending relief, so she shifted away from consolidating the barony’s resources. Her task now was to reorganize the stores and ensure that each family received adequate provisions for the weeks ahead.

 

Food was actually the least of her concerns, as the warehouse had been piled high with the timing of the annual conflict and the resulting delay in the winter markets. There were barrels of nuts and preserved fruits gathered from the surrounding forest. Arrowhead tubers and dried Watercress were sorted in crates waiting to be taken away. Bags of grain were in abundance as well – Mannagrass was native to the marshy valley and its sweetness fetched high prices outside of the barony. There were several racks of dried fish and though they couldn’t bring the hundreds of geese raised in the valley along with them on their journey, several smaller flocks had been gathered. The villagers would eat well with such an abundant surplus, far better than they usually would in the late winter and spring. At least they would have some small comfort in these troubled times.

 

Outside of food, however, there were still many necessities lacking – mostly goods that were due to be replaced in the winter trade. Nails, rivets, rope and blades; assorted tools and certain fabrics were all in short supply. Hopefully what was left would hold out on the journey to the Theocracy. The most valuable goods, such as furs, leatherwork and spare medicines, would be traded in the city that they arrived in for coinage. With this, they would hopefully be able to afford shelter while she petitioned for aid and found a place for her people in the foreign land.

 

Though it had only been a day since the decision to finally leave was made, her work was quickly progressing. Rather than becoming weary as the hours dragged on, the clear target for her efforts allowed her to focus and continue working tirelessly. The challenges it presented were enjoyable to her, despite the nature of the crisis that the village had found itself in, and the flow of quill on paper continued almost ceaselessly through the morning.

 

Almost. Ludmila’s hand stopped as an ugly blot of ink appeared on the page. The nib of her quill had worn down after a few hours of use, tip splitting far up the shaft of the feather and rendering it unusable. Tossing it into a basket alongside a half dozen other such broken quills, she reached for a replacement from the satchel hanging on the wall to the side of her chair. Taking her knife from her belt, she quickly cut a satisfactory nib in the new quill before lopping off the top of the feather so it wouldn’t interfere with her writing by continually brushing into her face.

 

The scribing of ink on parchment continued it’s relentless march.

 

Several minutes passed before the sound of boots coming up the lane to the manor could be heard and a shadow fell across the doorway. Ludmila had picked out several villagers to act as assistants to carry out her instructions. They traversed the village, going between the warehouse and the manor to piece together the orders she had written out for each household’s provisions. When an order was fully completed, they would carry it away to be delivered to the home it had been put together for. Over the course of the previous afternoon and the following morning, twenty-seven out of the village’s thirty households had received their portions. Each time, the team of assistants would return to pick up the sheet on her desk that detailed the next order.

 

The tread of boots approached her desk and a shadow fell across the periphery of her vision. Without looking up, Ludmila held out the next order with her left hand while continuing to work with her right.

 

“It’s time for lunch, Mistress Ludmila.”

 

She looked up from the desk at the sound of the unexpected voice. Bohdan’s Acolyte stood before her, holding a wooden tray with a steaming bowl and a portion of bread twisted and baked into a ring. It was a common meal for the village, which was surrounded by nature’s bounty; they had so many natural ingredients available nearby that one could always make a hearty stew to be eaten every day.

 

“Leave it in the kitchen please, Sophia,” Ludmila said. “I want to make sure these last few families are taken care of.”

 

With only three orders left to complete, she felt that this portion of her duties should be finished as soon as possible rather than break everyone’s rhythm to eat a meal when they were almost done.

 

“You need to take care of your health as well, Mistress,” the Acolyte protested. “You’ve been working nonstop since before the dawn.”

 

Sophia Dolynavec was of an age with Ludmila, and had a compassionate disposition that matched her gentle image. Her regard for the care of her fellow villagers had led to her being taken in as Bohdan’s disciple around the age of ten. The young woman’s passion for the people and dedication to her role had contributed significantly to her growth as a divine caster and a minister to the people; she was already capable of casting second-tier spells.

 

As she approached adulthood, she aspired to complete her training as a priestess in the Slane Theocracy before returning to Warden’s Vale, where she would continue her service. Bohdan was quite pleased that he had been blessed with such a capable successor, reassuring his often fretful disciple that someone of her calibre would be welcomed with open arms in the institutes of the Theocracy. Ludmila supposed that she would now get an early start on that education.

 

“The Rangers will be doing most of the work for the next few weeks,” Ludmila replied. “I will have plenty of time to relax then.”

 

Both of them knew that this was probably not the case. Ludmila’s industrious nature would likely have her find something to do along the journey south, but Sophia knew better than to try and harry her over it.

 

“At least eat it while it’s still warm,” she said. “The meal will lose flavour if it goes cold.”

 

With her last appeal, the Acolyte left the tray on an open space at the edge of the desk and turned away, exiting the manor and disappearing up the path.

 

Glancing sidelong at the steaming bowl of stew which filled the hall with its enticing aroma as she continued tabulating supplies, Ludmila finally gave in to its temptation after confirming that the figures on her latest order were accurate. She reached out and broke the ring of bread up into smaller pieces, dropping them into the stew and bringing the bowl to herself. Her father would have probably scolded her for eating in such a sloppy way, but it was better than wasting time nibbling at her meal with so many more important matters to attend to. She scooped up a thoroughly soaked piece of bread with the spoon and quickly shoved it in her mouth before any drops could spill across the desk.

 

“Mistress Ludmila, we’re ready for the next household…” A man entered the door, but his voice trailed off as he witnessed the young noblewoman stuff her face with a wooden spoon.

 

“Mmomph mmph mrh mm!”

 

Ludmila scowled at the villager’s wide-eyed, uncomprehending expression. Fishing up the order she had held out previously, she waggled it in his direction until he stepped forward to receive it and retreated from the hall.

 

They would depart by the afternoon.