\chapter{Ludmila Zahradnik}

Aemilia rushed down the village path, followed by her four skeletons, meeting her near the entrance to the settlement. She took one look at her mistress’ disheveled appearance and grass-stained clothing before flying into a panic.

 

“My Lady! What happened?” She said as she started fussing over Ludmila, “Are you alright? Are you injured anywhere?”

 

“I’ll be fine,” Ludmila replied as she walked past her maid. “Let’s head to the manor – we’re taking a trip to the city. Prepare a change of clothes and pack enough for an additional day or two.”

 

“Yes, my lady,” her maid ceased checking over her condition and fell in line behind her. “Is there anything else that needs to be prepared?”

 

“No, it will be a short trip. Leave the Skeletons to guard the village. We’ve about twenty minutes to get ready.”

 

Before they reached the manor, a new bout of agonized screaming rose over the hill. Apparently, the man had not had enough.

 

“What is that?” Aemilia cringed at the noise, “Does it have something to do with what happened to you?”

 

“Yes,” she answered tersely while stepping through the doorway of the manor. “There is a man down there being restrained by Nonna – Jeeves is gone.”

 

“W-what?!” Her maid gasped tearfully, “But why? Jeeves was so harmless!”

 

“The man thought that the village had been taken over by the Undead. He stopped me on the road near the bridge to prevent me from suffering a terrible fate at the hands of an evil necromancer.” Ludmila decided not to add who he thought the evil necromancer was, “When Jeeves came to meet me and hand over some paperwork I had asked him to deliver before our departure, he was destroyed by a weapon that was thrown at him.”

 

Aemilia’s hands stopped sorting through the luggage. Her face was conflicted: she probably realized that the man’s reaction was actually the normal one and her own was far from it. With the short time she had to prepare, she began working again after a moment, speaking sullenly.

 

“It’s still not fair,” she said. “He was working for you just like any of your retainers. If a Human servant was attacked like this simply because someone saw them and acted out of alarm, it would be murder.”

 

Ludmila agreed with her maid’s reasoning. While the more powerful Undead like Death Knights would probably not suffer any great injury from panic-driven assaults, the numerous Undead like the Skeleton labourers were probably just as susceptible to rocks and the like being thrown at them as Jeeves was. Something would need to be done so that outsiders did not randomly attack the Undead labourers, whatever their intentions.

 

They finished their preparations in silence, returning to where Nonna stood watch. The man now lay motionless on the side of the road. Ludmila bent over to pick up the rope again, binding his wrists behind his back.

 

“How many more times did you cast that spell on him?” She asked.

 

“Twice,” Nonna answered.

 

The man stirred at the sound of her voice. A bloodshot eye looked up balefully at her.

 

“Whuh…”

 

The sound slurred out of his mouth – he seemed entirely exhausted, but the waves of hostility coming out of him could be almost physically felt.

 

“Why…” His voice grated out, “You are one of us…a follower of The Six…you…betrayed…”

 

“You killed my warehouse manager,” Ludmila answered sternly. “One of my villagers. What do they do with murderers in the Theocracy?”

 

“…insane. This…insane. Betraying…humanity. Is that…what your priest…taught you, girl? …against the will…of the gods. This…heresy.”

 

His eyes closed with his accusation, and he did not stir beyond his uneven and ragged breathing. Ludmila turned and walked back across the road to where her Aemilia was watching worriedly. Straightening the parts of her mistress’ dress that had crinkled from tying up the man, she asked her a question.

 

“What was he saying, my lady?”

 

“He doesn’t understand why we sided with the Undead against a fellow Human.”

 

Ludmila supposed it was only the citizens of the Sorcerous Kingdom that might do so. No matter the race, nation or culture, the perception of the Undead was the same. They were beings that hated the living and only sought their destruction – regardless of how friendly you were to a skeleton that crawled out from somewhere, it would still try to take your life. Directly witnessing the foreigner act out of this common knowledge cast doubt on her confident prediction about how quickly the duchy would recover.

 

A fresh thought brought about by the incident prompted her to ask her maid a question she had not put much thought into before.

 

“Who do you worship, Aemilia?”

 

“The Six Great Gods, my lady,” Aemilia replied without hesitation. “This village worships The Six as well, I think? I saw the shrine at the top of the hill and was very pleased to have somewhere to offer my devotion and prayers.”

 

That she worshipped the same gods as Ludmila and the villagers of Warden’s Vale came as a bit of a surprise. Nearly everyone practiced one religion or the other – as the divine influence of the gods was something that could be plainly witnessed on a daily basis – but the majority of worship had shifted to the Four Great Gods in the northern Human nations. Worship of The Six had waned in Re-Estize to the point that only vestiges of the faith remained – such as the sites of the founding temples of cities like the cathedral in E-Rantel, or those smaller communities that were serviced by lone priests, such as Warden’s Vale.

 

“This man is from the Theocracy,” she told her maid. “He accused me of being a heretic and betraying Humanity.”

 

Aemilia looked back down the hill with a frown. It was evident that she, too, was troubled by the idea. When Warden’s Vale was a border territory of Re-Estize, the vast majority of the enemies that they stood against were the Demihumans and monsters that made the unclaimed wilderness their home. The doctrines of the Six Great Gods and the support of their institutions was an ideal arrangement given their shared interests in defending human realms against inhuman threats. With E-Rantel changing hands to the Sorcerous Kingdom, however, portions of the doctrine practiced in the Slane Theocracy were now at odds with the reality of the realm. Its temples in the north had a far less belligerent outlook, tempered by the reality of the nations they found themselves in, but it was still broadly the same faith.

 

Not only did the Sorcerous Kingdom have a multitude of Undead under an Undead sovereign, Momon had mentioned that there were other races as well – Nabe being from one such species. The faith of the Four Great Gods was more tolerant to non-Human races, but held an unequivocally hostile stance against the Undead that now left them in an awkward position. Since the Sorcerous Kingdom had retained the laws of Re-Estize, thus making it a secular state, Ludmila’s initial thought was that the Temple of the Six Great Gods would simply focus on catering to the Human population, reclaiming its place as one of the core elements of Human culture as other races started to appear in greater numbers. The Temples actually stood to gain in influence by doing so – acting as a sort of glue that bound the Human portions of the realm together; a part of the collective Human identity, so to speak.

 

The vehement reaction late that afternoon cast the shadow of doubt over her optimistic outlook, suggesting that it might have been foolish fancy. She had simply assumed that, like Aemilia and herself, others would also adapt and continue on with their lives. But if there were more zealous groups of people that rose up once they had gotten over their fear, it might become an uprising and the resulting crackdown would tear apart everything that Lady Shalltear had told her about what His Majesty desired to achieve. Perhaps she was overthinking it and the sheer might of the Sorcerer King’s servants would be a deterrent for any such actions, but this recent experience weighed heavily at the forefront of her mind.

 

A black hole in the air appearing near Nonna interrupted the flow of her worried thoughts. A few seconds after it had fully formed, two lithe figures appeared out of the portal, stepping soundlessly onto the grass. They seemed vaguely Human, heads scanning the surroundings purposefully as they took their positions. Four more appeared shortly after, spacing themselves out evenly over the hill. It was only then that Lady Shalltear appeared, accompanied by four of her Vampire Brides.

 

Ludmila immediately curtseyed, spreading the forest green skirts of her dress as she lowered her head. Aemilia followed suit.

 

“Welcome to Warden’s Vale, my lady,” she greeted her liege respectfully. “The Barony is at your service.”

 

Lady Shalltear’s crimson plate boots came into her view, parting the grass as she stepped forward. She was equipped in the sleek full plate armour that Ludmila had seen just once before, in the civil office.

 

“You may rise,” Lady Shalltear said.

 

“I did not expect for you to come personally, my lady,” Ludmila said as she straightened herself.

 

“It’s the first hostile intrusion into this realm since becoming a territory of His Majesty,” Lady Shalltear replied, “so it’s natural that I would respond. Besides, I’m curious what manner of individual would dare intrude in such an utterly pointless way.”

 

“Of course, my lady,” Ludmila pointed down the hill to where the man still lay across the road. “That is the person in question; his belongings are in the pile here.”

 

Lady Shalltear directed her Vampire Brides to retrieve the man and his belongings. The look on Lady Shalltear’s face when he was brought closer was somewhere between confusion and disappointment.

 

“Why would such a puny thing attack your fief?” She asked, “Was he really the only one?”

 

The idea that he was not alone did not occur to Ludmila at the time.

 

“He was the only person that appeared,” she answered. “We do not have the means to search the surrounding countryside even if it were the case.”

 

Lady Shalltear took a quick glance around at their surroundings before speaking once more.

 

“Come forth, my Household.”

 

Swarms of bats fluttered out from her shadow, filling the evening sky. They were followed by nearly a dozen of the wolf-like creatures that Ludmila had seen when the various types of Undead were displayed to her in the civil office.

 

“Search the surroundings for any intruders,” Lady Shalltear ordered and her Household immediately dispersed. “Four of you Hanzos; spread out and be ready to retrieve anything they find.”

 

The lithe, humanoid forms that were standing around the hill darted away at an unbelievable speed, disappearing out of sight within a few seconds. Lady Shalltear turned back to the two women and the Elder Lich on the hill.

 

“Thank you for your assistance, my lady,” Ludmila said. “This should normally be my duty.”

 

“You should already understand that restoring the productivity of the duchy takes priority,” her liege made a dismissive gesture with her hand. “Focusing on your work regarding those priorities was the right decision. Even if there were a hundred million men such as this, they could not hope to prevail.”

 

“About that…” Ludmila began tentatively, “I believe that there are a few things that must be made clear about this incident, my lady.”

 

“Go on…”

 

Lady Shalltear propped her weapon over her shoulder in the same way as she would her parasol as she listened to Ludmila recount the actions of the man and their resulting dialogue. As she shared her tale, both of the Vampire’s expressions turned incredulous.

 

“So he tried to save you from your own subordinates and ended up being captured instead?” Lady Shalltear’s soft laughter chimed over the hillside, “That’s quite amusing – maybe I should play the part of the damsel next time?”

 

“I believe that he was genuinely trying to help me,” Ludmila said, “which I cannot rightfully hold against him. However, it led to criminal acts in my territory…except Re-Estize has no laws regarding a case such as this. Does the Sorcerous Kingdom have any rulings over the destruction of Undead servants? Are they citizens as well? Property?”

 

“Now that you bring it up,” Lady Shalltear said, “I don’t think so. The relationship between the servants of His Majesty and intruders has always been inherently understood; establishing such measures has never warranted consideration.”

 

“Then what would the regular penalty be if, say, one of your attendants was slain in your demesne by an intruder?”

 

“Death.”

 

“...how about a Skeleton?”

 

“Death.”

 

“Trespassing?”

 

“Death.”

 

“...”

 

“I am very thorough in my duties,” Lady Shalltear smiled and puffed out her chest proudly.

 

“I can understand this sort of action being justifiable against hostile invaders, my lady,” Ludmila asked, “but what about those with less belligerent intentions? There are Druids that prefer to live out in nature, and they care very little for national borders. Adventurers often take work that will have them cross those boundaries multiple times as they fulfil their commissions.”

 

“I doubt there will be any of that sort of work for Adventurers here. The duchy has already been cleared of undesirable elements such as brigands and monsters that threaten the citizens. Merchants and travellers will, of course, be welcome so long as they do not run afoul of His Majesty’s will.”

 

“「Gate」.”

 

Lady Shalltear opened a new portal and waved the Vampire Brides through with the captured man.

 

“It doesn’t seem like there is anything nearby,” she said as the man was carried through and the Gate closed. “The search will continue, but there is little reason for us to remain here.”

 

Her immediate duties attended to, Lady Shalltear took on a more conversational tone.

 

“So this is your demesne...” she spoke as she turned to look around once again. “It’s much…larger than I thought it would be.”

 

“Frontier territories are large,” Ludmila replied, “but at the same time they are not well developed. Most of our resources are focused on maintaining the borders, so growth in other aspects is much slower.”

 

Ludmila cringed a bit internally as she saw Lady Shalltear’s gaze turn up to the village. Her liege’s expression did not change, but Ludmila still imagined comparisons being made. Surely Lady Shalltear’s residence was some grand palace in a wealthy, cosmopolitan city, and certainly nothing resembling the dark hole in the ground that was Ludmila’s manor.

 

“We are prepared to depart, my lady,” Ludmila imagined her pride being shredded to tatters and wanted to get away as quickly as possible. “If there is nothing further that needs to be addressed here…”

 

“Yes,” Lady Shalltear said, “we should get going. The Guardian Overseer should be calling for yet another one of her incessant meetings soon.”

 

The unenthused tone of her voice generated curiosity as to the content of those meetings but, before Ludmila could ask any questions about them, a new Gate opened before them and Lady Shalltear motioned them through. The familiar sight of the central district remained little changed since they had left earlier in the week.

 

Stepping off of the gazebo together, Ludmila turned to address Lady Shalltear.

 

“Thank you once again, my lady,” she said. “Will you be available to help transport us back tomorrow? My ship is still at the village.”

 

“Yes, of course,” Lady Shalltear replied. “Around this time, just before the daily meeting, should be fine.”

 

“Thank you, my lady,” Ludmila lowered her head, “We will be waiting for you then.”

\section*{The Rural Fief}

Rural fiefs consist of the majority of the land in agrarian realms, and are home to the vast majority of their population. Commonly situated far from prosperous centres of industry or trade, these poorly developed territories are those most likely to be granted to the gentry and lesser nobility of feudal societies. Like most other noble titles, these lands are granted in relation to the terms defined between liege and vassal and are often shrewdly negotiated in order to prevent the nobles within a lord’s demesne from obtaining too much wealth, influence and power. As such, most minor nobles tend to only earn enough to maintain a modest living not much better than that of a commoner – even as they are bound to the obligations outlined by their noble contract.

 

The smallest subdivisions of manorial land in Re-Estize are often cunningly calculated to only allow for a standard of living barely within one’s means – after taxes, usually…sometimes. This is due to the fact that the majority of the nation’s nobles do not have large, robust, personal retinues where a member of the gentry such as a Knight might need a larger allotment of land to maintain the standards of training, equipment and staff required for rendering military service. This lack of local military power also means that many nobles are unable to project their influence over what is technically their land, and rely on Adventurers in the event that threats arise.

 

As such, a prominent noble often practices Métayage – a form of sharecropping – in their fief. Hamlets and farming villages are locally managed by their tenants for a share of the harvest, and a single representative of variable education and skill – usually a chief or elder – is selected from amongst their number to tend to local administrative matters in a non-hereditary arrangement. While this system optimizes immediate tax revenues, increases the available pool of manpower for levy quotas and foregoes the need to worry about political infighting between aristocrats at the lowest divisions of a demesne, it also leaves next to no opportunity for development available to those at the bottom rungs of society. Disasters which lead to famine and the loss of what little development exists are particularly devastating.

 

The lords of fiefs which have become mired in subsistence may engage in – or be forced into – activities such as producing and trafficking contraband or engaging in banditry, slavery and other criminal acts in order to shore up their ailing finances or fund out-of-reach ambitions. Cases of extreme and arbitrary taxation and tolls to the point of starving one’s own tenants and suppressing trade are, unfortunately, heard of often enough. Contrary to the popular perception of ruling over pastoral vistas from wealthy manors, the majority of rural fiefs – and thus the majority of the nobility – exist in a state where they are little more than serfs themselves, inexorably bound to the land and the highlords ruling over them.

 

A minor noble might occasionally come across the opportunity to improve their demesne in a bid to lift themselves above the circumstances that entrap them. Technology or knowledge may become available to turn things for the better; new resources or more efficient forms of labour can provide the foundation that a fief needs to grow out of the rut that they have been stuck in for generations. Even so, these chances are rare and often require a requisite investment that an impoverished noble cannot afford, while covetous neighbors or even one’s own liege may plot to seize newfound sources of prosperity.