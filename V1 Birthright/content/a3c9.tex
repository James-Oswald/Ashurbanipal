\chapter{Ludmila Zahradnik}

Ludmila found Jeeves at the door to the barn, sitting on his shiny black box. He raised his head at her approach, and rose to greet her with a cheery voice.

 

“Ah, Lady Zahradnik!” He said, “Welcome back.”

 

“How have you been faring here?” She asked, “Have you encountered any difficulties?”

 

“There have been none,” the skeleton replied happily. “The inventories have also been compiled, though I’ve only had one come in to replace some broken plough parts and a chain...”

 

“The workers are clearing the field of hazards right now, so I don’t think they’ll be needing replacement equipment for the next little while,” she explained. “What about the broken parts? Do you think you can repair them?”

 

“I cannot say for certain,” Jeeves replied. “I’ve a vague sense that I can but, much like the repairs from earlier, I believe I need the appropriate tools and facilities.”

 

Ludmila grimaced. A smithy for the village was something that they had always desired, but could never afford to build. It was one of the things she wanted to look into immediately after getting her labourers working, but they probably wouldn’t see one constructed for a few seasons yet – at least until the demands of the population grew large enough to attract a blacksmith to Warden’s Vale.

 

“That’s probably not something we’ll have available for some time. I suppose that I’ll need to make another trip to the city to pick up more parts and some equipment that has been overlooked.” She looked around, “Where’s Nonna?”

 

“Flown off to inspect the work in the fields,” Jeeves pointed out in the direction where they had come from. “It seems that she has quite a number of tasks to fulfill for the central administration.”

 

“She?”

 

“Well I suppose it doesn’t really matter, but she does give off that impression.”

 

Ludmila had no idea what he was talking about. Aside from the voice that sounded somewhat like an elderly woman and the name she had given to her, nothing really gave off the impression that the Elder Lich was ‘female’. Perhaps Nonna sounded different to the Undead in the strange way that they communicated between themselves.

 

“I see,” she guessed it really didn’t matter for the moment. “Did she say what kind of tasks she had?”

 

“Hmm...as far as she’s shared with me, mostly things to do with your work out here – confirming that what is on paper works in practice. There have been a handful of disappointments and surprises so far.”

 

“Oh? Like what?”

 

Ludmila spoke while she retrieved a satchel from inside the barn which contained her lunch, then returned to lean against the barn wall beside Jeeves.

 

“The expectations for farming in particular have not held very well,” Jeeves turned to continue speaking with her. “There seems to be a lot more work involved than the information that was initially collected had suggested. The fault does not lie with you, of course. Nonna said something about the information coming from a source that was restricted to controlled environments and therefore not exposed to real conditions. There was another place – Carne? – where similarly problematic data was being collected, so what you’re doing here seems to verify the errors. Nonna seemed quite excited about the report she was compiling, which contains amendments and improvements: her unique contributions to the administration’s knowledge in this field.”

 

Was Nonna really that talkative? Ludmila could barely get a sentence out of her at a time, never mind this flowing rapport that the Elder Lich seemed to have with her newborn warehouse manager.

 

“Then there’s also the matter of your maid…”

 

“Luzi? Is something the matter with her?”

 

“Ah, my apologies – I meant nothing of the sort,” Jeeves said. “It’s more that her skilled direction of the Skeletons that were assigned to her has made Nonna consider avenues of employment for the Undead that were previously not considered due to their general reception amongst the populace so far.”

 

She recalled the discussion she had with Aemilia over dinner the previous evening. While the potential of the Undead was undeniable, there was also the idea that many would be robbed of their livelihood as a result. No matter how cheap goods became due to the cost of Undead labour, those not earning a wage at all would still be unable to afford to survive. Seeing how the more intelligent Undead were able to eventually figure out their tasks out on the field and coordinate with one another with the very rudimentary knowledge that accompanied her instructions, she imagined that Aemilia – who had collected the skeleton crew again in the morning – would have little difficulty assessing the extent of their capabilities as she went about the village cleaning up the homes and yards.

 

“Well, I wouldn’t get ahead of myself on that part just yet,” Ludmila said as she put away her pack after finishing lunch. “The people still need to earn enough to feed themselves. Implementing those notions immediately might result in a lot of unemployed citizens.”

 

She had given it more thought before she fell asleep the previous evening and continued to do so as the labourers worked in the fields. The standard arrangement between a lord and their farmer tenants in the duchy was divided into shares of the crops that they raised. A more prosperous territory with diverse industry and fertile lands might only have their lord requiring 60 percent of the harvest to maintain security, infrastructure and make investments towards development. A poor territory – or one that was administered by a more severe lord – might retain 80 percent of the harvest as it was their only source of revenue.

 

The majority of the food in Warden’s Vale was actually foraged, fished and hunted in its bountiful surroundings, so the territory under her father had kept the lord’s share at 70 percent – which mostly went towards replacement goods and equipment – and the villagers would barter amongst themselves to balance their needs. Even when the season was not so plentiful, there was never a problem with food due to the natural abundance of the large territory compared to its tiny population.

 

She supposed that, if one Skeleton was equivalent to the labour of an entire farming household and each farmer was allowed to manage four skeletons, taxes would need to change. With the projections in the almanac in mind, a single farmer would still retain twice as much as an entire farming household earned, even after taxes were raised to 90 percent. Contrary to the idea that a lord might incentivize migration by offering lower taxes for a few seasons, households with multiple adults capable of directing four Skeletons each would already temporarily become the wealthiest labourers in the history of the region – at least until prices of Undead-produced goods brought down the prices of their associated commodities in the region.

 

Ludmila was not sure how quickly this would happen, but she understood that lowering the tax rate would have every household trying to dump their excess at the same time, throwing the markets into chaos. Keeping this sort of panic-driven behaviour at bay seemed to be a prudent measure for all parties involved until regional logistics could match production and send the duchy’s goods out beyond the borders of the Sorcerous Kingdom where they could be absorbed by a far larger market. Rather than attracting tenants with a period of low taxes, she would instead turn the high initial tax revenues towards development: offering a higher quality of life in the demesne while commodity prices still remained high. This, in turn, would make her territory more attractive to valuable migrants possessing skills for the advanced industries that she desired…or so she hoped.

 

Adjusting the taxes to ensure that her tenants could afford an acceptable standard of living as the prices they could command for their crops slowly fell would be something she needed to keep a close eye on in the years to come. At least, in the worst case scenario, her people would never come close to starvation and she could rely on the liquidation rates offered by the central administration for lesser revenues.

 

On the other hand, her thoughts might be overly biased due to the current tenant-less nature of her fief. The well-developed territories straddling the main highways had their industries balanced around the existing labour of their tenants and there was not enough available land to broadly employ such measures that could be undertaken in Warden’s Vale. Even with the loss of the majority of the Royal Army, it was still only around two percent of the total population of Re-Estize and the levy for each fief was proportionately small compared to their populations. Even if the claims were accurate and half of the duchy’s subjects had fled, there would still be a massive surplus of labour if the Undead were adopted in their entirety.

 

Barring abandoning their subjects wholesale in a desire to maximize their own gains, the rest of the nobility would have to encourage other industries to take root in order to repurpose their excess labour: a process that might take several generations. Due to her unasked-for circumstances, Ludmila had the dubious luxury of building up Warden’s Vale from scratch using the new systems without needing to undergo such a balancing act.

 

It was a race to attract as many skilled labourers as possible to help develop an advanced workforce and new industries, and she had at least one advantage that was impossible for the others to gain. Unfortunately, the undeveloped state of her demesne also meant she was fighting an uphill battle to attract what she desired versus the wealthy fiefs of the interior. How she could compete with E-Rantel to draw immigrants to the frontier was beyond just making the place look nice.

 

Ludmila continued to mull over the development of her own territory as she walked back out to the fields. Luring out people like Germaine Lenez was a priority, but she wasn’t sure how she would receive her offer, or what she could even offer in the first place aside from the natural bounty of the land. She was savvy enough at her craft that she personally owned her own workshop at a young age and enticing her away from her market seemed difficult, to say the least. With talent came demand, and it wasn’t hard to imagine that anyone in her situation would be receiving offers from every noble in the duchy.

 

The most pressing, however, was her need for a blacksmith. As her labour force grew with the expansion of the farmlands, so would the demand for the production and maintenance of equipment. Building a mill on the river was also high on her list: grains could be ground into flour, and timber processed into lumber for the barony while making it easier to ship to the city. She had a few other ideas, but just building a smith and getting it started up would probably consume most of her funds so she would need to see what the incomes of the fief were like in practice before becoming too ambitious.

 

She was so deep in thought that she almost stumbled on a pile of stones laid on the path. Ludmila looked down balefully at the knee-high stack of rocks and, as she raised her head, she saw that the ‘stack’ was actually part of a long row that stretched all the way to the windbreak of the field that was being worked on, forming a stark line that followed the grassy trail for roughly two kilometres.

 

Is this a farm, or a quarry?

 

Ludmila marveled at the sheer volume of stones that had been delivered; out in the field, the group clearing the debris seemed to only be partway done, Bone Vultures continuously flying back and forth between the field and the road to add to the steadily growing pile. At least she would have some materials to help pave the roads with, whenever she got around to finding a ‘technical advisor.’ Stepping to the side, she continued down the grassy trail, watching the Undead continue to work in the field. Halfway to the end of the line of stones, Nonna descended from the sky and settled down in front of her.

 

“I have a proposal,” the Elder Lich spoke immediately upon landing.

 

“Go on…”

 

It was rare that Nonna actually initiated a conversation – had she ever, actually? – so Ludmila was curious what would prompt her to do so.

 

“Progress with the field work is much slower than anticipated in the materials provided by the administration due to unforeseen tasks. Reinforcements are suggested to improve the timelines for our processes.”

 

“I agree that it’s already going much slower than expected,” Ludmila replied, “but will the administration authorize so many labourers?”

 

“Priority has been placed on the reestablishment of supply lines in the Duchy of Re-Estize,” Nonna told her. “The vast majority of the assets made available for this objective have not yet been deployed due to the lack of activity from the other Human administrators. As such, utilizing them should not place any strain on the Sorcerous Kingdom’s available Undead labour pools.”

 

“How many did you want to bring in, exactly?” Ludmila asked.

 

“At least three other groups of labourers identical to the currently employed forces.”

 

“If the Theocracy sees this, they’re going to think we’re preparing for an invasion.”

 

A part of her laughed at the thought, but being struck preemptively by the Theocracy would normally mean the one-sided obliteration of her tiny fief; it was the first line of defence for the duchy against an advance through the Upper Reaches.

 

“Nonsense,” Nonna’s tone was dismissive. “His Majesty’s servants are under strict orders to not engage in such actions.”

 

“We may know this,” Ludmila said while looking over to the peaks in the south, “but they probably don’t. Did His Majesty inform the Theocracy of his benign policies?”

 

“Not as such, no,” the Elder Lich admitted. “However, after the declaration of war was delivered against Re-Estize, the Slane Theocracy maintained that they would not challenge His Majesty on the claim to E-Rantel.”

 

Ludmila raised her eyebrows and turned her head back to face the Elder Lich.

 

“Truly? The most powerful nation in the region voiced no opposition at all to the conquest?”

 

Bohdan had probably received a shock when he petitioned the Theocracy for assistance, then. Ludmila wondered if she would ever see her people again. Nonna only shook her head in response.

 

“In that case…” Ludmila pondered as she tapped her chin lightly, “the only traces of the Theocracy might be the eyes that probably everyone has sent to stay informed about what goes on in E-Rantel and along their borders. If they’ve assumed a neutral posture, they’d only move to attack if the Undead were marched across the wilderness that divides our borders from theirs.”

 

“Then you will issue the request?” Nonna asked.

 

“I would more than welcome the help,” Ludmila replied, “but what about the equipment? I don’t plan on maintaining such a large labour force until more of the lands are developed. The cost would be a burden on the fief’s budget.”

 

“It should not be unreasonable to request reimbursement, as long as the remaining equipment is returned with the labourers. You should make this detail clear with the civil office, however.”

 

“I’m going? I thought you wanted to make this quick by flying there yourself or something.”

 

“I will send word with my familiar to arrange transport for you to travel directly to E-Rantel.”

 

Come to think of it, Ludmila had not seen the little Devil since before they had left the city. The idea that a magic user could leave their companion somewhere to communicate over long distances like that was something she would have never conceived of herself. She stored away this useful piece of information in the corner of her mind.

 

“Very well, I’ll make the request – but not until today’s work is done,” she told Nonna. “The field should be cleared to plough in a few hours, so I can get the freed labourers to continue with the rest before I make preparations to leave.”

 

“Understood.”

 

They continued walking until they reached the windbreak that divided the first field from the next. Ludmila found a place to sit in the shade, watching the Undead continue their work. Their tireless movements made her wonder once again about the future of common labourers in the duchy.

 

“These Undead took very little instruction to perform these tasks,” she began as she posed a question to Nonna. “Is it possible for them to teach others? Maybe we could simply rotate groups of Undead in and familiarize thousands of them for various types of work by the time we’re done here.”

 

As the Elder Lich did not immediately reply, she focused her attention on one of the Death Knight teams, who was carrying another huge boulder to the roadside. This one seemed to not have been a part of the bedrock.

 

“While it is possible with the Death Knights,” Nonna finally said, “these Skeleton Warriors are similar to the ones crewing your vessel: they cannot teach – nor can they learn. Only Undead with their own intellect are capable of this.”

 

“I only gave them vague instructions though? I thought they did all that on their own.”

 

“This is provably false,” Nonna shook her head. “Regular Humans, such as your maid, can at most control a handful of Undead with specific instructions and direct supervision. There are over 150 Undead currently under your command who have only been given broad instructions and are not being directly supervised. Despite this, they carry on their tasks with an unexpected degree of fine control. It is…problematic.”

 

“…am I doing something strange?” Ludmila sat up at her words, “Should I stop?”

 

“No,” Nonna replied. “It is problematic in the sense that it stands as an anomaly to our current understanding of the world. My opening hypothesis is that it has something to do with your capability as a commander, as you have already demonstrated other abilities which are attributed to them. This would explain the enhanced performance of your labourers, as commander classes may subtly enhance the properties and actions of their beneficiaries. However, this same hypothesis challenges the rule that command abilities should not affect the summons of others.”

 

The unexpected piece of information relating to her own abilities gave Ludmila pause. With how casually Nonna had shared her ideas with her, she wondered if ‘classes’ and their qualities were simply common knowledge in the Sorcerous Kingdom that was taken for granted. The Sorcerer King’s servants seemed to possess a vast knowledge of these topics – perhaps it went hand in hand with their stupendous power.

 

“Lady Shalltear mentioned something about not having clear information about Human classes,” Ludmila said. “How well do the Undead labourers work when someone like a regular farm tenant is directing them?”

 

“Much the same as your maid, I suspect…though as she is under your direct command, your abilities may be in turn affecting her own competency as she carries out her duties in your service.”

 

Did that mean Aemilia’s mysterious sharpness when it came to her duties was Ludmila’s own fault? Thinking about all of the times she would catch and interrupt her when she was doing something that her maid thought was inappropriate…the idea that her maid’s ability to harry her mistress was being enhanced by her mistress’ own abilities felt very much like Ludmila was measuring out rope to hang herself with.

 

“An individual that is familiar with their own vocation should be able to convey those ideas to the labourers that they command,” Nonna continued. “A Farmer will be able to precisely direct the actions that they want their labourers to perform as long as they are straightforward. This does require testing, however.”

 

“I see,” Ludmila said. “Wait – what if I’m not able to command all these new skeletons that I’m bringing in? Surely this ability has limits…”

 

“You may consider that a test as well,” Nonna told her. “In the worst case scenario, they will carry out your most recent instructions in a basic manner until they have completed them. In any case, your own people should understand their own abilities…there is no documentation regarding yours?”

 

“There might be something recorded somewhere, but I’ve not personally seen anything of the sort. All of the bits and pieces that I know of come from the recounting of past events by the people around me, and from legends handed down through the generations by oral traditions and entertainment.”

 

“How foolish,” Nonna scoffed. “I propose that you remedy this ignorance to optimize the performance of yourself and your people.”

 

Ludmila wasn’t sure if she was being offered friendly advice or if she was being derided – perhaps it was both. 