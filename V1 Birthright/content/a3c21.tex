\chapter{Ludmila Zahradnik}

The two Death Knights in Ludmila’s escort hefted the largest trunk that had been carried over to the side of the road onto their shoulders. It had been trimmed of its branches, but the twelve metre long section of aspen still looked remarkably heavy. She kept looking down at the dirt road as they descended to the valley floor in front of their procession, expecting deep footprints to be left in their wake, though all they left were some shallow imprints in the mostly dry soil. She would need to see about laying stronger roads soon, or the dirt path would become a rugged nightmare with so many heavy loads being transported upon it.

 

Several wisps of smoke rose from the village in the distance: they were too voluminous to be cookfires from the homes and there had been no panicked villagers coming up the road to raise the alarm, so she thought it must be Ostrik starting his charcoal production.

 

With the skies being uncharacteristically clear for the past few days, the waters of the Katze river had receded somewhat. She directed the Death Knights to use the shallow ford that had appeared behind the bridge, not wanting to risk the wooden structure’s collapse. As the carts crossed, she saw that a section of the reeds and grasses growing along the shore nearby had already been cleared away. She didn’t see anyone working below, however – the women of the village must have stopped to prepare the evening meal. Large piles of wood and various pieces of burnable debris had appeared on the hillside, and Bone Vultures flying back and forth from the fields periodically added more to them. The sense that her demesne was slowly coming back to life gave her a feeling of energy that further contributed to her good mood.

 

After the Death Knights dropped off the log on the sandy flats, Ludmila parked the carts near the piles of wood and had the Undead transfer the contents of the carts to them. Walking up to the village, she found Aemilia standing at the entrance with a moist towel in hand.

 

“Welcome back, my lady,” her maid lowered her head as she offered the towel for Ludmila to refresh herself with. “Did something happen?”

 

Ludmila stopped wiping her hands and looked at her maid, puzzled.

 

“A few things happened…what do you mean?”

 

Aemilia pointed her finger and Ludmila looked down. Her legs were encrusted in green and brown muck up to the knee.

 

“Oh...I went into the marsh to see if the Undead could do something.”

 

“I will prepare a change of clothes, my lady.”

 

Aemilia’s maintained her even and respectful tone, yet Ludmila felt like she was being scolded for some reason. She retreated to her manor to change before any other flaws with her appearance could be found from the short trek through the nearby valley. After an early supper, she checked on the smith’s progress.

 

She found Ostrik tending to several large, smouldering mounds near to the woodpiles below where he had set up his forge the previous evening.

 

“Is this how charcoal is made?” She asked from the side as she watched the lazy trails of smoke travel south over the river.

 

Ostrik hadn't been paying attention to his surroundings and started, looking up from his work. He scowled for a moment, before he realized who had spoken.

 

“Yes, Lady Zahradnik,” he said as he rose to face her. “It’ll take about a week before anything comes out of it, though. Normally, you’d build a kiln away from everyone to keep the fumes away, but the wind is so consistent here that this spot will do for now. I’d still recommend building one if you plan on producing charcoal for the long term, though.”

 

She wasn’t sure that she would. Just glancing over Ostrik’s work, it felt like he was going through a lot of wood. Her own plans currently only extended to clearing enough farmland for five villages, then seeing what she could do about turning the harbour village into a town. The smith turned back to continue tending to the mounds, making holes and filling them in some sort of strange ritual.

 

“How long will what you have here last you?” Ludmila asked.

 

“Just this?” He looked around at the mounds: the one nearest to the village was small compared to the others, standing about a metre high and three metres across; the others were much larger, each around twice her height and over ten metres across. “This small one here is to just tide me over until the big ones are done – I might have to get another one started, just to be safe. Once the large ones are finished, I’ll be good with that portable forge until late next spring, probably. The bigger we make the new forge and furnace, the more it’ll burn, obviously...but I can’t say how much until we actually build it.”

 

Ludmila didn’t have the slightest clue what went into building one – or building anything, really. She thought she might be able to fish out some information from Ostrik, whose travelling lifestyle seemed to have made him quite knowledgeable.

 

“What will you need before you can start on that?” She asked.

 

“Carpenters, masons...depends on how fancy you want it made,” the smith answered. “If you want something with all the bells and whistles, you’ll be looking at involving Alchemists and other magical crafters, too. You’re set on materials here already – speaking of which, I brought up some of the ore I was talking about from the marsh.”

 

Ostrik finished his rounds and led her back up to the forge. The smith had fashioned a long wooden ladder, using it to climb directly up to his workplace. Ludmila looked down at it after scaling up after him.

 

“Make sure you don’t leave that there.” She warned him.

 

“Why’s that?”

 

“We’re on the edge of the wilderness,” Ludmila said. “Attacks can come at any time. The last thing we need is someone conveniently putting out a ladder for invaders to climb.”

 

“I mean no disrespect, Baroness,” the smith said as he looked to the Death Knight escorts who had crawled up the ladder after her, “but I don’t think you’ll have to worry about that with so many of these Undead here.”

 

Ludmila did not like his flippant response. As long as she lacked properly trained sentries and patrols to secure the borders again, it was all too easy to sneak into her territory through any number of routes.

 

“The village was attacked two days ago,” she tried to keep her tone even but couldn’t prevent the edge from entering her voice. “A sole intruder made it to the base of the village completely undetected and killed my warehouse manager.”

 

Ostrik turned around at her statement with an expression of disbelief. After meeting her gaze for a moment, he lowered his head.

 

“Apologies, my lady,” he said solemnly. “I forgot myself.”

 

In the silence that followed, a Skeleton walked by carrying a bucket filled with clay. It passed the two Humans and deposited its load in a pile behind the smith before returning down the path it had come from. It cared little for their interaction, blithely carrying on with its task. Ludmila looked past Ostrik to the pile: there was a short ring of the material fashioned beside it.

 

“A bloomery,” the smith answered the question on her face. “It’s a simple furnace for smelting iron ore. I’ve been putting it together between checking the charcoal piles.”

 

Ostrik walked over to a pile of ruddy orange rocks which looked to be as wet as the mud nearby. The pieces were of varying sizes; a few were larger than her head.

 

“This is the ore I was talking about,” he said somewhat proudly.

 

“I’ll take your word for it...but it just looks like a rock from the marsh.”

 

“Well, technically it is a rock from your marsh,” Ostrik replied. “This is called bog iron: it’s easy to extract and easy to find in places like this – when no one’s been harvesting it, the stuff just keeps piling up.”

 

“Harvest?” Ludmila tilted her head, “Do you mean to say that you can grow iron?”

 

“Hmm...not in the sense that you can grow crops or raise livestock,” Ostrik looked down at the pile of ore. “An old shaman I met out east just beyond Karnassus tried to explain it to me – something about spirits in the water gathering iron or some such. I don’t know how true that is, as I’ve never seen anything that looks like a spirit or a fairy or whatever while picking this stuff up. What is true is that it does appear in places like this. He said that areas with bog iron could be harvested once every few decades, so it’s a slow thing.”

 

Ludmila’s gaze turned to the familiar marshes of her home. They stretched far to the north, and took up two thirds of her land. The villagers had long since come to appreciate and harness the productivity of the valley in terms of its vegetation and wildlife, but they had never known there was such mineral wealth as well. She must have gone by similar lumps of ore thousands of times while traversing the marshes throughout her life.

 

“So when you said there was enough iron here to build a small city, you weren’t exaggerating.”

 

“No, my lady,” he grinned. “A family could make their livelihoods collecting ore and selling it, and they’d probably never run out with this much area to cover. It’d be a different story if you established a foundry here and started exporting your products, of course, but I meant what I said yesterday. Your territory has a balance of all the right things, as long as you know what to look out for. Even as much as I’ve seen in all my travels, there are probably people out there that’ll notice things that I haven’t.”

 

“That’s very encouraging to hear,” Ludmila turned back with a nod and a slight smile. “I’m glad you decided to come work for me, Mr. Kovalev.”

 

“Pleasure’s all mine, Lady Zahradnik. It feels good to be useful, especially in a promising place like this.”

 

Another Skeleton walked by with its bucket of clay, depositing it with a splat. Their eyes followed it as it came and went.

 

“Have you tried using the skeletons to find ore?”

 

“Sure did,” he laughed. “It was the first thing I tried. Thought I’d have it easy, but they can’t tell anything apart. Even scooping up this clay is hit and miss: sometimes I get mud and rocks, other times a clump of roots with a plant attached.”

 

“What have you found them useful for, then? Besides carrying mud around.”

 

“Carrying wood around,” he raised his hands disarmingly when Ludmila frowned a bit at his words. “They’re tireless, so they’re ideal for pumping bellows as well. As long as you have the fuel, the furnace that you build in the future should be able to easily turn out steel if it’s designed to harness the labour of a group of Skeletons operating a set of bellows. They can swing hammers too – I guess that’s a given since Skeletons always seem to carry weapons around – but I prefer to do that part myself. They’re not too great at working together with me and it breaks my flow constantly. I mostly have them moving things around for now. Maybe I’ll come up with other ideas later.”

 

“Make sure you report your findings to Nonna when you get the chance,” Ludmila told him. “Before I get going, I’ve a load of equipment that needs repairs.”

 

“Sure, let’s have a look.”

 

Ludmila motioned for one of the Death Knights to pick up the crate that she had delivered earlier to the warehouse nearby. It returned after a few minutes, placing it on the ground between them. Ostrik let out a sigh even as the crate was laid on the ground.

 

“What’s wrong?” Ludmila asked worriedly.

 

The smith picked up one of the broken tools, a shovel blade with a large chunk missing, leaving a sharp edge behind.

 

“This is cast iron,” he said simply.

 

“Is that bad?” Ludmila asked.

 

“It’s not something I can fix with what I have here,” the smith said. “Cast Iron is brittle and can’t be reworked like forged iron or steel – it just goes everywhere if you even try. Doesn’t mean that you’ve been cheated, though: it just means that whoever made these thought they were tight on time so they chose a method that could turn out a lot of equipment quickly.”

 

Ludmila considered the smith’s words. Forgemaster Mesmit’s account did match Ostrik’s assessment of what had happened. Knowing this did not help her at all, though.

 

“What can be done, then?” She asked.

 

“Well, if you’re desperate for replacements,” he answered, “you’ll have to return to the city to buy them. The scrap here can be sold to the forges in the city that can do something about it. If you think you can hold out, I would say keep the scrap here and we can turn it into steel when the forge furnace is ready.”

 

Since there should be enough spare equipment to finish the fieldwork, saving the materials to make steel tools sounded like a good idea. They would be far more durable and could be repaired, and having the work done locally meant she would ultimately save on the costs of purchasing them from the city.

 

“I’ll leave you to your work then, Mr. Kovalev,” she decided she had taken up enough of the smith’s time. “We can hold onto this scrap until it can be turned into steel. Hopefully I’ll be able to find who I need to start construction work soon.”

 

Ostrik waved absently as he returned to fashioning the bloomery. With most of what she had planned out for the day completed, she looked forward to the free time she would have to practice her skills with her new equipment. Approaching the warehouse, she slowed her steps when she noticed Nonna had returned.

 

“Did Mr. Boer speak with you about his findings?” Ludmila asked.

 

“He did,” Nonna replied without looking up from her work.

 

“After everything that’s come to light,” Ludmila said, “I’m curious about what went into the information provided by the administration’s materials. You mentioned that Carne didn’t have the same issues with equipment failure, yet the recommendations provided for spare equipment actually line up with our losses so far. Then there’s the yield estimates: they are far beyond what a farming village in this duchy can produce, yet the numbers seem without variation or expectations accompanying them. Why is that?”

 

Nonna’s pen stopped scratching, and the Elder Lich looked up at her quietly.

 

“The answer is as you have probably already surmised,” she said.

 

“You knew that the information was this unreliable?” Ludmila frowned.

 

“It was not until today that I requested the source materials,” Nonna replied, “due to all the discrepancies that I personally observed on the field. The components that went into the data provided by the almanacs are from various sources conducted under different circumstances: thus my creation of the new control sample that we spoke of earlier today.”

 

“Doesn’t that mean we practically went in blind on this? No wonder the lease was free…”

 

“Data is data,” Nonna said flatly. “The available data was that which was collected from past records of the region, some information from Carne Village and experimental results provided by the Imperial Ministry of Magic. Every action taken using the available materials is a hypothesis through which further observations are made possible, where information is proven or disproven and ultimately added to the Sorcerous Kingdom’s archives.

 

As observations are collected, knowledge expands. As Knowledge expands, rules, procedures and formulas are established. Anomalies that challenge our perception of the world inevitably present themselves, and so observations are made and knowledge expands once more…at least this is something one of my seniors has said.”

 

“I didn’t know Elder Liches had seniors,” Ludmila said bemusedly. “Who said this?”

 

“Titus Annaeus Secundus, Chief Librarian of Ashurbanipal – one of the leading researchers of the Sorcerous Kingdom.”

 

“He sounds like an interesting person,” Ludmila said. “Maybe I’ll be able to visit this library one day.”

 

“That is highly unlikely,” Nonna told her, “you have neither the permission nor the qualifications to enter.”

 

Ludmila smirked at Nonna’s flat refusal. Walking past her, she entered the warehouse to put away the produce she had returned with, in addition to the basket of fish that the captain had apparently succeeded in catching from the river. Scanning the shelves to find a place to arrange everything, an object in the near corner caught her eye.

 

It was a large, glossy black box of unknown construction which stood around the height of her knee. A strip of paper scrawled with unknown writing sealed the container. Eyes wide, she looked back to Nonna standing beside the entrance outside – who once again seemed to be busy writing in her tome.

 

“Was this always here?” Ludmila asked.

 

“No, it arrived several hours ago,” the Elder Lich said.

 

“How did it get inside?”

 

“...it was delivered, by order of the Royal Treasurer.”

 

“By who?”

 

Nonna did not answer, continuing her work in silence.

 

Ludmila knelt in front of the box, running a hand over its smooth surface. She hesitated for a moment before pulling open the seal and lifting the lid. With the noise of bones clattering against one another, a familiar looking Skeleton rose and formed itself, the lens of its golden monocle reflecting the afternoon light coming in from the warehouse door.

 

“Good day, madam!” It bowed slightly at the waist as it greeted her, “How may I be of assistance?”