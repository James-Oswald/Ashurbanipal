\chapter{Ludmila Zahradnik}

Before dawn, the pier was empty once again.

 

Ludmila sighed – something she felt was beginning to happen far too often – as she stood inside the shadowed wooden door frame of an emptied home. After a fitful night plagued by worries, she had risen from her bed an hour before sunrise to finish her preparations and tasks for the morning. In doing so, she hoped to head out early to find out more from the men that had arrived the previous evening as their families prepared for the day.

 

Instead, she was greeted by the conspicuous sight of the empty pier upon leaving the manor. There was no sign of the ship down the river and, after rushing down to the shore, she noted no broken ropes nor signs of damage to the moorings that would indicate that the ship had been swept away somehow.

 

It was only after she trudged back up to the village and looked around that she was able to piece together what had occurred: two of the returned men had stolen away some time during the night, taking their families with them. The first abandoned home she had come across was that of the man who barreled through her when the ship had bumped up against the pier. The second, in which she currently stood, belonged to the man handling the rudder that had slipped away during that same commotion. With the village as short-handed as it was, there was no watch posted at the harbour overnight and no one had noticed their departure.

 

Peering into the living space, Ludmila took a mental inventory of the interior. The home was stripped of bedding and clothing; the shelves and cooking area similarly bare. Clearly, the family had left with no intention of ever returning. Stepping back out into the path, she brooded over the increasingly dire situation that cast its shadow over the fief.

 

The river was the village’s lifeline to the interior of the duchy, so losing the vessel that was their main method of transport was a crippling blow. Over the recent generations, the borderlands of the Kingdom of Re-Estize had become a resurgent wilderness, turning Warden’s Vale into an isolated outpost of the Duchy of E-Rantel. The overland route leading north through the old roads and regrowth back to the western highway which led to the city would take heavily laden travellers nearly two weeks on foot with very limited capacity for cargo. The route also carried with it a high risk of being attacked by wild animals, wandering monsters and the occasional loose group of bandits or Demihumans. Fashioning a new vessel suited for transporting the barony’s goods would similarly be impossible with what they had on hand.

 

With the bounty of nature all around them the people never suffered any shortfalls of food, but they still needed to trade in the city markets for tools, parts and other necessities that were crafted elsewhere. The city in turn would be lacking in the produce and resources – timber, furs, hides and other raw goods harvested in the south – that they would exchange in turn. Though the production of their mostly undeveloped fief was tiny compared to the vast farmlands and forests managed in the interior, it still served as a small link in a chain of logistics that could be adversely affected by their missing contributions.

 

Not only would it cause uncertainty and hardship for others that depended on their trade; it would also create political ill-will against them for the instability that their failure to deliver would cause. Ludmila looked back outside, searching for signs that the others had become aware of what had occurred overnight and its implications.

 

Occasionally a villager would come up the path, carrying water drawn from further upriver, out of sight of the harbour. They were all simply focusing on their morning routines, preparing for another busy day. By the afternoon, however, it would be common knowledge and the topic of every discussion. Ludmila wondered what she should say when that happened.

 

“Mistress, the priest was looking for you.”

 

A young woman coming down the hill with an empty pair of buckets latched onto a carrying pole called out to her. Ludmila turned her head to look up in the direction of the voice. The woman's arm was stretched out to point further up the path, hand gesturing loosely.

 

“Did he say what it was about?” Ludmila asked.

 

In reply, the woman simply dropped her gesturing arm and shrugged. With the morning’s chores pressing her, she did not seem to express any interest beyond making sure Ludmila was aware that someone was looking for her. Her pace did not slow as she rounded the hill on her way to the river bank.

 

As Ludmila made her way back up further into the village, she shared a similar exchange with three other tenants on the same errand, but learned nothing new. This was how information travelled amongst the common folk who had no pages or couriers: tidbits being passed from person to person until it was seemingly no longer relevant. Whether it actually reached its intended – or unintended – recipient was something to be seriously considered in larger towns and cities, but in small villages it was a fairly effective way to spread word without interfering with the peoples’ daily lives.

 

Ludmila found the priest standing outside his home near the top of the hill. One of the elders of the village, he had a thin, weathered appearance that spoke of his service as minister to the village for multiple generations. Aside from the icon of the Six Great Gods that he wore around his neck, there was no real way to distinguish his plain-clothes appearance from any other old man in the village.

 

Wisps of white hair trailed loosely in the air as the old priest turned his head in Ludmila’s direction. He made no move to meet or greet her – an agent of the gods did not answer to mortals, after all. She came forward with the customary greeting, lowering her head as she curtsied.

 

“I hope the day finds you well, Priest Bohdan.”

 

Awaiting his response, she did not straighten herself. The aged priest extended a shaky hand, placing it firmly on Ludmila's lowered head.

 

“Blessings of The Six be upon you, Childe.”

 

She felt a wave of warmth as the priest’s Blessing washed over her. Proper supplication before a divine agent was answered with a gift from the gods. As far as his congregation was concerned, the spell allowed one to perform their daily tasks with slightly greater proficiency as well as tending to avert minor accidents and injuries for a short period of time. Though mostly a ritual exchange, it still held a bit of practical value for the people.

 

With the incidents as of late, Ludmila felt she would need any help she could get…but when she raised her head to face the priest, she only saw more trouble painted on his face.

 

“You sent word for me,” she said. “What’s the matter?”

 

Ludmila went straight to the problem at hand, but Bohdan seemed hesitant to respond.

 

“The men that were carried in yesterday,” he told her. “They passed away during the night.”

 

“What!”

 

The exclamation came out sounding something like a squawk. Looking around to see if her outburst had drawn any attention, Ludmila lowered her voice.

 

“How?”

 

“Of...dehydration,” the priest replied. “I could not discern any other cause.”

 

“Do you mean to tell me that men travelling on a river died of thirst?” Ludmila was incredulous.

 

“They did not go peacefully,” Bohdan replied gravely. “Their eyes were wide open the entire time, but they seemed to be staring…elsewhere. They moaned weakly through the night, thrashing at their surroundings until cuts and bruises covered their bodies and their fingers bled raw. We had to restrain the men to keep them from harming themselves further, but they still kept screaming in hoarse whispers until their throats cracked and bled and they could scream no more.”

 

The priest paused to swallow as he gave his report, a haunted expression on his face.

 

“In all my years I have not seen the like,” he said. “Just what happened to these men?”

 

That’s what I would like to know. Ludmila thought to herself.

 

What could disturb grown men to such an extent that they would die of thirst while travelling on a pristine river?

 

Resolving herself, Ludmila stepped into the building to see what had happened with her own eyes. She wrinkled her nose as the pungent odor of herbal tinctures and soiled cloth greeted her. The priest’s Acolyte was watching over the bodies, sharing the same grim expression as her mentor. The corpses of the two men lay sprawled on their cots, woven straps binding limbs still rigid in death. Bloodied fingers formed into grasping talons around wherever they could find purchase, leaving shallow marks and dark red smears on the surrounding furniture. One man lay on his back, face covered by a clean linen cloth. The other had somehow become tangled in his beddings and bindings, face hanging out with a ghastly expression over the edge of the covers that he lay upon.

 

Ludmila knelt to inspect the man. Even in death, his eyes were still wide open, his pallid skin only lending to his horrified visage. She couldn’t help but compare expressions between the corpse and its tender.

 

Reaching out, she pulled a wooden cup from under the cot. It had rolled along the floor, leaving a trail of clear liquid near her feet. Moistened towels were scattered about as well – the priest and his Acolyte had most likely tried to help the man take liquids, only to be met with violent resistance.

 

Placing the cup on a nearby counter, she turned to face the two defeated-looking caretakers.

 

“Could you make anything out in their delirium?”

 

“Nothing,” Bohdan replied.

 

He kept glancing back and forth between the bodies and the doorway, as did his assistant. The night’s vigil had left them plainly ill at ease, and they stayed well away from the apparent source of their anxiety. Ludmila breathed a sigh yet again. There was no sense in lingering and nothing further could be learned here. The priest and his Acolyte stood aside respectfully as she passed them on the way out, but were almost stepping into her heels in their haste to leave as well.

 

Most of the villagers were out and about by now, many were in or around their homes working on winter crafts, mending clothing or trying to fix odds and ends. There was less daylight in the winter months, so the majority of the activity was concentrated into the seven or eight hours of good light that reached their place in the valley.

 

Further afield, there were dozens foraging outside of the village. Though none dared enter the nearby forests with recent events casting a pall of uncertainty over the settlement, they were still comfortable enough to harvest the various plants that grew abundantly over the marshy valley floor. Watercress, Arrowheads, Mannagrass and various reeds and rushes carpeted the terrain, but there were never enough hands to even remotely harvest it all.

 

They also managed a flock of hundreds of grey geese, which dotted the vast feeding grounds, providing fowl and fresh eggs for the village on a regular basis. As well, there were a handful of men and women casting nets into the river for fish, eel, molluscs and crustaceans. With the seasonal conflict being so late this year, the winter markets would also experience the same delay and the village was using this time to increase the quantity of their exports.

 

Not that it could be delivered in a timely manner without their ship. The thought again made Ludmila keenly aware of the slowly-building mental pressure that had been accumulating since the previous evening. Giving her head a shake, she brought her thoughts back to the task at hand. With two dead and two missing, there was only one man left out of those who had returned.

 

She was well acquainted with the man; her own brothers being of a close age with him, and he was often seen together with them during their childhood. Even now, he usually served as a sentry for the manor and had a place on the regular border patrols. Since she was familiar with the man she had meant to see him first, but the events of the morning had simply swept her along powerlessly like some leaf in the current. He also had been the only one to walk away from the pier in a somewhat normal fashion, so she had hopes of getting some real information out of him.

 

On the way down to his home, Ludmila looked anxiously for signs of activity in his family’s home. For a tense moment, she thought that he, too, may have fled, but the thin wisp of smoke that rose from his family's home dispelled those fears.

 

Rapping her knuckles lightly on the thick wooden door, Ludmila announced herself.

 

“Milivoj, it’s me.”

 

It didn't take long for the door to open in response.

 

The tall man appeared, almost entirely filling the doorway. He still had a haggard appearance: dusty brown hair unkempt; face unshaven, tired and worn, but at least some color had returned to him. Through a space between Milivoj and the doorframe, she spotted his sister putting away breakfast. They seemed to be having a late morning, but she could hardly fault them for that after seeing what happened to the other men. For now, she was simply thankful that there was someone left to speak of what had occurred.

 

Milivoj seemed to sense what she had in mind, shifting uncomfortably and unable to meet her gaze. Ludmila opened with a courtesy, regardless.

 

“Are you well enough to speak for a while?”

 

After a short silence, Milivoj closed his hazel eyes and exhaled heavily. Stepping out of the doorway, he rested his back against the stony wall and slid down until he sat on the ground, nodding in acquiescence.

 

Milivoj’s defeated actions gave Ludmila pause. The entire situation since the previous evening had left her with an unstable feeling, like she was trying to cross thin ice; the people and world around her threatening to break apart with every step she took towards finding the answer to this mystery.

 

She looked for a place where she could sit, but ultimately decided to step back down onto the path to speak with him face to face. Upon turning around, she was surprised to find that Bohdan had followed her. The presence of the knowledgeable elder was welcome, however, so she paid him no mind as she started to question the waiting guard.

 

“What happened?”

 

As the words left her mouth, Ludmila felt that they were wholly inadequate. She wished she could reach out and snatch them out of the air so she could start over again, but Milivoj still flinched as if he had been struck. The big man was already sitting flat against the wall, but he seemed to sink into the stone as he deflated upon being called to recount his experience. His words came out in a hollow voice, unlike the boisterous man of mere weeks ago.

 

"We...lost?” He gave a short, weak laugh, but Ludmila failed to detect any humor in his words. “Everyone – everything. All is lost. We are lost.”