
\section*{The Temple of the Six}

The faith of the Six Great Gods of the Slane Theocracy – or all of Humanity, as their staunchest adherents would have one believe – is a powerful institution steeped in centuries of venerable tradition. With the legacy and bloodlines of their deities firmly within their grasp, agents of their faith are accustomed to dealing from positions of power and authority. Caretakers of their race, they guard Humanity against any and all that stand to threaten it; their agendas enforced far and wide through the strength of arms and magic provided by the elite forces cultivated through a long program of international eugenics: which has produced many of the most powerful individuals in the history of the Human race. Their institutions and temples are an ubiquitous sight across the realms of Humanity, they play a central role in all aspects of life.

 

Or so they had meant for this reality to be so.

 

Shortly after the defeat of the Demon Gods, the Theocracy embarked on a great venture to seed the lands wiped clean of undesirable elements with settlers of their own choosing. With one foot firmly planted in the doctrines of the faith and the other in the blessing and support of their fatherland, these Humans spread far and wide to the hopeful expectations of the Theocracy, who interspersed many strong bloodlines amongst them. Not long after, however, the fruit that they had expected to be borne through their efforts turned sour. The intricate web of their influence frayed as distance and time wore away at the bonds between the ancient nation and the nascent Human states subject to its influence. Apostasy eventually became so rampant that a schism manifested, resulting in the majority of the population worshiping only four of The Six in the northern nations of Roble and Re-Estize.

 

The latter established a secular government, but half of its territory broke away shortly after its founding to give rise to the Baharuth Empire: which retained this political separation from the Temples. Though their operations were adversely affected beyond the borders of the Slane Theocracy, the scattered Temples of the Six Great Gods remained even as the practice of their faith diminished. The approach they established in order to uphold their sacred charge over the Human nations diverged greatly in some aspects from the zealous practices of their home country. They still continue to strive to become a pillar of support for humanity, but the populations that they serve care little for their zeal and their rulers suffer no challenges to their authority.

 

As the Temples are an administrative organ in addition to a religious one, the work of the priesthood extends not only matters of faith, but also the careful monitoring of the myriad of bloodlines and cultural practices founded through their influence. Fortunately, a few important aspects have remained. Though not explicitly belligerent, Human-centric thinking is still prevalent in every one of these nations and the practice of monogamy is still widely promoted, making the tracing of valuable lineages easier to achieve. Though outwardly perceived as places of worship, every cathedral, temple and shrine is also an office: a part of the vast network of faithful administrators that helps to manage the future of Humanity and ensure its survival in a world of savage and powerful enemies.

\chapter{Ludmila Zahradnik}

Bishop Austine accompanied them as they made their way out of the Cathedral, the trio stopping at the top of the short set of stairs in front of the door. His gaze followed a patrol of Death Knights crossing the far end of the plaza, their plated footfalls clearly heard all the way from where they stood. Ludmila’s attention was directed to their left, however, to look at one of the nearby buildings.

 

“What has the Adventurer Guild been doing since the annexation?” She asked.

 

The Bishop turned his attention away from the Undead to follow her gaze to the nearby guildhall.

 

“As far as I have seen, not a whole lot,” Bishop Austine told her. “While they have not closed their doors, there are only a few parties of Adventurers going back and forth. It is nothing at all like what it used to be. The Magician Guild appears to be open as well, but the Wood Golems that they used to guard the place are gone – I suspect that they have long since transported all of their valuables away to other branches. According to one of our Clerics that paid a visit recently, the Adventurer Guild is in a similar situation. She said that, since the formal transfer of the duchy to the Sorcerer King, new requests have ceased entirely and most of the old ones are gone as well.”

 

“The source of most of their work is at a standstill, though,” Ludmila said. “The nobility has not exactly been in the position to commission new requests or fund bounties for our territories in the past few weeks, and merchant operations are nearly nonexistent.”

 

“That may be so,” he replied, “but just having a glimpse of the Sorcerer King’s forces, I have no doubt that the vast majority of what Adventurers normally deal with will no longer be an issue. It has become like the Theocracy, in that sense: they absorbed their local Adventurer Guild branches into the ranks of their forces long ago.”

 

Ludmila’s discourse with Lady Shalltear in the past few days suggested that this might be the case for the Sorcerous Kingdom, but she still was not sure of the exact details on how national security was handled now. There were requests for military assistance available to be submitted at the civil office, but surely there were a few things that Adventurers would be more suitable for over the Undead. The single-minded and uncommunicative attitudes of the Death Knights and Elder Liches stationed everywhere made it unlikely that they would be ideal for situations that required an unconventional or indirect approach, or one that saw benefit from a Human touch.

 

“I am sure there are still things that they can do,” Ludmila ventured. “The nobles occasionally hire Adventurers for entertainment, or as diplomatic couriers and escorts. Merchants commission them to find rare or difficult-to-obtain resources and various other things that employ their unique skillsets.”

 

“I suppose,” Bishop Austine said. “It would require that the Adventurers actually have those specific skillsets, however. The majority are essentially combat specialists.”

 

Ludmila frowned; there was no easy way around the problem for most. She very much would have liked the Adventurer Guild to continue existing in the city, as House Zahradnik had a long relationship with the organization. That was not something she could do much about though – she had her hands full seeing to her own matters. Turning to say her farewells to Bishop Austine, she returned to where her wagon was awaiting them near the main street crossing the plaza.

 

Taking her seat, she flipped through the addresses she needed to visit as Aemilia settled in beside her. Her maid had been something of a shadow during the entire visit, ever ready to come to her mistress’ assistance but never a distraction while she spoke with the Bishop. Ludmila’s fingers stopped as she flipped to the page with the ‘strange’ Farmer.

 

“Luzi,” she asked, “did the others say anything about this weird farmer?”

 

“Mrs. Ro’eh did not say anything, my lady,” Aemilia replied, “but the Linum sisters said he was ‘creepy’.”

 

That was not helpful at all – all it did was make Ludmila more apprehensive about meeting him. Looking over the addresses again, she realized that she had no idea where anything was in the city: it was actually Lady Shalltear and the Soul Eater that had done all of the navigation when they had spent the day together wandering around the city. She held out the sheets in front of herself and her maid.

 

“Which one of these is closest?” Ludmila asked her.

 

After glancing over the papers, Aemilia pointed to the nearest address. It was the Farmer, of course. With a sigh, she relayed the address to the Soul Eater, and it drove the wagon to its destination. The building they stopped in front of seemed normal enough: it was a narrow structure of four storeys, with sun-bleached wooden stairs running up the side. She thought about having the Death Knight follow her up to the third floor where the man resided but the massive, armoured figure would probably scare the residents witless as it stomped up and down the stairs.

 

They made their way up and lightly knocked on the door. She heard movement inside and, after a few seconds, it opened by a sliver and a short man with a pale face looked up at her questioningly.

 

“Moren Boer?” Ludmila kept her voice even.

 

“Yes?”

 

His voice did not seem strange at all. The more she thought about his description as being ‘creepy’, however, the more she tried to find abnormal things about his rather normal appearance.

 

“You expressed interest in the offer of tenancy over in Warden’s Vale?”

 

“Oh! Yes.”

 

His voice picked up slightly, and the doorway opened fully. There was really nothing at all strange about the man; he was short and a bit pale, but could have appeared anywhere and not looked out of place.

 

“Are we leaving now?” He asked, seeming uncertain on how to address her.

 

It struck Ludmila how odd the fact was for a noble to be personally going around the city picking up new tenants. Migration was usually organized by the temples, guilds or the city offices. Even in those cases, they would simply receive permission to immigrate and head off on their own. She supposed that it couldn’t be helped with E-Rantel in its current state, but at some point she wouldn’t be able to afford the time to see to such things.

 

“We will be leaving before the evening,” Ludmila said, “so you have some time to prepare. There is a gazebo near the Royal Villa in the central district – that is where we will be gathering.”

 

“Gathering…?” Moren’s brow furrowed, “Ehm…it should be plenty of time – I’ll be there.”

 

With that, he softly closed the door. Ludmila felt the entire buildup of tension leading up to their meeting a colossal waste of energy. What was ‘creepy’ about him anyways? After returning to the wagon, she turned to her maid.

 

“I didn’t see anything wrong with him, did you?”

 

“No, my lady,” Aemilia shook her head. “He was quite rude to you, though. My apologies for not setting him straight – I was too caught up in the whole thing about him.”

 

“No harm done, I think,” Ludmila said. “I doubt he realized who he was speaking to…not that I could blame him for not expecting a noble to come straight to his door.”

 

With the potentially problematic farmer turning out to be rather normal, Ludmila felt that she could get through the remainder in short order.

 

The next three stops were all woodcutters with young families. Each had a roughly similar set of circumstances: while the Sorcerous Kingdom was supplying provisions to the people, the air of uncertainty still weighed heavily upon them. With the industries inside the city mostly at a standstill, even those that were willing could not find employment, so they had chosen to leave rather than continue to face an unknown future. They all shared the same look of relief on their faces when Ludmila explained why she had come to see them and were issued the same instructions that she had left with Moren Boer.

 

The smith was all the way out in the military district, which was originally built as a staging area for the Royal Army of Re-Estize during conflicts with the Empire. Outside of these periods, most of it lay empty aside from those who had good reason to be there – usually some business involving the cemetery that occupied the entire western quarter of the district. As they approached the address, they saw a middle-aged man, with streaks of grey through his barely kept brown hair, sitting on a tree stump tinkering with something. A strange sort of cart was parked nearby, and he looked up at the sound of the approaching wagon. Seeing the Soul Eater headed his way, he stood up and backed away warily.

 

Ludmila stepped down off of the wagon and made her way across the trampled grass. The location indicated by the address was actually a mustering field for training levied men. The soil was rough and uneven, and she could see the imprints of feet in the dried ground where men once stood to practice for their upcoming skirmish. As she came nearer to him she noticed a small tent pitched near the side of the field, under the shade of a nearby building.

 

“Ostrik Kovalev?” Ludmila read the name written on his form.

 

“That’s me…” His words came slowly, and his wary stance remained unchanged.

 

“You responded to the call for a smith in Warden’s Vale?”

 

“That’s right.”

 

He straightened his posture, put his work aside and pulled a cloth from his coveralls to wipe his hands.

 

Ludmila wondered if there was some unwritten law that required all smiths to be gruff and short spoken. She looked over the area around him again. Beyond the tent and the cart, he had dug a fire pit and a clothesline was strung between two poles that had formerly propped up practice targets. Several articles of clothing were being hung out to dry in the afternoon sun.

 

“Transportation will be available in the central district late this afternoon,” she told him. “Do you have a family you need to gather before you leave?”

 

“‘Course not,” he grunted. “A traveling smithy isn’t something you’d want to drag a family around with.”

 

Though people who traveled to various locales to perform their trade existed, Ludmila had never heard of a smith that actively travelled before. It would certainly explain his accommodations and, by its appearance, she guessed that the strange cart was actually some sort of compact forge.

 

“If you are a travelling smith,” she said, “then what are you doing here? The city has plenty of forges.”

 

“I followed the Royal Army into E-Rantel during the winter,” Ostrik explained, “that many men coming in at once and the city gets swamped with work, so my skills are at a premium. Plenty to be made doing even simple things like making horseshoes, nails and fixing odds and ends. Been at it since the Empire started the whole staring contest years ago.”

 

“But, you can perform the work of a regular smith, yes?”

 

“Hm? Of course.” He said offhandedly, “This whole setup might not make me look it, but I am a Master Smith. The Guild won’t lie to you about that.”

 

“Is it better to be a travelling smith than it would be running your own forge in a city somewhere?” Ludmila asked.

 

“Depends,” he answered. “Maybe if I was up north in Blumrush’s territory working Mithril…but then I wouldn’t get to travel. I made most of my coin doing work for the army for a few months, then I wandered about as I pleased for the rest of the year. Plenty to see and learn by watching how other different masters work in different places. It might not be the most stable thing, but I’m happy with what I gained by doing it.”

 

“So why didn’t you just leave E-Rantel after what happened?”

 

“Lost everything in the rout,” the smith made a sour expression. “Well, nearly everything. There was no warning – not enough to tie everything down, at any rate – only a wave of panicked men that bowled everything over. The press of bodies just swept me away from where I set up shop. When I finally got back, most of everything was gone – as scared as they looked, some bastards weren’t so scared that they wouldn’t stop and ransack whatever was in the way. Emptied my tent and flattened it; my horses disappeared too. Only reason why the forge was probably left behind is because it’s too damn heavy for a guy to run off with.”

 

“That sounds terrible.”

 

“Oh, it was terrible.”

 

“Then this will be a temporary arrangement then?” Ludmila asked, “Once you get back on your feet, you will return to your travels?”

 

“Nah,” he said bitterly, “losing years of your livelihood kind of takes the shine out of an adventure like this. I figured I would just work under some lord somewhere and that’d be it – saw the posting at the Merchant Guild and thought ‘why not’.”

 

“Well, if it is any consolation,” Ludmila said, “I would gladly have you at Warden’s Vale. I thought I would need to wait much longer before my demesne grew large enough to attract a Master Smith.”

 

“Your demesne?” He frowned, “Sorry, my lady, I’ve never seen a noble running around like this before; figured you for a higher-up in the lord’s retinue.”

 

“I had just the same thought when I started doing this today,” Ludmila accepted his apology in good humour. “I do not plan on making it a regular thing, but if something needs to be done, then it needs to be done.”

 

“Fair enough,” Ostrik said as he started to take down the clothesline. “Better than one of those highlords that can’t even be arsed to tie their own shoes–er, Pardon the language. Do you have a way to move the forge? I don’t see a way to hitch it to your wagon.”

 

“How much does it weigh?”

 

“It’s mostly empty of fuel and ore, so ‘round 300 Kilograms. It’s already packed up, so if you have a way…”

 

The iron gang ploughs that the Death Knights could carry were a bit less than double the weight of the forge but, since it was on a wheeled cart, she figured that a single Death Knight could easily pull it along. She motioned to her footman.

 

“You shouldn’t have any problems transporting that, yes?” She asked her footman.

 

It stomped forward, sheathing its flamberge. Lifting the tongue of the cart with its free hand, the Death Knight pulled it free of the shallow rut that secured it. After pulling the forge along for a few metres, it looked to Ludmila and nodded. She mused over how expressive it seemed; maybe it was just restless from doing nothing but following the wagon around all day.

 

Ostrik shook his head from where he was watching near the tent.

 

“Who needs a damn horse when you have legendary terrors working for you?” He muttered, “Are you from the Sorcerous Kingdom, my lady?”

 

“I am formerly a noble of Re-Estize,” Ludmila replied. “My title is a part of the duchy, so I went with it.”

 

“Well you sure got used to things real quick. Everyone else is still hiding under their beds.”

 

“It’s not as bad as it seems – not any more, anyways,” Ludmila told him. “The territories are already getting back to work, so the city will soon follow. In the meantime, I intend on pressing every advantage I have: all of my tenants will use the Undead for simple labour, if there is a place for them in their professions.”

 

“You’re kidding,” Ostrik looked at the Death Knight. “You mean this guy’ll work for me now?”

 

“No, that’s one of my footmen. I’ve requested a few skeletons that will work for you.”

 

“I’ve seen people using elementals for work,” he said, “but Undead would be sure to have the neighbors reporting you to the authorities. Well, I’m not sure what I can do with some Skeletons, but I guess we’ll just have to see. I’ll need a few minutes here to finish packing up, if you don’t mind.”

 

Ostrik turned away and continued to put away his belongings while Ludmila returned to the wagon with Aemilia. It wasn’t long before the smith tossed his bags over the side of the vehicle and pulled himself onto the back. They picked up one of the woodcutters and his family that they came across walking through the city on the way to the central district, dropping them all off at the gardens near the gazebo. Her new tenants gaped wordlessly at the hundreds of Undead arranged on the road nearby. Ludmila looked around until she spotted a cluster of four Skeletons standing apart from the others.

 

“Luzi,” she said, pointing them out, “take those four and bring them to Mrs. Ro’eh. The staff in the city will practice directing them and evaluate their usefulness while we are away.”

 

“Yes, my lady.”

 

Aemilia stepped off of the wagon, and the new tenants gathered nearby watched wide eyed as the maid took command of the Skeletons and led them to the side, awaiting Ludmila.

 

“We are done here – thank you,” Ludmila said to the Soul Eater. “You may return now.”

 

They made their way back to the manor as the wagon rumbled away, giving Terah a shock with her new Undead assistants. After Aemilia instructed Terah and the Linum sisters on how to control the Undead labourers, Ludmila spoke.

 

“I expect you to be well-versed in working with the Undead by the time I return,” she told them. “Any new hires should be instructed on their use as well.”