\chapter{Ludmila Zahradnik}

The sensation of the ship listing more than usual and Aemilia’s sudden, dismayed cry abruptly woke Ludmila from her sleep. Her eyes snapped open and she reached for the spear that she had placed on the deck nearby. Throwing open her covers and jumping to her feet, her gaze darted over her surroundings. She only saw the Undead crew who seemed to be carrying on as usual, however; there was no sign of attackers along the shores to either side or in the water flowing around them.

 

“Apologies, my lady!” Aemilia spoke off to her side somewhere fretfully, “I did not mean to wake you so rudely.”

 

Ludmila turned her head in the direction of her maid's voice: by the looks of it, she was preparing breakfast and the movement of the ship had caused one of the bags to tip over, spilling its contents across the deck. After discerning the reason why Aemilia had cried out, Ludmila loosened her grip on her spear. She bent over and picked up one of the blankets, wrapping it about her shoulders to shield herself from the ever-present northern winds. A thin mist covered the river and the lower slopes of the valley; it was not enough to become an immediate hazard in the broad and deep Katze River, but it gave the cold a lingering dampness that clung to one’s skin and clothing.

 

She looked up, scanning her surroundings to orient herself to their current position on the river. She was familiar with its course and did not expect any rough travel conditions until later that day. After she registered the dense forests filling the horizon ahead and the northern extent of the barrier ranges to the southeast, she realized that they had travelled much further than she had expected. Looking up to the perfectly trimmed sail and back to the Undead still tirelessly rowing since they had embarked the previous day, she immediately understood why they had travelled so quickly.

 

The villagers crewing the ship on her previous journeys did not constantly row the vessel upstream, instead relying mostly on the wind to carry them forward. The speed of a Human crew would also slow at night, partially out of caution, whereas the Undead who were reputed to have Darkvision could navigate regardless of lighting with no worries. The ship was also not carrying its regular load, which would have been the supplies purchased for the rest of the year.

 

After a wide bend in the river, they would be coming up on the place where the boat had been run aground and the river made its sharpest turns before heading into the gorge south of Warden’s Vale. Once they headed south up through the steep and narrow canyon, the winds would drive them quickly to their destination and they would most likely arrive some time in the evening – a full day ahead of schedule. She walked to the rear of the vessel where the captain kept its hand on the tiller, and warned of the oncoming course of the river. It acknowledged her words and she returned to the hold, where Aemilia had just finished cleaning up the mess over the deck and resumed putting together breakfast.

 

“Were you able to sleep well, Aemilia?” She asked, “It must be a new experience for you to sail on a ship: some people have trouble getting used to it.”

 

“Yes, my lady, thank you.” Aemilia replied as she handed over Ludmila’s portion, “I slept well. It’s very cold this morning, though. If we hadn’t stopped to pick up that extra clothing at Corelyn Village, I’m not sure what I’d have done.”

 

“It gets colder as we travel upriver. It may have snowed in the past few days, considering the rain we’ve had in E-Rantel.”

 

“Snow, my lady?” Aemilia said, “That’s a rare sight in the city. I’ve probably only seen snow a half dozen times in my life, and it never stays on the ground like you see in the mountains to the north. Still, I’m very excited to see what sort of place Warden’s Vale is.”

 

The expression of child-like wonder in her tone made Ludmila wonder in turn about what sort of grand spectacle her excited maid was anticipating. It compelled her to try to lower the impossibly high bar at which Aemilia had seemingly set her expectations.

 

“Warden’s Vale is a beautiful place, but it’s still a frontier territory,” she stepped lightly with her words. “It’s nowhere near as developed as the interior regions of the duchy.”

 

“I see…” Aemilia replied, but the energy in her voice was undiminished, “but since you defend the borderlands, there should at least be a castle, yes?”

 

A Castle?

 

The words stabbed at Ludmila like a spear. Did the other, wealthier nobles even have castles? House Corelyn was absurdly wealthy, and even they did not have a castle or anything resembling one. Certainly, there were walled cities as well as fortifications watching over a few key strategic points in Re-Estize – the city of E-Rantel was an example of both – but the vast gap between reality and Aemilia’s expectations made Ludmila feel like she had already betrayed her maid somehow.

 

“There’s no castle,” Ludmila said, “but the village itself is built on a series of ramparts for defence.”

 

“Then...your manor and estate must be quite large with all the free land?”

 

Ludmila opened her mouth, then closed it again. She wasn’t even sure how a large estate and manor could fit in a village built on stony ramparts. Aemilia really did seem to have a whimsical image of the place where she would be arriving, not even understanding this much.

 

“My manor is in the village; it’s just a bit larger than a villager’s home there. The entire settlement is defensive in nature, so there are no vast estates or ornamental gardens. It’s just a village that’s only a fraction of the size of Corelyn Village…and even a wealthy Barony like that does not have a castle, or vast estates. Lady Corelyn does have a comfortable manor, though.”

 

“B-but the wilderness is teeming with savage Demihumans,” Aemilia said in disbelief at her mistress’ words. “With so few, how did you keep them at bay for so many generations?”

 

“Through constant vigilance,” Ludmila replied. “Every villager was trained to fight, and everyone participated in the patrols – including the members of House Zahradnik. I’m not sure what stories you’ve heard, but it isn’t as if all of the Demihumans in the wilderness are in that one small corner bordering our territory. And they do understand the concept of territory – a healthy respect of one’s neighbors is something that they appreciate, even if their neighbors are Humans. As long as you’re not considered an easy target for their predations, and don’t agitate them for senseless reasons, they’ll look elsewhere for more likely prey.”

 

“Then you’re not always fighting?” Aemilia asked.

 

“Not usually,” Ludmila answered. “In times of plenty, the Demihumans don’t need to range very far from their own territories to find food. The only instances where we usually see conflict is when there is famine or disaster in their lands, or new tribes migrate into the region and push out the old ones. When that happens, the balance of power is disrupted and so are the territories of the various Demihuman populations, so we must ensure that their respect for our territory is enforced. Stronger ones also appear to try and make a name for themselves by accomplishing some feat of strength, but it’s not something that happens very frequently.”

 

Aemilia had become fixated on her words, sitting on the edge of the deck overlooking the hold.

 

“How strong can these Demihumans become?”

Ludmila seated herself beside her, looking towards the mountains that loomed over the valley.

 

“They can be very strong,” she told her. “Many Demihumans have natural advantages that we don’t, so trying to compare them directly to a soldier or an Adventurer can be misleading. The last time such a Demihuman appeared in the fashion I described above was when I was still a child. The patrol that ran into him and his companions filled them full of arrows, killing most of his group as they charged their position. Even so, he still made it to their lines and killed a third of the patrol before he was finally put down with a combined effort. A Demihuman like that would be roughly as strong as an Orichalcum-rank Adventurer.”

 

Ludmila kept a straight face, even as she left out the fact that it was on this very patrol that her own mother had been killed, lest she worry Aemilia overly much. As the strongest patrol leader in the village, the Demihuman had marked her mother out as a worthy adversary to prove his own prowess as a warrior. It was an unfortunate fact of life on the frontier: the strong would seek the strong, in a never ending back and forth between all the races that lived there. If her mother had not been present, the entire patrol would have been wiped out and Warden’s Vale soon after – that was just how precarious their situation had become. Fortunately, her mother’s death had bought nearly a decade free from strong challengers: the latest in generations of sacrifice that enforced their status as a rival rather than a raid target.

 

“Are frontiersmen really that powerful?” Aemilia’s eyes grew wide, “I’ve heard stories, of course, but you make it seem like even Farmers on the border are as strong as Adventurers.”

 

Ludmila paused to think of a way to answer her question. The lives of city folk were so far removed from her own that she found herself taking many things for granted, even when she thought she was answering her maid’s questions thoroughly.

 

“Several years ago,” Ludmila said, “the wife of Count Jezne brought an Adventurer Bard that had earned her patronage to a luncheon for the noble ladies of the territory during the winter markets. One of the tales he sang revolved around a party of brave young Adventurers being set upon by a tribe of Goblins. Most in attendance were aghast when he described the Goblins’ dishonourable and underhanded tactics in battle, but I thought they seemed perfectly reasonable.”

 

“...are you saying that you fight like a Goblin, my lady?” Aemilia looked like she had been dunked headfirst into the river.

 

“Perhaps,” Ludmila smiled at the memory of the rather colourful story. “I’m not sure just how much that Bard embellished his tale. But I can tell you that in a life and death struggle, having a fair fight is the last thing on most people’s minds.”

 

She reached into a pocket inside her kirtle, pulling out a ceramic jar that fit into the palm of her hand. Unscrewing the cap, she displayed the grey paste inside, which gave off a light, spicy odour.

 

“This is a type of poison,” she said and Aemilia, who had been inching her face closer to the jar, suddenly pulled back upon hearing her words. “The jar contains enough for around a hundred arrows. It’s potent enough to probably paralyze even an Adamantite-rank Adventurer, unless they have items or magic to counteract it. Demihuman tribes usually do not have such items, so they rely on their own casters in the event that they are poisoned. A single frontier patrol, however, can unleash so many of these arrows in a short span of time that even several such casters have no hope of keeping up.”

 

Ludmila put the jar away and continued speaking.

 

“We use poison and traps; misdirection and ambushes: there is no limit to unfair methods. If a Demihuman reaches our lines, a half dozen fighters with spears will work in concert to take it down. If it can be harmed, then we’ll devise the most efficient ways to kill it. The entire purpose of a frontier territory is to clearly define where the wilderness ends and where our lands begin. With our limited numbers we must do so in the most cost effective manner that can be achieved, or we’ll inevitably be overrun.”

 

Aemilia’s look of childlike wonder had been mostly wiped away, replaced by something between a worried and resolute face.

 

“But…what about all the stories about honour and chivalry?” She said tentatively, “The old tales of gallantry and heroism that are sung of the nobles and knights that defend the realm?”

 

“I’d say those are fanciful tales spun to entertain the masses,” Ludmila replied plainly. “A noble does not prioritize a warrior’s honour, and it has no place when measured against duty. A noble’s duty is defined by the contracts between liege and vassal. Duty is a noble’s honour; fealty and obligation determine what is chivalrous. No amount of personal pride as a warrior will excuse negligence and failure in one’s duties. You must keep this in mind at all times as well, Aemilia: you are part of House Zahradnik now. As a member of a noble retinue, you must understand the difference between fantasy and reality.”

 

“Yes, my lady,” Aemilia lowered her head. “I will not fail you.”

 

Aemilia offered no more questions after that, returning to her work. The change from her former expression had Ludmila worrying about whether she had squashed her expectations a bit too ruthlessly. She was always chided by her family for getting too wrapped up in what she was saying and being too blunt with her words; it seemed like she might never grow out of it.

 

Over the next hour, the captain successfully made the turns in the river without incident and the Undead crew had navigated through the most hazardous section along the middle reaches of the Katze River. With their course now heading south, the sails fully caught the north wind and their journey picked up speed. Before noon, they were entering the long gorge that would end at Warden’s Vale. Aemilia gaped as they passed between the towering granite walls, nearly bare of vegetation. There were no paths or even shores; the cliffs, which kept rising higher to either side as they sailed deeper into the foothills, dropped straight into the river – giving no purchase for travelers on land. Small streams would tumble down from the heights every few kilometres, creating breathtaking waterfalls which churned the waters below them.

 

Aemilia had renewed some of her previous excitement with the unfamiliar scenery, but Ludmila saw the familiar passage as an opportunity to relax. The gorge was actually the safest stretch of the river, so deep that it was mostly unassailable by threats from land. By late afternoon, they had entered the borders of the barony.

 

The clouds that had rolled down from E-Rantel had long since passed, leaving trailing wisps in the cerulean skies. To the south, the peaks of the border ranges beyond the valley could be seen through the crisp air, still crowned in white. Sunlight glistened off of the melt running down the cliffs of the eastern shore and the river swelled from the advent of spring. Ludmila spotted patches of snow in the shadows of the high valleys that spilled into the river, so it seemed like her thoughts from the past few days about the weather were correct.

 

Beside her, Aemilia pulled a handkerchief out of her pocket and wiped away her tears.

 

Ludmila felt a twinge of guilt. She had taken her maid, who had been raised in the thriving city of E-Rantel, to the furthest borders of the realm. There were no bustling markets or magical conveniences and land was raw and undeveloped. There weren’t even people here any more, aside from themselves. She raised a hand and placed her arm over her maid’s shoulder.

 

“Please don’t cry, Aemilia,” she said in a comforting tone. “Things will surely get better in time.”

 

Her maid stiffened at her words. She sniffed, wiped her eyes, then sniffed again. Then she let out a shuddering laugh.


“I don’t doubt that they will, my lady,” she smiled. “That is not the reason for my tears, however. You said your home was beautiful…but that was just a tiny bit of an understatement.”