\chapter{Ludmila Zahradnik}

All of the tenants had gathered by the time Ludmila returned to the gardens from the manor. Nearly all were still gathered a distance away from the formations of Undead labourers. She walked up to stand in front of them and their eyes turned to her. There were sixteen in total: the smith and the farmer; the three woodcutters with their wives and children.

 

“As you may have noticed,” Ludmila addressed them, “Warden’s Vale will have plenty of the Undead around. The offer you responded to should have said as much but, if you are having second thoughts, you may withdraw – I have no need for tenants who cannot adapt to life on the frontier and embrace the new ways of the Sorcerous Kingdom.”

 

She wondered what she would do if they all just turned around and left. Fortunately, after a few minutes, none of them did. Their focus turned away from her, however, as Lady Shalltear arrived and riveted all of their gazes to her enthralling beauty. Ludmila came forward to pay her respects to her liege.

 

“I hope the evening finds you well, Lady Shalltear,” she greeted her.

 

“Are you all set here?” Lady Shalltear asked.

 

“We are all assembled, my lady,” Ludmila nodded. “This should have the demesne well on its way.”

 

“Good. Maybe Albedo will finally stop trying to have us join her in her boredom with some numbers to crunch.”

 

Lady Shalltear walked a short distance to the gazebo and opened Gate. The column of Undead lined up and marched in. When the last Death Knight pulled the portable forge through, Ludmila motioned for her new tenants to enter. Ostrik stopped just before the Gate before leaping through for some reason – the others similarly stopped to stare at the black portal hanging before them before hesitantly stepping into its threshold.

 

“Is there any limit to this magic, my lady?” Ludmila asked as the last of the villagers filtered through.

 

“It drains my mana as long as it continues to be used,” Lady Shalltear yawned as the long procession finally ended. “Just this much is not a problem, though – there are also ways to extend the duration with some assistance.”

 

“Do we even need roads if we can always rely on this magic?” Ludmila wondered.

 

“It’s certainly useful if you need to deliver something or transport large groups quickly,” Lady Shalltear told her, “but it’s not necessary for transporting regular cargo. A warehouse or port can only accommodate a certain amount of traffic at any given time; it doesn’t matter if it reaches that capacity through a constant stream of wagons or ships, or through magical transportation. In that sense, it’s an inefficient use of mana – and a liability should I require that mana for something else.”

 

“I see. You are quite knowledgeable about these things, my lady.”

 

“I suppose that’s why I’m the Minister of Transportation. I would still like for His Majesty to find more for me to do, though.” Lady Shalltear looked to the western wall, where the sun was beginning to dip below the massive barrier, “It’s about time for yet another meeting, you should get going.”

 

“Thank you once again for your kindness, my lady,” Ludmila curtsied deeply, “I hope to repay your favour soon.”

 

“Like I said, all you need to do is–”

 

Lady Shalltear’s voice cut off. A moment passed before she spoke again.

 

“Work hard and produce results that will help me gain the regard of His Majesty.”

 

Her liege’s words carried a hint of ambition that had not been there before, and Ludmila rose with a bright smile.

 

“Of course, my lady,” she said. “I am at your service.”

 

After stepping foot on the sandy flats below the entrance to the village, Ludmila looked around as the Gate closed behind her. The Undead had arranged themselves neatly once again while the villagers stood together on the path between the village and the pier. Nonna floated down from the west, and the new migrants shifted away as one.

 

“How are things out on the fields?” Ludmila asked as the Elder Lich settled down in front of her.

 

“The work continues,” Nonna replied. “Inventories for spare tools have fallen by 15 percent.”

 

Ludmila winced.

 

“How are we losing so many tools?” She asked, “I don’t recall losses like this from previous seasons.”

 

“Your statement appears to reflect the data collected from Carne,” Nonna nodded. “I have come up with several theories as to why this is.”

 

“We can discuss them when we’re out there tomorrow,” Ludmila told him. “I need to make sure the tenants are settled before evening falls. We have a smith now to help with repairs, but we need a long term solution that doesn’t constantly tie up skilled labour. I’ll need him for other things soon enough.”

 

“I concur,” Nonna nodded. “Delays and resources consumed by substandard equipment produces suboptimal results.”

 

“Do you have any ideas offhand?”

 

“Since a smith is present,” the Elder Lich stated, “the option of upgrading the equipment may be possible. Orichalcum or Adamantite should suffice.”

 

Behind them somewhere, Ostrik made a choking sound.

 

“That’s a joke, right?” Ludmila asked.

 

Nonna did not provide an answer. How much would a single Adamantite shovel even cost? Was there even enough in the duchy for all of that equipment? Definitely not.

 

“Have there been any more accidents out in the fields?”

 

“No.”

 

“Any new incidents in the village?”

 

“No.”

 

“Any sign of trespassers?”

 

“No.”

 

Ludmila made a face. She knew it was just the way that Nonna was, but she still preferred a more verbose rapport with her subordinates.

 

“One more thing before you get back to your work,” Ludmila motioned for the Elder Lich to accompany her as she walked over the new arrivals.

 

“This is Nonna,” Ludmila said to the villagers after they had turned their attention to Nonna – or rather, they wouldn’t stop staring at her, “if you have any issues with the Undead, please let her know.”

 

The line of faces all paled in unison when they heard her words.

 

“I meant issues you encounter as you work with the Undead,” Ludmila reiterated quickly. “Any surprising uses as well. Nonna is evaluating their work around the fief. What you learn here will be used all around the duchy, so it’s something of importance to the entire Sorcerous Kingdom.”

 

The villagers relaxed somewhat after she waved the Nonna away. She led the group up the road and into the settlement.

 

“You may choose any of the homes here to settle in with your families. Miss Luzi here has cleaned most of them up,” Ludmila looked towards Aemilia, who nodded. “There is a shrine at the top of the hill to offer your devotions and prayers. Conveniences such as magical faucets do not currently exist here, so water is drawn from the river. If you feel that you are ready, Miss Luzi can also teach you how to direct the Skeleton labourers to help you with various things. Shares of food for the week will be distributed once you’ve all moved in.”

 

As they dispersed, Ludmila called out to the Farmer.

 

“Moren Boer.”

 

“Yes, Baroness?”

 

“You’ll be moving into the hamlet out in the fields tomorrow morning, so don’t get too attached here.”

 

“...by myself, my lady?” He asked.

 

“There will be nearly 600 Undead labourers around you as well,” Ludmila told him. “I hope you don’t mind.”

 

“Of course, my lady.”

 

That the short, pale man had so smoothly accepted her words was the first thing Ludmila noted was odd. She couldn’t imagine a regular farm tenant being able to accept such an arrangement so quickly. She watched him as he disappeared into a home nearby, but she did not notice anything amiss.

 

Ostrik appeared from the home furthest to the west and walked back down towards her.

 

“That was quick,” Ludmila noted as he approached.

 

“I’m used to it,” he said. “Wanted to set things up while the light was still good.”

 

They walked back down to where the portable forge was parked on the flats. She issued orders to the new Undead farming teams and watched them march away before turning back to the smith.

 

“Do you know where you want to place your forge?”

 

“Yeah,” Ostrik replied. “Below the place I moved into. The wind comes in from the north, so working there won’t blow fumes over the village.”

 

Appreciative of such a consideration, Ludmila called the Death Knight pulling the portable forge over to follow them, as well as four Skeletons. The smith went ahead of them, continuing to speak as they walked.

 

“This is quite the place you got here, Lady Zahradnik,” he kept looking around at their surroundings. “Most of the duchy is farmlands and forests. I didn’t expect a beautiful place nestled at the base of the mountains like this. Wouldn’t mind finding a wife and settling down here.”

 

“Warden’s Vale was chosen for its natural beauty,” she replied. “Plus it’s defensible. We don’t have much else though.”

 

“I’ve seen a lot of places,” the smith told her. “This territory is very nicely balanced. As long as you have people here, you’ll be set.”

 

They stopped at the westernmost point of the lowest village terrace, overlooking the fork in the road which branched west into the fields across the wooden bridge. Ludmila crossed her arms against the wind as she watched Ostrik unpack his forge. Opening the locks, he lifted the hood which covered the cart.

 

The steel wagon was divided into two sections. One side opened and was pulled out to create a small extension, where he unpacked and placed the two bellows to point into an opening to the other side. The other section appeared to be an enclosed furnace with a bed for fuel and a window which faced them. There was a wood stump, which he rolled out and placed his anvil on. Lastly, he pulled his tools out from inside the furnace with a series of metal hooks, which he hung on the edge of the forge to arrange his equipment on.

 

“I’ve never seen something like this before,” Ludmila said as Ostrik drove pegs into the ground that were attached to chains securing the forge. “What can you do?”

 

“Any sort of smaller scale metalworking,” the smith said as he circled his forge. “I can even repair armour and make weapons using this. I’ll be getting things ready to build a real smithy on this part of the terrace here when there isn’t any work to do. Could make a few things if you need ‘em as well, but first I need to start some mounds for charcoal and put together a bloomery. You have materials for everything right here in this valley, so all we need is time and labour – which you seem to have plenty of with all these Undead here. If I can figure them out, that is.”

 

“Won’t you need iron ore?” Ludmila asked.

 

Ostrik looked up from his inspection and peered at her strangely.

 

“Was there a smith here before?”

 

“Not that I know of.”

 

“Then you probably have plenty of ore to work with for now.”

 

Ludmila frowned in confusion. Seeing this, Ostrik continued speaking.

 

“I’ll show you after I’m done here,” he said. “Need to get together with the woodcutters first if we’re going to get started.”

 

He returned to his work and seemed to have nothing else to say, so Ludmila turned back to walk down the lane. She looked up and saw several villagers moving about and, as she turned to follow the path up to the next terrace, she came across Aemilia. Her maid was followed by the village women, as well as three girls.

 

“My Lady,” she nodded: the maid and those following her were all carrying several empty buckets each. “We were just headed down to draw water from the river. There are a few of the men waiting at the warehouse for you.”

 

The group continued down the way, stopping in front of the remaining Undead that had not been deployed to the fields. Aemilia spoke at length, during which several of the women backed away but, when she finished, the children immediately dashed forward. They handed out their buckets to the awaiting Skeletons and led them off towards the riverbank. Seeing this, the adults hesitantly came forward and led away their own. Ludmila wondered whether the girls had been thrilled at the prospect of having the tiresome work handled by the Undead, or maybe it was the idea that they now had underlings to order around. She initially thought that it would take days for her to convince anyone to start employing the Skeletons, but it seemed that the children were far more flexible than their parents.

 

The men of the village were waiting in front of the warehouse, accompanied by their sons. They stopped speaking between themselves as they noticed her approach, turning to bow awkwardly.

 

“All settled?” Ludmila asked.

 

“Yes. Thank you, Lady Zahradnik,” the men exchanged glances before one of them spoke. “The construction of this village is…different, but the homes are clean and warm and there is plenty of room to expand. When we were gathered in the city, I thought all those Undead you brought would be everywhere, but I can barely see any of them. It’s better than the city – those huge, armoured monsters are everywhere: standing guard and constantly patrolling the streets.”

 

“This place has more than its fair share of Death Knights,” she explained as she unlocked the warehouse. “All of the Undead will be put to work here – several of them will be assigned to assist you in yours as well.”

 

“...is that safe, my lady?” He spoke tentatively, “The Unde–”

 

“You may consider the Undead to be your fellow labourers here,” Ludmila told them. “As was said earlier, I intend for you to become pioneers in what is surely a new reality for everyone here – the same reality that the entire duchy will eventually need to face. You had best become used to it, and apply yourselves well. Your wives and daughters are already discovering how convenient they are; I doubt you will be able to part them from their newfound helpers once they have.”

 

As if on cue, the line of bucket-wielding Skeletons came up the lane with the group that had come down a short while before, disappearing into their respective homes.

 

Ludmila entered the warehouse and began to distribute supplies to the villagers. There was only enough for Aemilia and herself for the remainder of the season, which would only last a week between all the new immigrants. She had no worries about the apparent lack of supply, however: the natural abundance around them would amply feed her people with plenty to spare until more regular sources of food were established.

 

The men returned to their homes with their provisions and Ludmila entered the warehouse again, stretching as she looked around the nearly empty building. Her gaze eventually fell to the corner where she had placed Jeeves’ box. It, too, had disappeared as if it had been inextricably connected to the existence of the Undead ‘butler’. She felt a pang as she continued to stare at the empty corner. He had been so upbeat about his future duties and then his existence had been callously extinguished for being in the wrong place at the wrong time for the wrong reasons.

 

A shadow fell across the doorway and Ludmila looked up. Nonna had returned, taking her place beside where Jeeves’ box once rested. The warehouse had become a base of operations of sorts for the Elder Lich, and the tome she always carried opened in her withered hand. The scratching of pen on paper went on as she jotted down notes in a strange language that Ludmila had never seen before. She assumed it was the information that Nonna had been sent to collect and assess in her tireless quest for efficiency.

 

“Do you miss Jeeves at all?” Ludmila asked the Elder Lich.

 

The sound of writing continued; Nonna did not care to answer. Ludmila pressed on however, determined to find out once and for all how the Undead felt about the loss of one of their own.

 

“He was Undead as well, wasn’t he?” Ludmila said, “He was so happy with his work here. Don’t you think what happened was terrible?”

 

Nonna’s pen continued across the paper, but her voice rose up to answer Ludmila’s question as the Elder Lich continued her work.

 

“There are fates that are worse than the one that he met; there are worse fates than death.”

 

“What could be worse than death?” Ludmila scowled a bit, “As long as you continue to exist, you have the future to look forward to and new experiences and opportunities will inevitably present themselves.”

 

The Elder Lich stopped writing and looked up at her. Nonna’s glowing crimson gaze examined Ludmila as they stood across from one another in the shadows of the warehouse.

 

“To be abandoned by your Masters,” Nonna finally said. “Left without explanation or even a single word. You will never know why they left, or where they even went and you will never see them again. Only left to wonder why – was it something you did? Did they tire of you? Were you inadequate in some way? You will stand alone: without guidance or purpose, as you watch the endless eons ebb and flow until – even as you have pondered these same thoughts for an eternity – existence itself crumbles into dust.”

 

Ludmila stood aghast at Nonna’s words. It was an incomprehensible fate, yet the words twisted into her soul like an icy blade spreading cold tendrils of helpless despair throughout her being.

 

“There are worse fates than death,” Nonna repeated herself. “Jeeves met his end in the service of his master: it is an enviable thing.”