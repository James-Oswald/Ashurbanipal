\chapter{Ludmila Zahradnik}

Despite her words, Bohdan had not lost any of his momentum and the priest continued to press his case. Seeing this, Ludmila motioned for him to follow, and together they walked back up the village lane to speak inside the manor hall. Pulling off her scarf, she went to the kitchen area at the back of her family’s home to bring out two wooden cups filled with cool water. Ludmila moistened her throat before picking up where they had left off.

 

“You understand why I consider your plan unacceptable, yes?”

 

“Mistress Ludmila,” Bohdan smiled reassuringly at her worries, “I understand the root of your concerns, but I think you are underestimating the people. Perhaps you’ve been influenced by other nobles on these matters, but surely you understand that frontiersmen are not the same as the folk of the interior.”

 

The wise priest was right, as usual. When she was still a child, she had gauged everyone by the same measure, believing everyone was the same in regards to what they were broadly capable of. As the years went by, however, she came to understand that this was simply not true. Because the Baron only brought his family to E-Rantel once or twice a year, the difference in physical ability displayed between visits was plainly apparent. Through friendly sparring and training between them, it became clear that the gap in ability between her brothers and the other noble scions of the duchy only grew wider with every season. By the time they approached marriageable age, they were looked up to as reliable cornerstones in matters of martial prowess.

 

Even amongst the noblewomen, she could feel through simple interactions that she was a bit stronger, faster and more resilient than the other ladies of the ducal nobility. She wasn’t sure if it was something to be particularly proud about, but it was a fact all the same. Since she could now often keep up with her brothers in their training and participated in wilderness patrols with few issues; Ludmila suspected that she would also be able to best the young noblemen of the ducal court in combat as well.

 

When inevitably approached by his children as to why this was, their father said it was due to a difference in heritage. The settlers of the borderlands were often descended from Adventurers, and the original Frontier Nobles were powerful Adventurers that had claimed the lands as their own by expanding and taming the borders of Re-Estize. Through maintaining these bloodlines and honing their strength against the dangers of the vast wilderness beyond, they were always notably superior to their peers in the more peaceful regions of the Kingdom – at least when it came to martial matters. Ludmila had at least one close friend from the inland territories who could fly circles around her in other subjects.

 

The common people of their village were much the same, to a certain extent. In addition to being farmers, woodsmen, hunters and tradesmen, they were also Fighters and Rangers. By necessity, everyone was trained from a young age and involved in patrolling the border and defending the fief from the myriad threats that faced them. This militant culture resulted in a population that was generally stronger in combat than the citizens that lived in the interior, who focused primarily on their domestic tasks. Small Demihuman tribes could not muster a force to intercept a caravan full of frontiersmen without a good chance of being instead destroyed themselves.

 

They may have looked the same as every other Human, but as Bohdan had reminded her, they were not the same as those from the Kingdom’s heartlands. Despite this, Ludmila still had other doubts. Bohdan, however, pushed forward.

 

“Besides, what other choice have we?” He said, “We will not outrun the Empire’s forces if we flee west, and certain death lies to the north at E-Rantel. We cannot cross the river eastwards and scale the sheer cliffs on the other side to reach the main highway between E-Rantel and the Theocracy, so we can only go south.”

 

Ludmila bit her lip. Truthfully, there seemed to be no good options. Instead, she reached for a compromise.

 

“If the capitulation of the Kingdom is the objective of the Empire, then they should be moving westwards towards the Royal Capital to take advantage of their momentum and secure a quick victory. There’s a good chance that they would ignore an out-of-the-way border territory like ours. We can put together a raft and set a watch down the river closer to the city. If they dispatch forces south, the Rangers we send can sail back up the river faster than any soldiers can travel by land. If the Empire heads to the Kingdom’s heartlands, they can report back as well to give us some peace of mind. We can wait for the terms of settlement without having our spring activities interrupted.”

 

It was a proposal she was confident that the Baron would have approved. The frontiers of Re-Estize were sparsely populated and poorly developed; they presented next to no major military threat, especially when the levy had already been raised and there was relatively little remaining in terms of manpower that could be mobilized. Despite the loss at Katze, the vast and wealthy interior and coastal territories in the west still had upwards to four million adult men.

 

The Empire would have to strike quickly at the heart of the Kingdom after successfully completing the siege of E-Rantel before a response could be mustered, or face a number of challenges. The personal retinues of the great nobles and various mercenary companies would become involved as well, representing an even greater threat. Such an attack was considered an impossibility under normal circumstances: while the Empire’s Legions were a standing army of well-trained soldiers, they had nowhere near the numbers to occupy all of the territories that lay between. Even a successful siege of E-Rantel was questionable, as relief would presumably arrive before the walls could be breached.

 

In her mind, given the Empire’s tendency to preserve the strength of their Legions, their most reasonable course at this point was to pursue concessions after quickly scoring a series decisive victories. Once terms had been settled, the territories in question would be sorted out in a diplomatic manner. In other words, rather than pointlessly taking such a monumental risk in crossing the wilderness before the shadow of an unrealized fear, she thought that they should only evacuate the villagers when it was proven to be a threat.

 

Bohdan remained stubbornly unconvinced, however. His wisps of white hair waved lazily about as he shook his head vehemently.

 

“The legions of the damned care not for the customs of the living,” he said. “They will hunt us with an untiring, unrelenting hatred no matter how many or how few we may be. The sooner we leave, the greater our chances of avoiding a terrible fate at their hands.”

 

“The Legions?” Ludmila raised an eyebrow, “Though admittedly powerful, Milivoj only saw one Undead being riding on those beasts. How did we arrive at Legions?”

 

“Milivoj saw only one, yes,” Bohdan said. “But that doesn't mean there aren’t more – there are always more. The fact that the Imperial Legions were so far away and did not advance during the battle is already suspect. Perhaps they wished to conceal their true nature so as to not alert the population at large. A single feat of legendary might is far removed from the day to day reality of the people, but stories of being destroyed by endless waves of the Undead would definitely warn them away far in advance of their arrival.”

 

Ludmila took a moment to digest his words. In addition to the national catastrophe of the army being destroyed, the mounting irregularities put events beyond her scope of knowledge and experience as the Baron’s proxy – she was barely piecing things together in her mind as they went. When Lord Zahradnik left with his men less than a month ago, she had been expected as the remaining member of the House to keep the daily affairs of the fief in order and at most deal with minor issues here and there.

 

Not in her wildest imaginations did she expect that the regular routines of the season would come to this. Never mind navigating a safe route through the fiasco, she still struggled to even develop a clear picture of their situation.

 

“E-Rantel sits at a major trade hub between Baharuth, Re-Estize and the Slane Theocracy,” she noted. “News from the Empire of such a thing would have travelled far in advance by way of merchants and wayfarers…never mind that: refugees would be swarming over the borders. Nations of millions do not simply fall to the Undead overnight.”

 

“Ah, but that is where you are wrong, Mistress Ludmila.”

 

Bohdan immediately jumped onto the tail of her statement, raising a gnarled finger beside his head.

 

“Excuse me?” Ludmila blinked, “What are you talking about?”

 

She could not keep the puzzled expression off of her face. Noble children were usually taught the history of the region to one degree or another and she definitely did not recall anything like this being a part of it. Bohdan served in the role of tutor to House Zahradnik in several subjects, so her confusion was clearly evident to the priest.

 

“It is a tale from before the history of Re-Estize – from so long ago that I did not consider it essential for a young noble’s education in these times.”

 

He bowed his aged head in apology, though Ludmila also thought it was a reasonable decision. Bohdan settled into a more relaxed posture, putting his hands into his sleeves in front of him.

 

“During my training as an Acolyte in the Theocracy,” he told her, “before I came to be a missionary here, it was something that was taught. Keep in mind that this is from a century ago; the events and memory of the Thirteen Heroes were still very much a recent thing in the hearts and minds of the people.

 

Before the chaos that came with the advent of the Demon Gods, a group of faraway nations lost contact with their neighbors all at once. The investigations from the surrounding countries that were sent shortly after all reported the same dreadful findings: each kingdom, to a man, had been transformed somehow into Undead. Kingdoms much like our own, inhabited by millions, turned into a necropolis before anyone realized what had occurred. It is said that the dreaded Vampire Lord, Landfall, was also discovered there. Many of the more...ardent...adherents of Surshana amongst my peers firmly believed that the events were connected: a great ritual cast by an evil, ambitious individual to obtain immortality as one of the Undead.”

 

It was a tale told in tones reminiscent of better times; when Ludmila and her brothers sat in the hall of the manor listening to the old priest relate the histories of the world. A part of her took comfort in the familiar feeling, but her mind also worked to continue to stitch together what her tutor had shared with her own working knowledge.

 

While she had not specifically known of the event that Bohdan had just disclosed, tales surrounding the Thirteen Heroes spun by the bards and minstrels of the realm were still quite widespread and well received. She had heard the story of Landfall several times from different performers while visiting E-Rantel over the years. In one version, he was an unspeakably evil noble, whose atrocious actions against even his own people eventually led to his rise as a Vampire. In another, he was an ancient horror that lurked in the shadows of cities and villages, preying on the people indiscriminately to quench his insatiable thirst for blood.

 

In both tales, the Vampire Lord had brought about the ruin of entire nations through his actions, earning him the moniker ‘Landfall’. It was not until the Thirteen Heroes confronted him that he was defeated and his reign of terror was brought to an end. There was a third version of the tale that suggested that he still lurked somewhere amongst the people, continuing to prey on the young women drawn helplessly by his charms, and that history would once again repeat itself some day.

 

This last version was somehow the most popular amongst the women of the court. For her part, Ludmila could not understand how anyone could feel that way about a creature that considered them food. By and large, however, the tale of Landfall was still an enthralling one: valiant heroes overcoming their vile adversary. Now that it was being applied directly to their own situation, however, it went from fantastical to nightmarish.

 

“Do you mean to say that this Fluder Paradyne has performed such a ritual to become an…Elder Lich?” Ludmila frowned as she voiced the unfamiliar term. “And that the Empire has similarly fallen to the Undead?”

 

Bohdan nodded in affirmation.

 

“The precedent exists,” he said, “and based on Milivoj’s account, I find it not unlikely that this is what has occurred.”

 

This time, unlike the tales of the past, there were no legendary heroes to save them. Apparently, the only legendary figure of their time was the subject in question. The irony would have probably made her laugh helplessly if she wasn’t preoccupied with the survival of the demesne and it’s people. With a sigh, she finally relented.

 

“Very well,” she said. “I will organize the evacuation. Please let the people know.”

 

Bohdan nodded and bowed in gratitude.

 

“Thank you, Mistress Ludmila. You have made the right choice.”