\chapter{Ludmila Zahradnik}

The strip came off cleanly, leaving no marks on the box. The box itself showed no signs of doing anything – though why she thought it might, she had no idea. Ludmila leaned forward to flip open the cover. At first the box appeared to be empty, but as she poked her head over the edge of the opening, she saw that the bottom of the container was littered with bones. Atop the layer of bones was a single, bleached, skull whose crimson eyes flared to life even as she looked down at it.

 

Startled, Ludmila pulled back and ended up with her rear on the deck. There was a brief series of scraping sounds, followed by the noise of bones clattering against each other as a Skeleton rose and formed itself. She felt a tugging at her elbow as Aemilia grasped her sleeve and drew close in reaction to the Undead figure rising before them. When it stood fully upright, the skeleton crawled out of the box and onto the deck. It was roughly as tall as her waist, leaving much to be desired when it came to the entire string of events that led up to its appearance. Several accessories came into existence on its form: a collared and pleated white bib adorned with a black tie and buttons hung off of its neck; cuffs and links to match appeared over its wrists. A clean, white cloth was draped over its left arm and a golden monocle fastened with a delicate golden chain materialized over one of its eye sockets.

 

“Good day, madam!” It bowed slightly as the crystal lens of its monocle gleamed in the afternoon sun, “How may I be of assistance?”

 

Aemilia inched away as it suddenly introduced itself, tugging Ludmila sideways as her sleeve was yanked along by her maid. It took a moment to untangle herself before she stood to speak to the waiting Skeleton.

 

“What are – wait,” she looked from the Skeleton in front of her to the Skeletons crewing the ship. “You Skeletons can speak?”

 

She thought of all the time she had just spent instructing the new captain. Even a little bit of verbal feedback would have been nice.

 

“You Skeletons?” Its voice turned indignant, drawing her attention to the boxed Skeleton again, “I am not as they!”

 

Its posture was the very image of affront. Ludmila examined the strangely animated Undead being.

 

“Then what are you?” she asked. “What is your name?”

 

“My name? My name…” It pondered her question for a moment before responding, “Well, if you must call me by a name, then that name would be Jeeves. As for what I am…well, I am Jeeves – gentleman and butler extraordinaire!”

 

Jeeves turned his head up proudly as he declared this, tucking his left arm up against his puffed-up ribcage.

 

“You’re a butler?” Ludmila was dubious.

 

“Indeed, madam!” Jeeves replied, his head still held high, “If it is within my ability to provide, then I shall most gladly oblige.”

 

“Well then, Jeeves,” she was still very much skeptical, “what is it that you do?”

 

“I provide service for all that would have need of me, of course!” He continued in his upbeat tone, “Adventurers, explorers, prospectors, soldiers and merchants – all those and more! From the frozen rivers of Nifelheim to the fiery fields of Muspelheim my kind come to their call: to relieve them of their burdens; offering refreshment and the refurbishment of their battered arms.”

 

Never mind loosely following Ludmila’s knowledge of what a Butler was, Jeeves had headed straight off into a totally unfamiliar direction. She had never heard of the places he had mentioned before, but the services he offered seemed more that of a peddler than that of a butler. She decided to put the latter part of his claim to the test.

 

“I would like to browse your goods,” she told him.

 

“Certainly!”

 

A full minute passed while the two women waited for Jeeves to do something.

 

“Well?” Ludmila asked the ‘butler’ that was standing stiffly in front of them.

 

“That’s…” a hint of panic entered Jeeves’ formerly proud voice. “I’m sure there’s…”

 

Jeeves made a series of gestures, stopping after each as if he expected something to happen. Then he turned around and leaned over to poke his head into his box and abruptly brought it back up again.

 

“My inventory! It’s gone!” Jeeves’ voice rose with his panic, “What's going on?! Wait – you,” he pointed at Aemilia. “Sell me that basket!”

 

The maid looked askance at the Undead being holding his hands out at her and turned to Ludmila with a question on her face. Her mistress motioned for her to entertain his request, so she hesitantly rose to her feet with the wicker lunch basket in her hands, clearing her throat.

 

“I would like to sell this baske–”

 

“Just give it to me!” Jeeves nearly shouted.

 

“Hieeee!”

 

Aemilia half-threw the basket at Jeeves as she shied away in distress. Jeeves did not seem to care how he had received the item, however, and made a motion as if he was placing a coin into the maid’s outstretched hand. As he turned to put the basket into his box, Ludmila looked at Aemilia’s open palm. There were no coins: silver, copper or otherwise. Jeeves returned to face them, sounding pleased.

 

“Ah, much better,” he said, then looked back at the women, “...what?”

 

“You didn’t pay us anything for the basket.” Ludmila said while she soothed her maid.

 

“That’s preposterous,” he replied. “Do you take me for some thieving cur? Your payment is right there…”

 

His voice trailed off as he looked down to the maid’s empty hand. Ludmila thought if Undead could break out into a cold sweat, now would be about the time that he would be doing so. He suddenly turned around.

 

“Wait, where are you going?”

 

“...if you’ll excuse me,” Jeeves told them, “I’m afraid I have other business to attend to.”

 

The Skeleton crawled back into his box, closing the cover on top of himself. It was not long until a muffled racket rose from the box again.

 

“What! What is this thing doing in here?!”

 

The cover of the box flipped open again, and the wicker basket that had been ‘purchased’ came flying out. Ludmila had to jump out of her seat to catch it before it went overboard.

 

“Why am I still here? I should be gone,” Jeeves’ voice continued to rise from his box. “This isn’t right. Unnatural!”

 

Skeletal fingers grasped the edges of the opening as he pulled himself upright. Half of his skull poked out from the lip of the container, and he scanned his surroundings. The crimson points of his eyes fixed to the railing of the ship and the river beyond.

 

“If I cannot disappear normally,” he said quietly, “then I’ll make myself disappear!”

 

Knocking over his box as he heaved himself back out, Jeeves tumbled out onto the deck, scrabbling on all fours towards the side of the ship. As he grasped the railing to hurl himself into the river, Ludmila reached out and grabbed Jeeves by the spine. She easily held onto him, as he lacked flesh and blood or any heavy equipment.

 

“Eh? No! Unhand me, you evil fiend!” The Undead being flailed about as he was pulled away from the river, “This isn’t right! I shouldn’t exist any more! I want to die! Let me dieeeeeee–”

 

The suicidal Skeleton’s voice was abruptly cut off as he was tossed back into his box and the lid was slammed shut. Ludmila sat on top of the box to make sure Jeeves couldn’t escape; his continued struggling could still be felt through the cover. She noticed the imposing Undead warrior looking in her direction as she kept Jeeves trapped beneath her.

 

“Can Skeletons even drown?” She asked.

 

The captain shook its head; Ludmila sighed. Of course not.

 

Everyone remained silent, save for the struggling Jeeves, as they continued up the river. It was nearly an hour before the movement in the box stopped, but that in itself caused Ludmila to wonder what he was up to. The Undead did not require sleep, nor did they get tired, so there should have been no reason for him to stop along those grounds. Perhaps he had just given up, deciding that struggling was futile. She stood up off of the cover and warily eyed the shiny black container, waiting for him to suddenly pop out in another futile suicide attempt. When nothing happened, she quickly grabbed the box and placed it back in its spot in the hold, wedging one of her bags filled with clothing between the cover and the deck so Jeeves couldn’t open the container and sneak out.

 

Ludmila sat back down and took a deep breath, attempting to reorganize her thoughts. She wanted to review all the materials that were packed away to prepare for the tasks ahead of her in Warden’s Vale, but the entire encounter with Jeeves had thrown her off entirely with its sheer strangeness. Reaching into a bag that she had placed near her seat, she pulled out several files and settled down to look over their contents.

 

Unfortunately, the peace did not last very long. She had barely gotten past a few pages when an odd noise rose from the hold. It only took a moment of her stopping to focus on it before recognizing the muffled voice of Jeeves, steadily growing louder over time. Considering how he had been shut away in his container and packed tightly in the hold, he must have been shouting quite loudly.

 

“There’s no point in existence.” He moaned, “I can’t do anything. I’m useless.”

 

“Why must I live this waking nightmare…”

 

“I always thought ten minutes was a perfectly respectable lifespan. I was right.”

 

“WANTING TO GO OUTSIDE WAS A TERRIBLE IDEA. THE BOX DOES NOT JUDGE. IT JUST HATES.”

 

The sound of his voice rose and fell over the wind, the moans and wails swirling in eddies around the ship and up the banks of the river on either side. It stopped periodically, as if squelched by the benevolent will of the merciful Surshana, only to slowly pick up again over several minutes. If Ludmila had been a tenant watching from one of the vineyards that grew on the gentle slopes of the valley, the sight and sound of this ship being crewed by the Undead would have definitely caused her to believe it to be some cursed vessel sailing up from the Katze Plains.

 

“Ah, to be cradled in the darkness of my cathedral of melancholic oblivion. With naught by my failure and impotence; twin razors to etch exquisite lines of endless agony over my withered soul for the rest of this eternity damned!”

 

Knowing the absolute ridiculousness of the situation, however, all she could think of was how she might be able to stop it before someone actually did witness their passage and run off to spread horrifying rumors that would send all of the citizens scurrying back into their homes. Even Aemilia had developed something between a disgusted and pitying expression rather than the wary one she originally carried upon seeing Jeeves for the first time, which grew even more pronounced with every new line that was carried into the air.

 

“Is there nothing you can do about this?” She asked the Elder Lich, “Every Ogre and Owlbear in the border ranges is going to come down to sink us – if only to end this ungodly noise!”

 

“「Silence」.”

 

The Undead caster did not even bother turning around to cast the spell. All at once, the sound of Jeeves’ mournful voice stopped, and only the sounds of the river and the wind remained.

 

With a sigh of relief, Ludmila settled down once again to get back to her reading. Though she had finalized and submitted her requests for labour the previous day, she still had trouble believing what all the reference materials provided had claimed. The 4000 acres of fallow terraces that she had received Undead labour for would normally take several farming communities months to clear, plough, sow and harrow – it wasn’t an even distribution of labour, either. The preparations leading up to both planting and harvest required far more labour than it took to maintain the fields while they grew and ripened.

 

Almanacs were not unheard of in Re-Estize…but nothing in such specific and concise detail. They usually listed recommendations for seasonal crops, advised on weather conditions and sometimes even promoted the use of magic to aid in boosting production. There were also such considerations as diseases, pests and various projections on yield. The numbers in the one on her lap seemed entirely uniform and arbitrary: as if tests had been done in controlled conditions on a few crops, then added to the book with no further thought. The fact that much of it was left mostly blank beyond headers and fields lent heavily to this idea.

 

That an Undead labourer – namely the Skeletons – had a distribution of 1 per 50 acres meant that one Skeleton was projected to do more than the work of an entire farming household over the course of a season. She thought it might add up, considering that the Undead were tireless and did not travel back and forth from their homes, but there were other factors that went into it and she only understood things from her position as a noble managing a fief – she was no Farmer, and did not know much about the precise details of running a farm.

 

What she did know, however, is that most of the professions that serviced populations of farm tenants would see drastically lowered demand since the Skeletal labour did not eat or need housing, did not need clothing or shoes and did not need apothecaries, priests or recreation. Besides a blacksmith to repair and replace broken equipment, they did not need much of anything in the way of supporting industries.

 

It was a boon in the sense that she did not immediately have any villagers that filled those roles and nearly all of the produce would become an export of the barony. It was also worrisome, however, because those very same industries and the populations that they would service also gave rise to more specialized labour that she wanted to promote in the future. Basic industries usually paved the way for more advanced industries, and going backwards was impossible unless one had a spectacular amount of wealth to charter a town or a city.

 

The yield projections provided were the most boggling aspect of the entire thing. The almanac gave a range of 15 to 20 bushels per acre for oats, using the regular methods of regional farming tenants. If the prescribed methods were used and druidic magic was applied, it gave an estimate which increased the yield by fifty percent with a guarantee of the upper range – barring something happening to physically destroy the crops. The regular amount of 15 to 20 bushels per acre was already considered a bumper crop in the best regions of Re-Estize; being able to achieve a harvest of 30 bushels beggared the imagination.

 

The evening that Yuri Alpha had presented her with the materials, she had stumbled across this information and, no matter how many times she tabulated the figures, the numbers came out the same in the end. If conditions were good and she managed both a midsummer and late autumn harvest, the current allotment of terraces in her small, undeveloped territory would yield a total of 240,000 bushels of oats annually. Assuming 3 bushels per acre were retained for seeding the next crop, it was still 216,000 bushels per annum.

 

The amount was enough grain to feed over 13,000 adults for an entire year – the population of a small city. Warden’s Vale used to have slightly over one hundred, and now it would have only two, so it would not be unfair to say that the entire harvest would basically be sold at market if things remained as they were. She shook her head, trying to imagine what sort of chaos it would create when the nobles with vast agricultural lands adopted the new methods of cultivation.

 

The ridiculous quantities of produce would enter the markets and the valuations in those markets would implode catastrophically, making basic staples so cheap that small territories like her own would not be able to generate a reasonable income by exporting them. It was the entire reason she was so confident sharing her thoughts with Lady Shalltear, and also the reason she was already starting to feel desperate for ways to shift the main products of her demesne to something other than what Undead labour could provide cheaply.


Still, that was only if what was written on paper could be proven in practice. In order to figure out all that was required to achieve the projected figures, she would need to produce at least one harvest. Before that, she wanted to at least have a small framework of tenants in her demesne to manage the various tasks associated with directing the Undead labourers, managing inventories and to lay the groundwork for future development. She could not do everything herself, nor did she consider it an efficient use of her time in the long run. Aemilia would be busy around the manor and village and she had no idea when her ‘attaché’ would abruptly detach itself from her to return to the central administration. With this in mind, her thoughts inevitably slid back to the assistance provided by the Royal Treasurer.